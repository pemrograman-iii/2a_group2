%%%%%%%%%%%%%%
%% Run LaTeX on this file several times to get Table of Contents,
%% cross-references, and citations.

%% If you have font problems, you may edit the w-bookps.sty file
%% to customize the font names to match those on your system.

%% w-bksamp.tex. Current Version: Feb 16, 2012
%%%%%%%%%%%%%%%%%%%%%%%%%%%%%%%%%%%%%%%%%%%%%%%%%%%%%%%%%%%%%%%%
%
%  Sample file for
%  Wiley Book Style, Design No.: SD 001B, 7x10
%  Wiley Book Style, Design No.: SD 004B, 6x9
%
%
%  Prepared by Amy Hendrickson, TeXnology Inc.
%  http://www.texnology.com
%%%%%%%%%%%%%%%%%%%%%%%%%%%%%%%%%%%%%%%%%%%%%%%%%%%%%%%%%%%%%%%%

%%%%%%%%%%%%%
% 7x10
%\documentclass{wileySev}

% 6x9
\documentclass{wileySix}

\usepackage{graphicx}
\usepackage{listings}
\usepackage{float}

\usepackage{color}

\definecolor{codegreen}{rgb}{0,0.6,0}
\definecolor{codegray}{rgb}{0.5,0.5,0.5}
\definecolor{codepurple}{rgb}{0.58,0,0.82}
\definecolor{backcolour}{rgb}{0.95,0.95,0.92}

\lstdefinestyle{mystyle}{
    backgroundcolor=\color{backcolour},
    commentstyle=\color{codegreen},
    keywordstyle=\color{magenta},
    numberstyle=\tiny\color{codegray},
    stringstyle=\color{codepurple},
    basicstyle=\footnotesize,
    breakatwhitespace=false,
    breaklines=true,
    captionpos=b,
    keepspaces=true,
    numbers=left,
    numbersep=5pt,
    showspaces=false,
    showstringspaces=false,
    showtabs=false,
    tabsize=2,
    language=sh
}

\lstset{style=mystyle}

%%%%%%%
%% for times math: However, this package disables bold math (!)
%% \mathbf{x} will still work, but you will not have bold math
%% in section heads or chapter titles. If you don't use math
%% in those environments, mathptmx might be a good choice.

% \usepackage{mathptmx}

% For PostScript text
\usepackage{w-bookps}

%%%%%%%%%%%%%%%%%%%%%%%%%%%%%%%%%%%%%%%%%%%%%%%%%%%%%%%%%%%%%%%%
%% Other packages you might want to use:

% for chapter bibliography made with BibTeX
% \usepackage{chapterbib}

% for multiple indices
% \usepackage{multind}

% for answers to problems
% \usepackage{answers}

%%%%%%%%%%%%%%%%%%%%%%%%%%%%%%
%% Change options here if you want:
%%
%% How many levels of section head would you like numbered?
%% 0= no section numbers, 1= section, 2= subsection, 3= subsubsection
%%==>>
\setcounter{secnumdepth}{3}

%% How many levels of section head would you like to appear in the
%% Table of Contents?
%% 0= chapter titles, 1= section titles, 2= subsection titles,
%% 3= subsubsection titles.
%%==>>
\setcounter{tocdepth}{2}

%% Cropmarks? good for final page makeup
%% \docropmarks

%%%%%%%%%%%%%%%%%%%%%%%%%%%%%%
%
% DRAFT
%
% Uncomment to get double spacing between lines, current date and time
% printed at bottom of page.
% \draft
% (If you want to keep tables from becoming double spaced also uncomment
% this):
% \renewcommand{\arraystretch}{0.6}
%%%%%%%%%%%%%%%%%%%%%%%%%%%%%%

%%%%%%% Demo of section head containing sample macro:
%% To get a macro to expand correctly in a section head, with upper and
%% lower case math, put the definition and set the box
%% before \begin{document}, so that when it appears in the
%% table of contents it will also work:

\newcommand{\VT}[1]{\ensuremath{{V_{T#1}}}}

%% use a box to expand the macro before we put it into the section head:

\newbox\sectsavebox
\setbox\sectsavebox=\hbox{\boldmath\VT{xyz}}

%%%%%%%%%%%%%%%%% End Demo


\begin{document}


\booktitle{Cerdas Menguasai Python}
\subtitle{Dalam 24 Jam}

\authors{Rolly M. Awangga\\
\affil{Informatics Research Center}
%Floyd J. Fowler, Jr.\\
%\affil{University of New Mexico}
}

\offprintinfo{Cerdas Menguasai Python, First Edition}{Rolly M. Awangga}

%% Can use \\ if title, and edition are too wide, ie,
%% \offprintinfo{Survey Methodology,\\ Second Edition}{Robert M. Groves}

%%%%%%%%%%%%%%%%%%%%%%%%%%%%%%
%%
\halftitlepage

%\titlepage


\begin{copyrightpage}{2019}
\input{info/copyrightpage}
\end{copyrightpage}

\dedication{`Jika Kamu tidak dapat menahan lelahnya belajar,
Maka kamu harus sanggup menahan perihnya Kebodohan.'
~Imam Syafi'i~}

\begin{contributors}
\input{info/contributors}
\end{contributors}

\contentsinbrief
\tableofcontents
\listoffigures
\listoftables
\lstlistoflistings


\begin{foreword}
\input{info/foreword}
\end{foreword}

\begin{preface}
\input{info/preface}
\end{preface}


\begin{acknowledgments}
\input{info/acknowledgments}
\end{acknowledgments}

\begin{acronyms}
\input{info/acronyms}
\end{acronyms}

\begin{glossary}
\input{info/glossary}
\end{glossary}

\begin{symbols}
\input{info/symbols}
\end{symbols}

\begin{introduction}
\input{info/introduction}
\end{introduction}

%%%%%%%%%%%%%%%%%%Isi Buku_
%TEORI
%\chapter{Judul Bagian Pertama}
%\input{chapters/Teori/1}
%PRAKTEK
%\chapter{Judul Bagian Pertama}
%\input{chapters/Praktek/1}

%TEORI
%\chapter{Judul Bagian Pertama}
%\input{chapters/Teori/2}
%PRAKTEK
%\chapter{Judul Bagian Pertama}
%\input{chapters/Praktek/2}

%TEORI
%\chapter{Judul Bagian Pertama}
%\section{Muhammad Tomy Nur Maulidy}
{\Large \textbf{Pemahaman Teori}}
\subsection{Soal No. 1}
Apa itu fungsi, inputan fungsi dan kembalian fungsi dengan contoh kode program lainnya.

\hfill \break
Fungsi memiliki tujuan agar kita dapat memecah program besar menjadi sub-sub program yang lebih sederhana.pada masing-masing  fitur pada program dapat dibuat dalam satu fungsi. Pada saat kita membutuhkan suatu fitur maka kita tinggal memanggil fungsi yang telah kita buat. Fungsi pada python dibuat dengan menggunakan kata kunci def dan diikuti dengan nama fungsi yang telah kita buat seperti contoh dibawah ini :
 \lstinputlisting[firstline=10, lastline=10]{src/3/1174031/chapter3/1174031.py}
Inputan fungsi merupakan masukan yang kita berikan pada program dan program akan menampilkan hasil dari inputan yang telah kita masukkan atau akan menampilkan hasil pada proses selanjutnya. contoh dari inputan fungsi sebagai berikut :
 \lstinputlisting[firstline=11, lastline=11]{src/3/1174031/chapter3/1174031.py}
Pengembalian fungsi memiliki tujuan untuk mengembalikan nilai dari hasil yang telah di proses. Dalam hal ini menggunakan kata kunci return yang diikuti dengan nilai atau variabel yang akan dikembalikan.
 \lstinputlisting[firstline=10, lastline=14]{src/3/1174031/chapter3/1174031.py}

\subsection{Soal 2}
Apa itu paket dan cara pemanggilan paket atau library dengan contoh kode program lainnya.

\hfill \break
Library atau paket adalah modul-modul yang menyusun python. Modul-modul tersebut ditulis oleh berbagai orang dari seluruh dunia dan memiliki fungsi masing-masing untuk melakukan suatu hal. contoh kode programnya adalah sebagai berikut :
 \lstinputlisting[firstline=17, lastline=18]{src/3/1174031/chapter3/1174031.py}

\subsection{Soal 3}
Jelaskan Apa itu kelas, apa itu objek, apa itu atribut, apa itu method dan contoh kode program lainnya masing-masing.

\hfill \break
kelas adalah Prototype atau blueprint untuk menciptakan suatu object  yang mendefinisikan seperangkat atribut yang menjadi ciri objek kelas apa pun. Objek ialah instansiasi atau perwujudan dari sebuah kelas. Bila kelas adalah prototipenya, dan objek adalah hasil dari class jadinya. Atribut merupakan data dari anggota (variabel kelas, variabel contoh) dan metode, yang diakses dengan notasi titik. Sedangkan method fungsi yang didefinisikan di dalam suatu kelas.
 \lstinputlisting[firstline=21, lastline=40]{src/3/1174031/chapter3/1174031.py}

\subsection{Soal 4}
Jelaskan cara pemanggilan library kelas dari instansiasi dan pemakaiannya dengan contoh program lainnya.

\hfill \break
cara pemanggilan  library kelas dari instansiasi dan pemakaiannya adalah dengan cara meng-import library yang ada di dalam satu folder dengan menggunakan kode berikut :
 \lstinputlisting[firstline=43, lastline=51]{src/3/1174031/chapter3/1174031.py}

\subsection{Soal 5}
Jelaskan dengan contoh pemakaian paket dengan perintah from kalkulator import Penambahan disertai dengan contoh kode lainnya.

\hfill \break
contoh kodenya adalah sebagai berikut :
 \lstinputlisting[firstline=54, lastline=57]{src/3/1174031/chapter3/1174031.py}

\subsection{Soal 6}
Jelaskan dengan contoh kodenya, pemakaian paket fungsi apabila file library ada di dalam folder.

\hfill \break
 Pemakaian paket adalah perkumpulan fungsi-fungsi. contoh kodenya adalah sebagai berikut :
 \lstinputlisting[firstline=60, lastline=73]{src/3/1174031/chapter3/1174031.py}

\subsection{Soal 7}
Jelaskan dengan contoh kodenya, pemakaian paket kelas apabila file library ada di dalam folder.

\hfill \break
 \lstinputlisting[firstline=76, lastline=84]{src/3/1174031/chapter3/1174031.py}


\section{Dwi Yulianingsih}
\subsection{Soal 1}
Isi jawaban soal ke-1

Kalau mau dibikin paragrap \textbf{cukup enter aja}, tidak usah pakai \verb|par| dsb

%\subsection{Soal 2}
%Isi jawaban soal ke-2

%\subsection{Soal 3}
%Isi jawaban soal ke-3

\section{Harun Ar-Rasyid}
\subsection{Soal 1}
Isi jawaban soal ke-1

Kalau mau dibikin paragrap \textbf{cukup enter aja}, tidak usah pakai \verb|par| dsb

%\subsection{Soal 2}
%Isi jawaban soal ke-2

%\subsection{Soal 3}
%Isi jawaban soal ke-3

\section{Sri Rahayu}
\subsection{Soal 1}
Isi jawaban soal ke-1

Kalau mau dibikin paragrap \textbf{cukup enter aja}, tidak usah pakai \verb|par| dsb

%\subsection{Soal 2}
%Isi jawaban soal ke-2

%\subsection{Soal 3}
%Isi jawaban soal ke-3

\section{Doli Jonviter}
\subsection{Soal 1}
Isi jawaban soal ke-1

Kalau mau dibikin paragrap \textbf{cukup enter aja}, tidak usah pakai \verb|par| dsb

%\subsection{Soal 2}
%Isi jawaban soal ke-2

%\subsection{Soal 3}
%Isi jawaban soal ke-3

\section{Rahmatul Ridha}
\subsection{Soal 1}
Isi jawaban soal ke-1

Kalau mau dibikin paragrap \textbf{cukup enter aja}, tidak usah pakai \verb|par| dsb

%\subsection{Soal 2}
%Isi jawaban soal ke-2

%\subsection{Soal 3}
%Isi jawaban soal ke-3

\section{Tomy Prawoto}
\subsection{Soal 1}
Isi jawaban soal ke-1

Kalau mau dibikin paragrap \textbf{cukup enter aja}, tidak usah pakai \verb|par| dsb

%\subsection{Soal 2}
%Isi jawaban soal ke-2

%\subsection{Soal 3}
%Isi jawaban soal ke-3

%PRAKTEK
%\chapter{Judul Bagian Pertama}
%\section{Muhammad Tomy Nur Maulidy}
{\Large \textbf{Praktek}}
\subsection{Soal No. 1}

\hfill \break
 \lstinputlisting[firstline=87, lastline=121]{src/3/1174031/chapter3/1174031.py}

\subsection{Soal 2}

\hfill \break
\lstinputlisting[firstline=124, lastline=129]{src/3/1174031/chapter3/1174031.py}

\subsection{Soal 3}

\hfill \break
 \lstinputlisting[firstline=132, lastline=139]{src/3/1174031/chapter3/1174031.py}

\subsection{Soal 4}

\hfill \break
 \lstinputlisting[firstline=142, lastline=146]{src/3/1174031/chapter3/1174031.py}

\subsection{Soal 5}

\hfill \break
 \lstinputlisting[firstline=149, lastline=153]{src/3/1174031/chapter3/1174031.py}

\subsection{Soal 6}

\hfill \break
 \lstinputlisting[firstline=156, lastline=161]{src/3/1174031/chapter3/1174031.py}

\subsection{Soal 7}

\hfill \break
 \lstinputlisting[firstline=164, lastline=169]{src/3/1174031/chapter3/1174031.py}

 \subsection{Soal 8}

\hfill \break
 \lstinputlisting[firstline=172, lastline=178]{src/3/1174031/chapter3/1174031.py}

 \subsection{Soal 9}

\hfill \break
 \lstinputlisting[firstline=181, lastline=186]{src/3/1174031/chapter3/1174031.py}

 \subsection{Soal 10}

\hfill \break
 \lstinputlisting[firstline=189, lastline=204]{src/3/1174031/chapter3/1174031.py}

 \subsection{Soal 11}

\hfill \break
 \lstinputlisting[firstline=8, lastline=22]{src/3/1174031/chapter3/main.py}

 \subsection{Soal 12}

\hfill \break
 \lstinputlisting[firstline=24, lastline=39]{src/3/1174031/chapter3/main.py}

\section{Dwi Yulianingsih}
\subsection{Soal 1}
Isi jawaban soal ke-1

Kalau mau dibikin paragrap \textbf{cukup enter aja}, tidak usah pakai \verb|par| dsb

%\subsection{Soal 2}
%Isi jawaban soal ke-2

%\subsection{Soal 3}
%Isi jawaban soal ke-3

\section{Harun Ar-Rasyid}
\subsection{Soal 1}
Isi jawaban soal ke-1

Kalau mau dibikin paragrap \textbf{cukup enter aja}, tidak usah pakai \verb|par| dsb

%\subsection{Soal 2}
%Isi jawaban soal ke-2

%\subsection{Soal 3}
%Isi jawaban soal ke-3

\section{Sri Rahayu}
\subsection{Soal 1}
Isi jawaban soal ke-1

Kalau mau dibikin paragrap \textbf{cukup enter aja}, tidak usah pakai \verb|par| dsb

%\subsection{Soal 2}
%Isi jawaban soal ke-2

%\subsection{Soal 3}
%Isi jawaban soal ke-3

\section{Doli Jonviter}
\subsection{Soal 1}
Isi jawaban soal ke-1

Kalau mau dibikin paragrap \textbf{cukup enter aja}, tidak usah pakai \verb|par| dsb

%\subsection{Soal 2}
%Isi jawaban soal ke-2

%\subsection{Soal 3}
%Isi jawaban soal ke-3

\section{Rahmatul Ridha}
\subsection{Soal 1}
Isi jawaban soal ke-1

Kalau mau dibikin paragrap \textbf{cukup enter aja}, tidak usah pakai \verb|par| dsb

%\subsection{Soal 2}
%Isi jawaban soal ke-2

%\subsection{Soal 3}
%Isi jawaban soal ke-3

\section{Tomy Prawoto}
\subsection{Soal 1}
Isi jawaban soal ke-1

Kalau mau dibikin paragrap \textbf{cukup enter aja}, tidak usah pakai \verb|par| dsb

%\subsection{Soal 2}
%Isi jawaban soal ke-2

%\subsection{Soal 3}
%Isi jawaban soal ke-3


%TEORI
\chapter{Library CSV dan Pandas}
%\input{chapters/Teori/4}
%PRAKTEK
\chapter{Praktek Library CSV dan Pandas}
\input{chapters/Praktek/4}

%TEORI
\chapter{PySerial}
%\input{chapters/Teori/5}
%PRAKTEK
\chapter{Praktek PySerial}
%\input{chapters/Praktek/5}

\chapter{Matplotlib}
\chapter{Matplotlib}
%\input{chapters/6/1174006}
%\input{chapters/6/1174031}
%\input{chapters/6/1174012}
%\input{chapters/6/1174017}
%\input{chapters/6/1174026}
%\input{chapters/6/1174003}
%\input{chapters/6/1174021}
%\input{chapters/6/1174xxx}
\bibliographystyle{IEEEtran}
%\def\bibfont{\normalsize}
\bibliography{references}


%%%%%%%%%%%%%%%
%%  The default LaTeX Index
%%  Don't need to add any commands before \begin{document}
\printindex

%%%% Making an index
%%
%% 1. Make index entries, don't leave any spaces so that they
%% will be sorted correctly.
%%
%% \index{term}
%% \index{term!subterm}
%% \index{term!subterm!subsubterm}
%%
%% 2. Run LaTeX several times to produce <filename>.idx
%%
%% 3. On command line, type  makeindx <filename> which
%% will produce <filename>.ind
%%
%% 4. Type \printindex to make the index appear in your book.
%%
%% 5. If you would like to edit <filename>.ind
%% you may do so. See docs.pdf for more information.
%%
%%%%%%%%%%%%%%%%%%%%%%%%%%%%%%

%%%%%%%%%%%%%% Making Multiple Indices %%%%%%%%%%%%%%%%
%% 1.
%% \usepackage{multind}
%% \makeindex{book}
%% \makeindex{authors}
%% \begin{document}
%%
%% 2.
%% % add index terms to your book, ie,
%% \index{book}{A term to go to the topic index}
%% \index{authors}{Put this author in the author index}
%%
%% \index{book}{Cows}
%% \index{book}{Cows!Jersey}
%% \index{book}{Cows!Jersey!Brown}
%%
%% \index{author}{Douglas Adams}
%% \index{author}{Boethius}
%% \index{author}{Mark Twain}
%%
%% 3. On command line type
%% makeindex topic
%% makeindex authors
%%
%% 4.
%% this is a Wiley command to make the indices print:
%% \multiprintindex{book}{Topic index}
%% \multiprintindex{authors}{Author index}

\end{document}

