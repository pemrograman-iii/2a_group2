\section{Kadek Diva Krishna Murti}
\subsection{Soal 1}
\textbf{Pengenalan CSV}

Comma Separated Values (CSV) adalah suatu format data yang di mana setiap bagian data dipisahkan dengan tanda koma (,). Format CSV biasanya berfungsi untuk menukar atau mengonversi data ke format lainnya 
%\cite{shafranovich2005common}.

\textbf{Sejarah Format CSV}

IBM Fortran (level H extended) compiler di bawah OS/360 mendukung format CSV pada tahun 1972. FORTRAN 77 mendefinisakan penulisannya dimana input atau output penulisannya menggunakan tanda koma atau spasi untuk pembatas antar data dan penulisan tersebut telah disetujui pada tahun 1978.

Osborne Executive computer yang mengembangkan SuperCalc spreadsheet pada tahun 1983 membuat konvensi kutipan CSV yang memungkinkan string mengandung koma.

Inisiatif standardisasi utama - mentransformasikan "definisi fuzzy de facto" menjadi definisi yang lebih tepat dan de jure - adalah pada tahun 2005, dengan RFC4180, mendefinisikan CSV sebagai Tipe Konten MIME. Kemudian, pada 2013, beberapa kekurangan RFC4180 ditangani oleh rekomendasi W3C.

Pada 2014 IETF menerbitkan RFC7111 yang menjelaskan aplikasi fragmen URI pada dokumen CSV. RFC7111 menentukan bagaimana rentang baris, kolom, dan sel dapat dipilih dari dokumen CSV menggunakan indeks posisi.

Pada 2015 W3C, dalam upaya meningkatkan CSV dengan semantik formal, mempublikasikan draft rekomendasi pertama untuk standar metadata CSV, yang dimulai sebagai rekomendasi pada bulan Desember tahun yang sama.

\textbf{Contoh penggunaan format CSV}

\lstinputlisting[caption = Contoh penggunaan format CSV., firstline=1, lastline=3]{src/4/1174006/Teori/teori.csv}

\subsection{Soal 2}
Aplikasi-aplikasi yang dapat menciptkan file csv, yaitu:

\begin{enumerate}
	\item Editor teks (Notepad, Sublime, Atom, dan lain-lain)
	\item Spreadsheet (Microsoft Excel dan lain-lain)
\end{enumerate}

\subsection{Soal 3}
Cara menulis dan membaca file csv di excel atau spreadsheet, sebagai berikut:

\textbf{Menulis File CSV}

\begin{enumerate}
	\item Pertama silahkan buka aplikasi Excel dengan cara klik ''Start'', cari Excel, kemudian tekan Enter.
	
	\begin{figure}[H]
		\includegraphics[width=9cm]{figures/4/1174006/Teori/t1.png}
		\centering
	\end{figure}
	
	\item Setelah aplikasi terbuka silahkan klik ''Blank Workbook''.
	
	\begin{figure}[H]
		\includegraphics[width=10cm]{figures/4/1174006/Teori/t2.png}
		\centering
	\end{figure}
	
	\item Kemudian isi sesuai dengan data yang ingin dibuat.
	
	\begin{figure}[H]
		\includegraphics[width=10cm]{figures/4/1174006/Teori/t3.png}
		\centering
	\end{figure}
	
	\item Setelah selesai dibuat, silahkan simpan file tersebut dengan cara mengklik ''File'', lalu klik ''Save''.
	
	\begin{figure}[H]
		\includegraphics[width=10cm]{figures/4/1174006/Teori/t4.png}
		\centering
	\end{figure}
	
	\item Kemudian isi kolom ''File name'' dengan nama file anda dan kolom ''Save as type'' pilih yang berekstensi .csv.
	
	\begin{figure}[H]
		\includegraphics[width=9cm]{figures/4/1174006/Teori/t5.png}
		\centering
	\end{figure}
	
	\item Lalu tinggal klik ''Yes''.
	
	\begin{figure}[H]
		\includegraphics[width=7cm]{figures/4/1174006/Teori/t6.png}
		\centering
	\end{figure}
	
	\item Kemudian file yang Anda telah terbuat tadi tersimpan dengan ekstensi .csv. Untuk melihat isi filenya tinggal klik dua kali pada file tersebut.
	
	\begin{figure}[H]
		\includegraphics[width=10cm]{figures/4/1174006/Teori/t8.png}
		\centering
	\end{figure}
	
	\item Berikut ini adalah isi dari file yang tadi Anda buat.
	
	\begin{figure}[H]
		\includegraphics[width=8cm]{figures/4/1174006/Teori/t7.png}
		\centering
	\end{figure}
\end{enumerate}

\textbf{Melihat File CSV di Excel atau Spreadsheet}

\begin{enumerate}
	\item Pertama klik dua kali pada file yang yang berekstensi CSV.
	
	\begin{figure}[H]
		\includegraphics[width=10cm]{figures/4/1174006/Teori/t8.png}
		\centering
	\end{figure}
	
	\item Kemudian file akan terbuka secara otomatis di aplikasi Excel atau spreadsheet.
	
	\begin{figure}[H]
		\includegraphics[width=10cm]{figures/4/1174006/Teori/t9.png}
		\centering
	\end{figure}
\end{enumerate}

\subsection{Soal 4}
Sejarah library csv

Library csv mengimplementasikan kelas untuk membaca dan menulis data tabular dalam format CSV. Hal ini memungkinkan programmer untuk mengatakan, "tulis data ini dalam format yang disukai oleh Excel," atau "baca data dari file ini yang dihasilkan oleh Excel," tanpa mengetahui detail yang tepat dari format CSV yang digunakan oleh Excel. Pemrogram juga dapat menggambarkan format CSV yang dipahami oleh aplikasi lain atau menentukan format CSV tujuan khusus mereka sendiri.

\subsection{Soal 5}
Sejarah library pandas

Pada 2008, pengembangan pandas dimulai di AQR Capital Management. Pada akhir 2009 telah menjadi open source, dan secara aktif didukung hari ini oleh komunitas individu yang berpikiran sama di seluruh dunia yang menyumbangkan waktu dan energi berharga mereka untuk membantu membuat panda open source menjadi mungkin.

Sejak 2015, pandas adalah proyek yang disponsori NumFOCUS. Ini akan membantu memastikan keberhasilan pengembangan panda sebagai proyek sumber terbuka kelas dunia.

\subsection{Soal 6}
Fungsi-fungsi yang terdapat di library csv, yaitu:
\begin{enumerate}
	\item reader
	
	Fungsi ini digunakan untuk membaca isi file berformat CSV dari list.
	
	\lstinputlisting[caption = Membaca file berformat CSV list., firstline=7, lastline=13]{src/4/1174006/Teori/1174006.py}
	
	\item DictReader
	
	Fungsi ini digunakan untuk membaca isi file berformat CSV dari dictionary.
	
	\lstinputlisting[caption =  Membaca file berformat CSV dictionary., firstline=15, lastline=21]{src/4/1174006/Teori/1174006.py}
	
	\item write
	
	Fungsi ini digunakan untuk menulis file berformat CSV dari list.
	
	\lstinputlisting[caption =  Menulis file berformat CSV list., firstline=23, lastline=30]{src/4/1174006/Teori/1174006.py}
	
	\item DictWrite
	
	Fungsi ini digunakan untuk menulis file berformat CSV dari dictionary.
	
	\lstinputlisting[caption =  Menulis file berformat CSV dictionary., firstline=32, lastline=41]{src/4/1174006/Teori/1174006.py}
	
\end{enumerate}

\subsection{Soal 7}
Fungsi-fungsi yang terdapat di library pandas, yaitu:
\begin{enumerate}
	\item read\_csv
	
	Fungsi ini digunakan untuk membaca isi file berformat CSV
	
	\lstinputlisting[caption =  Membaca file berformat CSV pandas., firstline=43, lastline=47]{src/4/1174006/Teori/1174006.py}
	
	\item to\_csv
	
	Fungsi ini digunakan untuk menulis file berformat CSV
	
	\lstinputlisting[caption =  Menulis file berformat CSV pandas., firstline=49, lastline=53]{src/4/1174006/Teori/1174006.py}
	
\end{enumerate}

\subsection{Kode Program Teori}
\begin{figure}[H]
	\includegraphics[width=10cm]{figures/4/1174006/Teori/kode_teori1.png}
	\centering
\end{figure}

\begin{figure}[H]
	\includegraphics[width=10cm]{figures/4/1174006/Teori/kode_teori2.png}
	\centering
\end{figure}

\subsection{Cek Plagiat Teori}

\begin{figure}[H]
	\includegraphics[width=10cm]{figures/4/1174006/Teori/plagiat_teori.png}
	\centering
\end{figure}


\section{Damara Benedikta}
\subsection{Soal 1}
 CSV (Comma Separated Value) merupakan suatu  format basis data sederhana yang dimana setiap record yang ada dipisahkan dengan tanda koma (,) atau titik koma (;). Format data file csv dapat diolah dengan berbagai text editor dengan mudah. Anda tidak perlu (dan Anda tidak akan) membuat pengurai CSV Anda sendiri dari awal. Ada beberapa perpustakaan yang dapat diterima yang dapat Anda gunakan. Pustaka csv Python akan berfungsi untuk sebagian besar kasus. Jika pekerjaan Anda memerlukan banyak data atau analisis numerik, panda library juga memiliki kemampuan penguraian CSV, yang seharusnya menangani sisanya. Dalam bahasa pemrograman Python telah disediakan modul csv yang khusus untuk mengolah data berformat csv.  Untuk memanipulasi data csv dengan python tentunya yang pertama dilakukan adalah mengimport modul csv dengan perintah import csv. File CSV biasanya dibuat oleh program yang menangani sejumlah besar data. Mereka adalah cara yang nyaman untuk mengekspor data dari spreadsheet dan basis data serta mengimpor atau menggunakannya dalam program lain. Misalnya, Anda dapat mengekspor hasil program penambangan data ke file CSV dan kemudian mengimpornya ke dalam spreadsheet untuk menganalisis data, menghasilkan grafik untuk presentasi, atau menyiapkan laporan untuk publikasi. Contoh nya adalah sebagai berikut :

 \lstinputlisting[firstline=8, lastline=20]{src/4/1174012/Teori/damdam.py}

\subsection{Soal 2} 
 Ada beberapa aplikasi yang dapat menciptakan file dengan format csv diantaranya google sheet, number di MacOS dan microsoft excel.

\subsection{Soal 3}
 Cara membuat file csv di excel cukup mudah yaitu :
\begin{itemize}
	\item Buat foldernya
	\item Pilih save as
	\item pilih file dengan format csv
\end{itemize}
Cara membaca file di csv :
\begin{itemize}
	\item Klik data - get external data - form text
	\item Akan muncul Text Import Wizard, arahkan pada file csv yang ingin anda buka lalu Open.
	\item Setelah File terbuka, akan muncul Text Import Wizard.
	\item Pilih Delimited, Kemudian Next (Di sini, bisa juga menentukan baris awal yang akan di import)
	\item Centrang pada Tab dan Comma (Atau sesuai pengaturan File Anda) lalu Next.
	\item Atur Format data pada tiap kolom yang tampil dan klik Finish
\end{itemize}

\subsection{Soal 4}
 CSV digunakan untuk memudahkan data science dan analis karena dinilai terdapat banyak kemudahan yang diperoleh. CSV dapat dimaksimalkan jika dipaduka dengan python karena python adalah bahasa pemrograman yang support ke banyak library termasuk csv. Maka karena itulah perpaduan python dan csv seringkali digunakan oleh perusahaan-perushaan besar dalam mengolah datanya.

\subsection{Soal 5}
Pandas merupakan sebuah tool yang dapat digunakan sebagai alat analisis data dan struktur untuk bahasa pemrograman Python. Pandas dapat mengolah data dengan mudah, salah satu fitur yang ada dalam pandas adalah Dataframe. Fitur dataframe dapat membaca sebuah file dan menjadikannya tabble, juga dapat mengolah suatu data dengan menggunakan operasi seperti join, group by dan teknik lainnya yang terdapat pada SQL. Dalam hal ini pandas tidak jauh beda dengan csv yaitu memiliki keunggulan dalam pengolahan data-data besar dan dapat disupport dengan baik dengan python walaupun mengimport data dalam jumlah banyak.

\subsection{Soal 6}
 Library csv memiliki keunggula-keunggulan dibandingkan format data lainnya merupakah soal kompatibilitas. File csv dapat digunakan, diolah, diekspor/impor, dan dimodifikasi menggunakan berbagai macam perangkat lunak dan bahasa pemrograman. Pada library csv mempunyai fungsi import dan eksport data yang baik dan bisa digunakan dalam jumlah besar.

\subsection{Soal 7}
pandas menyediakan beberapa fungsi operasi untuk mengolah data. Contoh jika menggunakan series bisa mencari nilai max, min, dan mean secara langsung, bahkan juga bisa melakukan operasi perpangkatan pada nilai Series secara langsung.
Pandas dapat mengolah suatu data dan mengolahnya seperti join, distinct, group by, agregasi, dan teknik seperti pada SQL. Hanya saja dilakukan pada tabel yang dimuat dari file ke RAM.


\subsection{bukti bebas plagiarisme}
\begin{figure}[H]
\centering
\includegraphics[width=10cm]{figures/4/1174012/Teori/ss1.png}
\caption{SS Bebas Plagiarisme}
\label{damara}
\end{figure}
%%%%%%%%%%%%%%%%%%%%%%%%%%%%%%%%%%%%%%%%%%%%%%
\section{Felix Setiawan Lase}
\subsection{Soal 1}
\textbf{Pengenalan CSV}

File CSV (Nilai Terbatas Koma) adalah jenis file khusus yang dapat Anda buat atau edit di Excel. File CSV menyimpan informasi yang disimpan dengan koma alih-alih menyimpan informasi dalam kolom.

\textbf{Sejarah Format CSV}

Kompiler Fortran IBM (tingkat lanjut H) di bawah OS / 360 mendukung format CSV pada tahun 1972. FORTRAN 77 mendefinisikan penulisannya di mana penulisan input atau output menggunakan koma atau spasi untuk batas antara data dan penulisan disetujui pada tahun 1978.

Pada 2014 IETF menerbitkan RFC7111 yang menjelaskan penerapan fragmen URI dalam dokumen CSV. RFC7111 menentukan bagaimana berbagai baris, kolom, dan sel dapat dipilih dari dokumen CSV menggunakan indeks posisi.

Pada 2015, W3C, dalam upaya meningkatkan CSV dengan semantik formal, menerbitkan rancangan rekomendasi pertama untuk standar metadata CSV, yang dimulai sebagai rekomendasi pada bulan Desember tahun yang sama.

\textbf{Contoh penggunaan format CSV}

\lstinputlisting[caption = Contoh penggunaan format CSV., firstline=1, lastline=3]{src/4/1174026/Teori/teori.csv}

\subsection{Soal 2}
Aplikasi-aplikasi yang dapat menciptkan file csv, yaitu:

\begin{enumerate}
	\item Editor teks (Notepad, Sublime, Atom, dan lain-lain)
	\item Spreadsheet (Microsoft Excel dan lain-lain)
\end{enumerate}

\subsection{Soal 3}
Cara menulis dan membaca file csv di excel atau spreadsheet, sebagai berikut:

\textbf{Menulis File CSV}

\begin{enumerate}
	\item Pertama silahkan buka aplikasi Excel dengan cara klik ''Start'', cari Excel, kemudian tekan Enter.
	
	\begin{figure}[H]
		\includegraphics[width=9cm]{figures/4/1174026/Teori/t1.png}
		\centering
	\end{figure}
	
	\item Setelah aplikasi terbuka silahkan klik ''Blank Workbook''.
	
	\begin{figure}[H]
		\includegraphics[width=10cm]{figures/4/1174026/Teori/t2.png}
		\centering
	\end{figure}
	
	\item Kemudian isi sesuai dengan data yang ingin dibuat.
	
	\begin{figure}[H]
		\includegraphics[width=10cm]{figures/4/1174026/Teori/t3.png}
		\centering
	\end{figure}
	
	\item Setelah selesai dibuat, silahkan simpan file tersebut dengan cara mengklik ''File'', lalu klik ''Save''.
	
	\begin{figure}[H]
		\includegraphics[width=10cm]{figures/4/1174026/Teori/t4.png}
		\centering
	\end{figure}
	
	\item Kemudian isi kolom ''File name'' dengan nama file anda dan kolom ''Save as type'' pilih yang berekstensi .csv.
	
	\begin{figure}[H]
		\includegraphics[width=9cm]{figures/4/1174026/Teori/t5.png}
		\centering
	\end{figure}
	
	\item Lalu tinggal klik ''Yes''.
	
	\begin{figure}[H]
		\includegraphics[width=7cm]{figures/4/1174026/Teori/t6.png}
		\centering
	\end{figure}
	
	\item Kemudian file yang Anda telah terbuat tadi tersimpan dengan ekstensi .csv. Untuk melihat isi filenya tinggal klik dua kali pada file tersebut.
	
	\begin{figure}[H]
		\includegraphics[width=10cm]{figures/4/1174026/Teori/t8.png}
		\centering
	\end{figure}
	
	\item Berikut ini adalah isi dari file yang tadi Anda buat.
	
	\begin{figure}[H]
		\includegraphics[width=8cm]{figures/4/1174026/Teori/t7.png}
		\centering
	\end{figure}
\end{enumerate}

\textbf{Melihat File CSV di Excel atau Spreadsheet}

\begin{enumerate}
	\item Pertama klik dua kali pada file yang yang berekstensi CSV.
	
	\begin{figure}[H]
		\includegraphics[width=10cm]{figures/4/1174026/Teori/t8.png}
		\centering
	\end{figure}
	
	\item Kemudian file akan terbuka secara otomatis di aplikasi Excel atau spreadsheet.
	
	\begin{figure}[H]
		\includegraphics[width=10cm]{figures/4/1174026/Teori/t9.png}
		\centering
	\end{figure}
\end{enumerate}

\subsection{Soal 4}
Sejarah library csv

Perpustakaan CSV mengimplementasikan kelas untuk membaca dan menulis data tabular dalam format CSV. Ini memungkinkan programmer untuk mengatakan, "tulis data ini dalam format yang disukai Excel," atau "baca data dari file ini yang dihasilkan oleh Excel," tanpa mengetahui detail pasti dari format CSV yang digunakan oleh Excel. Pemrogram juga dapat menggambarkan format CSV yang dimengerti oleh aplikasi lain atau menentukan format CSV spesifik mereka sendiri.
	

\subsection{Soal 5}
Sejarah library pandas

Tahun 2008, pengembangan profesional dimulai di AQR Capital Management. Pada akhir 2009 ini telah menjadi open source, dan secara aktif didukung hari ini oleh komunitas individu yang berpikiran sama di seluruh dunia yang menyumbangkan waktu dan energi berharga mereka untuk membantu membuat panda open source menjadi mungkin.

	Sejak tahun 2015, Pandas adalah proyek yang disponsori oleh NumFOCUS. Ini akan membantu memastikan keberhasilan pengembangan Panda sebagai proyek open source kelas dunia.
	

\subsection{Soal 6}
Fungsi-fungsi yang terdapat di library csv, yaitu:
\begin{enumerate}
	\item reader
	
	Fungsi ini digunakan untuk membaca isi file berformat CSV dari list.
	
	\lstinputlisting[caption = Membaca file berformat CSV list., firstline=7, lastline=13]{src/4/1174026/Teori/1174026.py}
	
	\item DictReader
	
	Fungsi ini digunakan untuk membaca isi file berformat CSV dari dictionary.
	
	\lstinputlisting[caption =  Membaca file berformat CSV dictionary., firstline=15, lastline=21]{src/4/1174026/Teori/1174026.py}
	
	\item write
	
	Fungsi ini digunakan untuk menulis file berformat CSV dari list.
	
	\lstinputlisting[caption =  Menulis file berformat CSV list., firstline=23, lastline=30]{src/4/1174026/Teori/1174026.py}
	
	\item DictWrite
	
	Fungsi ini digunakan untuk menulis file berformat CSV dari dictionary.
	
	\lstinputlisting[caption =  Menulis file berformat CSV dictionary., firstline=32, lastline=41]{src/4/1174026/Teori/1174026.py}
	
\end{enumerate}

\subsection{Soal 7}
Fungsi-fungsi yang terdapat di library pandas, yaitu:
\begin{enumerate}
	\item read\_csv
	
	Fungsi ini digunakan untuk membaca isi file berformat CSV
	
	\lstinputlisting[caption =  Membaca file berformat CSV pandas., firstline=43, lastline=47]{src/4/1174026/Teori/1174026.py}
	
	\item to\_csv
	
	Fungsi ini digunakan untuk menulis file berformat CSV
	
	\lstinputlisting[caption =  Menulis file berformat CSV pandas., firstline=49, lastline=53]{src/4/1174026/Teori/1174026.py}
	
\end{enumerate}

\subsection{Kode Program Teori}
\begin{figure}[H]
	\includegraphics[width=10cm]{figures/4/1174026/Teori/kode_teori1.png}
	\centering
\end{figure}

\begin{figure}[H]
	\includegraphics[width=10cm]{figures/4/1174026/Teori/kode_teori2.png}
	\centering
\end{figure}

\subsection{Cek Plagiat Teori}

\begin{figure}[H]
	\includegraphics[width=10cm]{figures/4/1174026/Teori/plagiat_teori.png}
	\centering
\end{figure}
%%%%%%%%%%%%%%%%%%%%%%%%%%%%%%%%%%%%%%%%%%%%%

\section{Dwi Septiani Tsaniyah}
\subsection{Soal 1}
\textbf{Pengenalan CSV}

\textbf{Sejarah Format CSV}

File CSV (Nilai Berbatas Koma) adalah tipe file khusus yang dapat Anda buat atau edit di Excel. File CSV menyimpan informasi yang dipisahkan oleh koma, bukan menyimpan informasi dalam kolom. Saat teks dan angka disimpan dalam file CSV, mudah untuk memindahkannya dari satu program ke program lain. Misalnya, Anda dapat mengekspor kontak dari Google ke dalam file CSV, kemudian mengimpornya ke Outlook.
Creating Shared Value (CSV) adalah sebuah konsep dalam strategi bisnis yang menekankan pentingnya memasukkan masalah dan kebutuhan sosial dalam perancangan strategi perusahaan. CSV merupakan pengembangan dari konsep tanggung jawab sosial perusahaan (Corporate social responsibility, CSR). Konsep ini pertama kali diperkenalkan oleh Michael Porter dan Mark Kramer pada tahun 2006. Konsep CSV didasari pada ide adanya hubungan interdependen antara bisnis dan kesejahteraan sosial. Porter mengkritik bahwa selama ini bisnis dan kesejahteraan sosial selalu ditempatkan berseberangan. Pebisnis pun rela mengorbankan kesejahteraan sosial demi keuntungan semata, misalnya dengan melakukan proses produksi yang tidak memperhatikan lingkungan atau menciptakan polusi. CSV menekankan adanya peluang untuk membangun keunggulan kompetitif dengan cara memasukan masalah sosial sebagai bahan pertimbangan utama dalam merancang strategi perusahaan.
contoh : Ketika Toyota memperkenalkan Prius, sebuah kendaraan hybrid listrik/bensin, Toyota berhasil mendapatkan keunggulan kompetitif dengan memasarkan sebuah kendaraan yang tidak hanya memberikan keuntungan ekonomis, namun juga berdampak positif bagi lingkugan. Urbi, sebuah perusahaan konstruksi asal Meksiko, mengembangkan pasar perumahan dengan memberikan kredit murah untuk pekerja dengan gaji kecil, Whole Foods Market telah menjadi pemimpin kategori di segmen supermarket dengan menawarkan makanan organik dan alami kepada konsumen yang sadar lingkungan. Perusahaan juga dapat meningkatkan keunggulan kompetitif dengan melakukan investasi di komunitas di mana mereka beroperasi. Nestlé, misalnya, berhubungan sangat dekat dengan Distrik Susu Moga di India, melakukan investasi pada infrastruktur lokal, dan mentransfer teknologi kelas dunia untuk membangun rantai suplai yang kompetitif sekaligus meningkatkan kesejahteraan sosial melalui peningkatan kesehatan masyarakat, pendidikan yang lebih baik, dan pertumbuhan ekonomi.

\subsection{Soal 2}
Aplikasi-aplikasi yang dapat menciptkan file csv, yaitu:

\begin{itemize}
\item Texteditor , Seperti notepad,visual studio code,atom,sublime dan lain sebagainya
\item Program Spreadsheet , Seperti excell,google spreadshare,LibreOfficecalc
\end{itemize}

\subsection{Soal 3}
\begin{enumerate}
\item Cara menulis dan membaca file csv di excel atau spreadsheet, sebagai berikut:
 Ada dua cara untuk mengimpor data dari file teks dengan Excel dapat membukanya di Excel, atau mengimpornya sebagai rentang data eksternal. Untuk mengekspor data dari Excel menjadi file teks, gunakan perintah Simpan Sebagai dan ubah tipe file dari menu menurun.
\item Ada dua format file teks yang biasanya digunakan:
File teks berbatas (.txt), dengan karakter TAB (kode karakter ASCII 009) yang biasanya memisahkan setiap bidang teks.
File teks nilai yang dipisahkan koma (.csv), dengan karakter koma (,) yang biasanya memisahkan setiap bidang teks.
\end{enumerate}

\subsection{Soal 4}
Sejarah library csv

Library csv mengimplementasikan kelas untuk membaca dan menulis data tabular dalam format CSV. Hal ini memungkinkan programmer untuk mengatakan, "tulis data ini dalam format yang disukai oleh Excel," atau "baca data dari file ini yang dihasilkan oleh Excel," tanpa mengetahui detail yang tepat dari format CSV yang digunakan oleh Excel. Pemrogram juga dapat menggambarkan format CSV yang dipahami oleh aplikasi lain atau menentukan format CSV tujuan khusus mereka sendiri.

\subsection{Soal 5}
Sejarah library pandas

Pada 2008, pengembangan pandas dimulai di AQR Capital Management. Pada akhir 2009 telah menjadi open source, dan secara aktif didukung hari ini oleh komunitas individu yang berpikiran sama di seluruh dunia yang menyumbangkan waktu dan energi berharga mereka untuk membantu membuat panda open source menjadi mungkin.

Sejak 2015, pandas adalah proyek yang disponsori NumFOCUS. Ini akan membantu memastikan keberhasilan pengembangan panda sebagai proyek sumber terbuka kelas dunia.

\subsection{Soal 6}
Fungsi-fungsi yang terdapat di library csv, yaitu:
\begin{enumerate}
	\item reader
	Fungsi ini digunakan untuk membaca isi file berformat CSV dari list.
\end{enumerate}

\subsection{Soal 7}
Jelaskan fungsi-fungsi yang terdapat di library csv
\begin{enumerate}
	\item Terdapat 2 fungsi yang bisa digunakan oleh library csv
	Pertama,fungsi membaca file csv.
\end{enumerate}


\section{Muhammad Fahmi}
\subsection{Soal 1}
Pengenalan CSV

CSV adalah singkatan dari \textit{Comma Separated Value} adalah salah satu tipe file yang digunakan secara luas untuk keperluan programming. Tidak hanya itu, CSV pun sering digunakan dalam pengolahan suatu informasi yang dihasilkan dari spreadsheet yang akan diproses lebih lanjut melalui mesin analitik. CSV juga dianggap sebagai file yang agnostik karena dapat digunakan oleh berbagai database untuk keperluan proses backup data. File CSV sangat mudah untuk dikerjakan secara terprogram. Bahasa apa pun yang mendukung input file teks dan manipulasi string (seperti Python) dapat bekerja dengan file CSV secara langsung.
\textbf{Contoh}
\lstinputlisting[frame=single, caption=Contoh CSV, firstline=1, lastline=13]{src/4/1174021/Teori/1174021.py}

Hasil yang diatas adalah : 
	\begin{figure}[H]
		\includegraphics[width=10cm]{figures/4/1174021/Teori/7.png}
		\centering
	\end{figure}

\subsection{Soal 2}
Aplikasi-aplikasi menciptakan file CSV

\begin{itemize}
	\item Text Editor
	Ada beberapa Text Editor untuk menciptakan file CSV diantara lain : 
	\begin{enumerate}
		\item Notepad
		\item Notepad++
		\item Sublime Text
		\item Visual Studio Code
		dll	
	\end{enumerate}

	\item Program Spreadsheet 
	Ada beberapa Program Spreadsheet untuk menciptakan file CSV diantara lain : 
	\begin{enumerate}
		\item Microsoft Excel
		\item WPS
		\item Google Spreadsahre
		\item LibreOfficecalc 
		dll	
	\end{enumerate}
\end{itemize}

\subsection{Soal 3}
Menulis dan membaca file CSV

\begin{enumerate} 
	\item Menulis File CSV \\
	Cara membuat file CSV sederhana yang menulis sejumlah data. Hasilnya akan berupa file CSV di satu tempat dengan file Python, penulis file CSV.
	
	Berikutnya adalah kode untuk menulis file CSV menggunakan modul CSV bawaan yang dimiliki Python:
	
	\lstinputlisting[frame=single, caption=Menulis file CSV, firstline=17, lastline=37]{src/4/1174021/Teori/1174021.py}
	
	Hasil yang diatas adalah : 
	\begin{figure}[H]
		\includegraphics[width=10cm]{figures/4/1174021/Teori/8.png}
		\centering
	\end{figure}

	\item Membaca File CSV \\
	Sekarang kita akan mencoba membaca file CSV yang telah dihasilkan oleh aplikasi atau program lain. Dalam Python, hasil membaca setiap baris dalam file CSV akan dikonversi menjadi daftar Python.
	
	Berikut adalah sebuah kode sederhana untuk membaca file CSV :
	\lstinputlisting[frame=single, caption=Membaca file CSV, firstline=42, lastline=53]{src/4/1174021/Teori/1174021.py}
\end{enumerate}


\subsection{Soal 4}
Sejarah Library CSV

CSV diciptakan untuk memudahkan data science dan analis karena CSV terdapat beberapa kemudahan dalam menggunakannya, CSV dapat dimaksimalkan jika dipadukan dengan Python karena Python adalah salah satu bahasa pemrograman yang bisa support ke banyak library termasuk CSV. Maka CSV menjadi salah satu pilihan yang digunakan oleh perusahaan-perushaan besar dalam mengolah datanya. Library CSV juga dibuat untuk mempermudah jika ingin melakukan export dan import dalam file CSV.

\subsection{Soal 5}
Sejarah Library Pandas

Panda library dibuat agar bahasa pemrograman python dapat bersaing R dan matlab, yang digunakan untuk mengolah banyak data, membutuhkan data besar, data mining data sains dan sebagainya.
panda adalah pustaka berlisensi BSD dan sumber terbuka yang menyediakan struktur data yang mudah digunakan dan berkinerja tinggi serta analisis data untuk bahasa pemrograman Python.
Dengan demikian, Pandas adalah pustaka analisis data yang memiliki struktur data yang kita butuhkan untuk membersihkan data mentah menjadi bentuk yang cocok untuk analisis (mis. Tabel). Penting untuk dicatat di sini bahwa karena melakukan tugas-tugas penting seperti menyinkronkan data untuk perbandingan dan menggabungkan set data, menangani data yang hilang, dll. Pandas awalnya dirancang untuk menangani data keuangan, karena alternatif umum adalah menggunakan spreadsheet (seperti Microsoft Excel).

\subsection{Soal 6}
Jelaskan fungsi-fungsi yang terdapat di library CSV

Ada 2 fungsi yang terdapat pada library CSV yaitu :
\begin{enumerate} 
	\item Menulis File CSV 
	Cara membuat file CSV sederhana yang menulis sejumlah data. Hasilnya akan berupa file CSV di satu tempat dengan file Python, penulis file CSV.
	
	Berikutnya adalah kode untuk menulis file CSV menggunakan modul CSV bawaan yang dimiliki Python:
	
	\lstinputlisting[frame=single, caption=Menulis file CSV, firstline=17, lastline=37]{src/4/1174021/Teori/1174021.py}
	
	Hasil yang diatas adalah : 
	\begin{figure}[H]
		\includegraphics[width=10cm]{figures/4/1174021/Teori/8.png}
		\centering
	\end{figure}
	
	\item Membaca File CSV 
	Sekarang kita akan mencoba membaca file CSV yang telah dihasilkan oleh aplikasi atau program lain. Dalam Python, hasil membaca setiap baris dalam file CSV akan dikonversi menjadi daftar Python. \\
	
	Fungsi ini bisa menggunakan list dan dictionary
	
	\begin{itemize}
		\item Dengan List :
		Berikut adalah sebuah kode sederhana untuk membaca file CSV :
		\lstinputlisting[frame=single, caption=List, firstline=1, lastline=13]{src/4/1174021/Teori/1174021.py}
		
		\item Dengan Dictionary : 
		\lstinputlisting[frame=single, caption=Dictionary, firstline=57, lastline=68]{src/4/1174021/Teori/1174021.py}
		
	\end{itemize}
	
\end{enumerate}

\subsection{Soal 7}
Jelaskan fungsi-fungsi yang terdapat di library pandas.

Tidak jauh berbeda dengan fungsi yang ada pada Library CSV, hanya saja panda lebih mudah, singkat dan lebih rapih. Berikut contohnya :
\lstinputlisting[frame=single, caption=Pandas, firstline=73, lastline=75]{src/4/1174021/Teori/1174021.py}



\section{Harun Ar-Rasyid}
\subsection{Soal 1}
Isi jawaban soal ke-1

Kalau mau dibikin paragrap \textbf{cukup enter aja}, tidak usah pakai \verb|par| dsb

%\subsection{Soal 2}
%Isi jawaban soal ke-2

%\subsection{Soal 3}
%Isi jawaban soal ke-3

\section{Sri Rahayu}
\subsection{Soal 1}
Isi jawaban soal ke-1

Kalau mau dibikin paragrap \textbf{cukup enter aja}, tidak usah pakai \verb|par| dsb

%\subsection{Soal 2}
%Isi jawaban soal ke-2

%\subsection{Soal 3}
%Isi jawaban soal ke-3

\section{Doli Jonviter}
\subsection{Soal 1}
Isi jawaban soal ke-1

Kalau mau dibikin paragrap \textbf{cukup enter aja}, tidak usah pakai \verb|par| dsb

%\subsection{Soal 2}
%Isi jawaban soal ke-2

%\subsection{Soal 3}
%Isi jawaban soal ke-3

\section{Rahmatul Ridha}
\subsection{Soal 1}
Isi jawaban soal ke-1

Kalau mau dibikin paragrap \textbf{cukup enter aja}, tidak usah pakai \verb|par| dsb

%\subsection{Soal 2}
%Isi jawaban soal ke-2

%\subsection{Soal 3}
%Isi jawaban soal ke-3

\section{Tomy Prawoto}
\subsection{Soal 1}
Isi jawaban soal ke-1

Kalau mau dibikin paragrap \textbf{cukup enter aja}, tidak usah pakai \verb|par| dsb

%\subsection{Soal 2}
%Isi jawaban soal ke-2

%\subsection{Soal 3}
%Isi jawaban soal ke-3
