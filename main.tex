%%%%%%%%%%%%%%
%% Run LaTeX on this file several times to get Table of Contents,
%% cross-references, and citations.

%% If you have font problems, you may edit the w-bookps.sty file
%% to customize the font names to match those on your system.

%% w-bksamp.tex. Current Version: Feb 16, 2012
%%%%%%%%%%%%%%%%%%%%%%%%%%%%%%%%%%%%%%%%%%%%%%%%%%%%%%%%%%%%%%%%
%
%  Sample file for
%  Wiley Book Style, Design No.: SD 001B, 7x10
%  Wiley Book Style, Design No.: SD 004B, 6x9
%
%
%  Prepared by Amy Hendrickson, TeXnology Inc.
%  http://www.texnology.com
%%%%%%%%%%%%%%%%%%%%%%%%%%%%%%%%%%%%%%%%%%%%%%%%%%%%%%%%%%%%%%%%

%%%%%%%%%%%%%
% 7x10
%\documentclass{wileySev}

% 6x9
\documentclass{wileySix}

\usepackage{graphicx}
\usepackage{listings}
\usepackage{float}

\usepackage{color}

\definecolor{codegreen}{rgb}{0,0.6,0}
\definecolor{codegray}{rgb}{0.5,0.5,0.5}
\definecolor{codepurple}{rgb}{0.58,0,0.82}
\definecolor{backcolour}{rgb}{0.95,0.95,0.92}

\lstdefinestyle{mystyle}{
    backgroundcolor=\color{backcolour},
    commentstyle=\color{codegreen},
    keywordstyle=\color{magenta},
    numberstyle=\tiny\color{codegray},
    stringstyle=\color{codepurple},
    basicstyle=\footnotesize,
    breakatwhitespace=false,
    breaklines=true,
    captionpos=b,
    keepspaces=true,
    numbers=left,
    numbersep=5pt,
    showspaces=false,
    showstringspaces=false,
    showtabs=false,
    tabsize=2,
    language=sh
}

\lstset{style=mystyle}

%%%%%%%
%% for times math: However, this package disables bold math (!)
%% \mathbf{x} will still work, but you will not have bold math
%% in section heads or chapter titles. If you don't use math
%% in those environments, mathptmx might be a good choice.

% \usepackage{mathptmx}

% For PostScript text
\usepackage{w-bookps}

%%%%%%%%%%%%%%%%%%%%%%%%%%%%%%%%%%%%%%%%%%%%%%%%%%%%%%%%%%%%%%%%
%% Other packages you might want to use:

% for chapter bibliography made with BibTeX
% \usepackage{chapterbib}

% for multiple indices
% \usepackage{multind}

% for answers to problems
% \usepackage{answers}

%%%%%%%%%%%%%%%%%%%%%%%%%%%%%%
%% Change options here if you want:
%%
%% How many levels of section head would you like numbered?
%% 0= no section numbers, 1= section, 2= subsection, 3= subsubsection
%%==>>
\setcounter{secnumdepth}{3}

%% How many levels of section head would you like to appear in the
%% Table of Contents?
%% 0= chapter titles, 1= section titles, 2= subsection titles,
%% 3= subsubsection titles.
%%==>>
\setcounter{tocdepth}{2}

%% Cropmarks? good for final page makeup
%% \docropmarks

%%%%%%%%%%%%%%%%%%%%%%%%%%%%%%
%
% DRAFT
%
% Uncomment to get double spacing between lines, current date and time
% printed at bottom of page.
% \draft
% (If you want to keep tables from becoming double spaced also uncomment
% this):
% \renewcommand{\arraystretch}{0.6}
%%%%%%%%%%%%%%%%%%%%%%%%%%%%%%

%%%%%%% Demo of section head containing sample macro:
%% To get a macro to expand correctly in a section head, with upper and
%% lower case math, put the definition and set the box
%% before \begin{document}, so that when it appears in the
%% table of contents it will also work:

\newcommand{\VT}[1]{\ensuremath{{V_{T#1}}}}

%% use a box to expand the macro before we put it into the section head:

\newbox\sectsavebox
\setbox\sectsavebox=\hbox{\boldmath\VT{xyz}}

%%%%%%%%%%%%%%%%% End Demo


\begin{document}


\booktitle{Cerdas Menguasai Python}
\subtitle{Dalam 24 Jam}

\authors{Rolly M. Awangga\\
\affil{Informatics Research Center}
%Floyd J. Fowler, Jr.\\
%\affil{University of New Mexico}
}

\offprintinfo{Cerdas Menguasai Python, First Edition}{Rolly M. Awangga}

%% Can use \\ if title, and edition are too wide, ie,
%% \offprintinfo{Survey Methodology,\\ Second Edition}{Robert M. Groves}

%%%%%%%%%%%%%%%%%%%%%%%%%%%%%%
%%
\halftitlepage

%\titlepage


\begin{copyrightpage}{2019}
\input{info/copyrightpage}
\end{copyrightpage}

\dedication{`Jika Kamu tidak dapat menahan lelahnya belajar,
Maka kamu harus sanggup menahan perihnya Kebodohan.'
~Imam Syafi'i~}

\begin{contributors}
\input{info/contributors}
\end{contributors}

\contentsinbrief
\tableofcontents
\listoffigures
\listoftables
\lstlistoflistings


\begin{foreword}
\input{info/foreword}
\end{foreword}

\begin{preface}
\input{info/preface}
\end{preface}


\begin{acknowledgments}
\input{info/acknowledgments}
\end{acknowledgments}

\begin{acronyms}
\input{info/acronyms}
\end{acronyms}

\begin{glossary}
\input{info/glossary}
\end{glossary}

\begin{symbols}
\input{info/symbols}
\end{symbols}

\begin{introduction}
\input{info/introduction}
\end{introduction}

%%%%%%%%%%%%%%%%%%Isi Buku_
%TEORI
%\chapter{Judul Bagian Pertama}
%\input{chapters/Teori/1}
%PRAKTEK
%\chapter{Judul Bagian Pertama}
%\input{chapters/Praktek/1}

%TEORI
%\chapter{Judul Bagian Pertama}
%\input{chapters/Teori/2}
%PRAKTEK
%\chapter{Judul Bagian Pertama}
%\input{chapters/Praktek/2}

%TEORI
%\chapter{Judul Bagian Pertama}
%\section{Muhammad Tomy Nur Maulidy}
{\Large \textbf{Pemahaman Teori}}
\subsection{Soal No. 1}
Apa itu fungsi, inputan fungsi dan kembalian fungsi dengan contoh kode program lainnya.

\hfill \break
Fungsi memiliki tujuan agar kita dapat memecah program besar menjadi sub-sub program yang lebih sederhana.pada masing-masing  fitur pada program dapat dibuat dalam satu fungsi. Pada saat kita membutuhkan suatu fitur maka kita tinggal memanggil fungsi yang telah kita buat. Fungsi pada python dibuat dengan menggunakan kata kunci def dan diikuti dengan nama fungsi yang telah kita buat seperti contoh dibawah ini :
 \lstinputlisting[firstline=10, lastline=10]{src/3/1174031/chapter3/1174031.py}
Inputan fungsi merupakan masukan yang kita berikan pada program dan program akan menampilkan hasil dari inputan yang telah kita masukkan atau akan menampilkan hasil pada proses selanjutnya. contoh dari inputan fungsi sebagai berikut :
 \lstinputlisting[firstline=11, lastline=11]{src/3/1174031/chapter3/1174031.py}
Pengembalian fungsi memiliki tujuan untuk mengembalikan nilai dari hasil yang telah di proses. Dalam hal ini menggunakan kata kunci return yang diikuti dengan nilai atau variabel yang akan dikembalikan.
 \lstinputlisting[firstline=10, lastline=14]{src/3/1174031/chapter3/1174031.py}

\subsection{Soal 2}
Apa itu paket dan cara pemanggilan paket atau library dengan contoh kode program lainnya.

\hfill \break
Library atau paket adalah modul-modul yang menyusun python. Modul-modul tersebut ditulis oleh berbagai orang dari seluruh dunia dan memiliki fungsi masing-masing untuk melakukan suatu hal. contoh kode programnya adalah sebagai berikut :
 \lstinputlisting[firstline=17, lastline=18]{src/3/1174031/chapter3/1174031.py}

\subsection{Soal 3}
Jelaskan Apa itu kelas, apa itu objek, apa itu atribut, apa itu method dan contoh kode program lainnya masing-masing.

\hfill \break
kelas adalah Prototype atau blueprint untuk menciptakan suatu object  yang mendefinisikan seperangkat atribut yang menjadi ciri objek kelas apa pun. Objek ialah instansiasi atau perwujudan dari sebuah kelas. Bila kelas adalah prototipenya, dan objek adalah hasil dari class jadinya. Atribut merupakan data dari anggota (variabel kelas, variabel contoh) dan metode, yang diakses dengan notasi titik. Sedangkan method fungsi yang didefinisikan di dalam suatu kelas.
 \lstinputlisting[firstline=21, lastline=40]{src/3/1174031/chapter3/1174031.py}

\subsection{Soal 4}
Jelaskan cara pemanggilan library kelas dari instansiasi dan pemakaiannya dengan contoh program lainnya.

\hfill \break
cara pemanggilan  library kelas dari instansiasi dan pemakaiannya adalah dengan cara meng-import library yang ada di dalam satu folder dengan menggunakan kode berikut :
 \lstinputlisting[firstline=43, lastline=51]{src/3/1174031/chapter3/1174031.py}

\subsection{Soal 5}
Jelaskan dengan contoh pemakaian paket dengan perintah from kalkulator import Penambahan disertai dengan contoh kode lainnya.

\hfill \break
contoh kodenya adalah sebagai berikut :
 \lstinputlisting[firstline=54, lastline=57]{src/3/1174031/chapter3/1174031.py}

\subsection{Soal 6}
Jelaskan dengan contoh kodenya, pemakaian paket fungsi apabila file library ada di dalam folder.

\hfill \break
 Pemakaian paket adalah perkumpulan fungsi-fungsi. contoh kodenya adalah sebagai berikut :
 \lstinputlisting[firstline=60, lastline=73]{src/3/1174031/chapter3/1174031.py}

\subsection{Soal 7}
Jelaskan dengan contoh kodenya, pemakaian paket kelas apabila file library ada di dalam folder.

\hfill \break
 \lstinputlisting[firstline=76, lastline=84]{src/3/1174031/chapter3/1174031.py}


\section{Dwi Yulianingsih}
\subsection{Soal 1}
Isi jawaban soal ke-1

Kalau mau dibikin paragrap \textbf{cukup enter aja}, tidak usah pakai \verb|par| dsb

%\subsection{Soal 2}
%Isi jawaban soal ke-2

%\subsection{Soal 3}
%Isi jawaban soal ke-3

\section{Harun Ar-Rasyid}
\subsection{Soal 1}
Isi jawaban soal ke-1

Kalau mau dibikin paragrap \textbf{cukup enter aja}, tidak usah pakai \verb|par| dsb

%\subsection{Soal 2}
%Isi jawaban soal ke-2

%\subsection{Soal 3}
%Isi jawaban soal ke-3

\section{Sri Rahayu}
\subsection{Soal 1}
Isi jawaban soal ke-1

Kalau mau dibikin paragrap \textbf{cukup enter aja}, tidak usah pakai \verb|par| dsb

%\subsection{Soal 2}
%Isi jawaban soal ke-2

%\subsection{Soal 3}
%Isi jawaban soal ke-3

\section{Doli Jonviter}
\subsection{Soal 1}
Isi jawaban soal ke-1

Kalau mau dibikin paragrap \textbf{cukup enter aja}, tidak usah pakai \verb|par| dsb

%\subsection{Soal 2}
%Isi jawaban soal ke-2

%\subsection{Soal 3}
%Isi jawaban soal ke-3

\section{Rahmatul Ridha}
\subsection{Soal 1}
Isi jawaban soal ke-1

Kalau mau dibikin paragrap \textbf{cukup enter aja}, tidak usah pakai \verb|par| dsb

%\subsection{Soal 2}
%Isi jawaban soal ke-2

%\subsection{Soal 3}
%Isi jawaban soal ke-3

\section{Tomy Prawoto}
\subsection{Soal 1}
Isi jawaban soal ke-1

Kalau mau dibikin paragrap \textbf{cukup enter aja}, tidak usah pakai \verb|par| dsb

%\subsection{Soal 2}
%Isi jawaban soal ke-2

%\subsection{Soal 3}
%Isi jawaban soal ke-3

%PRAKTEK
%\chapter{Judul Bagian Pertama}
%\section{Muhammad Tomy Nur Maulidy}
{\Large \textbf{Praktek}}
\subsection{Soal No. 1}

\hfill \break
 \lstinputlisting[firstline=87, lastline=121]{src/3/1174031/chapter3/1174031.py}

\subsection{Soal 2}

\hfill \break
\lstinputlisting[firstline=124, lastline=129]{src/3/1174031/chapter3/1174031.py}

\subsection{Soal 3}

\hfill \break
 \lstinputlisting[firstline=132, lastline=139]{src/3/1174031/chapter3/1174031.py}

\subsection{Soal 4}

\hfill \break
 \lstinputlisting[firstline=142, lastline=146]{src/3/1174031/chapter3/1174031.py}

\subsection{Soal 5}

\hfill \break
 \lstinputlisting[firstline=149, lastline=153]{src/3/1174031/chapter3/1174031.py}

\subsection{Soal 6}

\hfill \break
 \lstinputlisting[firstline=156, lastline=161]{src/3/1174031/chapter3/1174031.py}

\subsection{Soal 7}

\hfill \break
 \lstinputlisting[firstline=164, lastline=169]{src/3/1174031/chapter3/1174031.py}

 \subsection{Soal 8}

\hfill \break
 \lstinputlisting[firstline=172, lastline=178]{src/3/1174031/chapter3/1174031.py}

 \subsection{Soal 9}

\hfill \break
 \lstinputlisting[firstline=181, lastline=186]{src/3/1174031/chapter3/1174031.py}

 \subsection{Soal 10}

\hfill \break
 \lstinputlisting[firstline=189, lastline=204]{src/3/1174031/chapter3/1174031.py}

 \subsection{Soal 11}

\hfill \break
 \lstinputlisting[firstline=8, lastline=22]{src/3/1174031/chapter3/main.py}

 \subsection{Soal 12}

\hfill \break
 \lstinputlisting[firstline=24, lastline=39]{src/3/1174031/chapter3/main.py}

\section{Dwi Yulianingsih}
\subsection{Soal 1}
Isi jawaban soal ke-1

Kalau mau dibikin paragrap \textbf{cukup enter aja}, tidak usah pakai \verb|par| dsb

%\subsection{Soal 2}
%Isi jawaban soal ke-2

%\subsection{Soal 3}
%Isi jawaban soal ke-3

\section{Harun Ar-Rasyid}
\subsection{Soal 1}
Isi jawaban soal ke-1

Kalau mau dibikin paragrap \textbf{cukup enter aja}, tidak usah pakai \verb|par| dsb

%\subsection{Soal 2}
%Isi jawaban soal ke-2

%\subsection{Soal 3}
%Isi jawaban soal ke-3

\section{Sri Rahayu}
\subsection{Soal 1}
Isi jawaban soal ke-1

Kalau mau dibikin paragrap \textbf{cukup enter aja}, tidak usah pakai \verb|par| dsb

%\subsection{Soal 2}
%Isi jawaban soal ke-2

%\subsection{Soal 3}
%Isi jawaban soal ke-3

\section{Doli Jonviter}
\subsection{Soal 1}
Isi jawaban soal ke-1

Kalau mau dibikin paragrap \textbf{cukup enter aja}, tidak usah pakai \verb|par| dsb

%\subsection{Soal 2}
%Isi jawaban soal ke-2

%\subsection{Soal 3}
%Isi jawaban soal ke-3

\section{Rahmatul Ridha}
\subsection{Soal 1}
Isi jawaban soal ke-1

Kalau mau dibikin paragrap \textbf{cukup enter aja}, tidak usah pakai \verb|par| dsb

%\subsection{Soal 2}
%Isi jawaban soal ke-2

%\subsection{Soal 3}
%Isi jawaban soal ke-3

\section{Tomy Prawoto}
\subsection{Soal 1}
Isi jawaban soal ke-1

Kalau mau dibikin paragrap \textbf{cukup enter aja}, tidak usah pakai \verb|par| dsb

%\subsection{Soal 2}
%Isi jawaban soal ke-2

%\subsection{Soal 3}
%Isi jawaban soal ke-3


%TEORI
\chapter{Library CSV dan Pandas}
%\section{Kadek Diva Krishna Murti}
\subsection{Soal 1}
\textbf{Pengenalan CSV}

Comma Separated Values (CSV) adalah suatu format data yang di mana setiap bagian data dipisahkan dengan tanda koma (,). Format CSV biasanya berfungsi untuk menukar atau mengonversi data ke format lainnya 
%\cite{shafranovich2005common}.

\textbf{Sejarah Format CSV}

IBM Fortran (level H extended) compiler di bawah OS/360 mendukung format CSV pada tahun 1972. FORTRAN 77 mendefinisakan penulisannya dimana input atau output penulisannya menggunakan tanda koma atau spasi untuk pembatas antar data dan penulisan tersebut telah disetujui pada tahun 1978.

Osborne Executive computer yang mengembangkan SuperCalc spreadsheet pada tahun 1983 membuat konvensi kutipan CSV yang memungkinkan string mengandung koma.

Inisiatif standardisasi utama - mentransformasikan "definisi fuzzy de facto" menjadi definisi yang lebih tepat dan de jure - adalah pada tahun 2005, dengan RFC4180, mendefinisikan CSV sebagai Tipe Konten MIME. Kemudian, pada 2013, beberapa kekurangan RFC4180 ditangani oleh rekomendasi W3C.

Pada 2014 IETF menerbitkan RFC7111 yang menjelaskan aplikasi fragmen URI pada dokumen CSV. RFC7111 menentukan bagaimana rentang baris, kolom, dan sel dapat dipilih dari dokumen CSV menggunakan indeks posisi.

Pada 2015 W3C, dalam upaya meningkatkan CSV dengan semantik formal, mempublikasikan draft rekomendasi pertama untuk standar metadata CSV, yang dimulai sebagai rekomendasi pada bulan Desember tahun yang sama.

\textbf{Contoh penggunaan format CSV}

\lstinputlisting[caption = Contoh penggunaan format CSV., firstline=1, lastline=3]{src/4/1174006/Teori/teori.csv}

\subsection{Soal 2}
Aplikasi-aplikasi yang dapat menciptkan file csv, yaitu:

\begin{enumerate}
	\item Editor teks (Notepad, Sublime, Atom, dan lain-lain)
	\item Spreadsheet (Microsoft Excel dan lain-lain)
\end{enumerate}

\subsection{Soal 3}
Cara menulis dan membaca file csv di excel atau spreadsheet, sebagai berikut:

\textbf{Menulis File CSV}

\begin{enumerate}
	\item Pertama silahkan buka aplikasi Excel dengan cara klik ''Start'', cari Excel, kemudian tekan Enter.
	
	\begin{figure}[H]
		\includegraphics[width=9cm]{figures/4/1174006/Teori/t1.png}
		\centering
	\end{figure}
	
	\item Setelah aplikasi terbuka silahkan klik ''Blank Workbook''.
	
	\begin{figure}[H]
		\includegraphics[width=10cm]{figures/4/1174006/Teori/t2.png}
		\centering
	\end{figure}
	
	\item Kemudian isi sesuai dengan data yang ingin dibuat.
	
	\begin{figure}[H]
		\includegraphics[width=10cm]{figures/4/1174006/Teori/t3.png}
		\centering
	\end{figure}
	
	\item Setelah selesai dibuat, silahkan simpan file tersebut dengan cara mengklik ''File'', lalu klik ''Save''.
	
	\begin{figure}[H]
		\includegraphics[width=10cm]{figures/4/1174006/Teori/t4.png}
		\centering
	\end{figure}
	
	\item Kemudian isi kolom ''File name'' dengan nama file anda dan kolom ''Save as type'' pilih yang berekstensi .csv.
	
	\begin{figure}[H]
		\includegraphics[width=9cm]{figures/4/1174006/Teori/t5.png}
		\centering
	\end{figure}
	
	\item Lalu tinggal klik ''Yes''.
	
	\begin{figure}[H]
		\includegraphics[width=7cm]{figures/4/1174006/Teori/t6.png}
		\centering
	\end{figure}
	
	\item Kemudian file yang Anda telah terbuat tadi tersimpan dengan ekstensi .csv. Untuk melihat isi filenya tinggal klik dua kali pada file tersebut.
	
	\begin{figure}[H]
		\includegraphics[width=10cm]{figures/4/1174006/Teori/t8.png}
		\centering
	\end{figure}
	
	\item Berikut ini adalah isi dari file yang tadi Anda buat.
	
	\begin{figure}[H]
		\includegraphics[width=8cm]{figures/4/1174006/Teori/t7.png}
		\centering
	\end{figure}
\end{enumerate}

\textbf{Melihat File CSV di Excel atau Spreadsheet}

\begin{enumerate}
	\item Pertama klik dua kali pada file yang yang berekstensi CSV.
	
	\begin{figure}[H]
		\includegraphics[width=10cm]{figures/4/1174006/Teori/t8.png}
		\centering
	\end{figure}
	
	\item Kemudian file akan terbuka secara otomatis di aplikasi Excel atau spreadsheet.
	
	\begin{figure}[H]
		\includegraphics[width=10cm]{figures/4/1174006/Teori/t9.png}
		\centering
	\end{figure}
\end{enumerate}

\subsection{Soal 4}
Sejarah library csv

Library csv mengimplementasikan kelas untuk membaca dan menulis data tabular dalam format CSV. Hal ini memungkinkan programmer untuk mengatakan, "tulis data ini dalam format yang disukai oleh Excel," atau "baca data dari file ini yang dihasilkan oleh Excel," tanpa mengetahui detail yang tepat dari format CSV yang digunakan oleh Excel. Pemrogram juga dapat menggambarkan format CSV yang dipahami oleh aplikasi lain atau menentukan format CSV tujuan khusus mereka sendiri.

\subsection{Soal 5}
Sejarah library pandas

Pada 2008, pengembangan pandas dimulai di AQR Capital Management. Pada akhir 2009 telah menjadi open source, dan secara aktif didukung hari ini oleh komunitas individu yang berpikiran sama di seluruh dunia yang menyumbangkan waktu dan energi berharga mereka untuk membantu membuat panda open source menjadi mungkin.

Sejak 2015, pandas adalah proyek yang disponsori NumFOCUS. Ini akan membantu memastikan keberhasilan pengembangan panda sebagai proyek sumber terbuka kelas dunia.

\subsection{Soal 6}
Fungsi-fungsi yang terdapat di library csv, yaitu:
\begin{enumerate}
	\item reader
	
	Fungsi ini digunakan untuk membaca isi file berformat CSV dari list.
	
	\lstinputlisting[caption = Membaca file berformat CSV list., firstline=7, lastline=13]{src/4/1174006/Teori/1174006.py}
	
	\item DictReader
	
	Fungsi ini digunakan untuk membaca isi file berformat CSV dari dictionary.
	
	\lstinputlisting[caption =  Membaca file berformat CSV dictionary., firstline=15, lastline=21]{src/4/1174006/Teori/1174006.py}
	
	\item write
	
	Fungsi ini digunakan untuk menulis file berformat CSV dari list.
	
	\lstinputlisting[caption =  Menulis file berformat CSV list., firstline=23, lastline=30]{src/4/1174006/Teori/1174006.py}
	
	\item DictWrite
	
	Fungsi ini digunakan untuk menulis file berformat CSV dari dictionary.
	
	\lstinputlisting[caption =  Menulis file berformat CSV dictionary., firstline=32, lastline=41]{src/4/1174006/Teori/1174006.py}
	
\end{enumerate}

\subsection{Soal 7}
Fungsi-fungsi yang terdapat di library pandas, yaitu:
\begin{enumerate}
	\item read\_csv
	
	Fungsi ini digunakan untuk membaca isi file berformat CSV
	
	\lstinputlisting[caption =  Membaca file berformat CSV pandas., firstline=43, lastline=47]{src/4/1174006/Teori/1174006.py}
	
	\item to\_csv
	
	Fungsi ini digunakan untuk menulis file berformat CSV
	
	\lstinputlisting[caption =  Menulis file berformat CSV pandas., firstline=49, lastline=53]{src/4/1174006/Teori/1174006.py}
	
\end{enumerate}

\subsection{Kode Program Teori}
\begin{figure}[H]
	\includegraphics[width=10cm]{figures/4/1174006/Teori/kode_teori1.png}
	\centering
\end{figure}

\begin{figure}[H]
	\includegraphics[width=10cm]{figures/4/1174006/Teori/kode_teori2.png}
	\centering
\end{figure}

\subsection{Cek Plagiat Teori}

\begin{figure}[H]
	\includegraphics[width=10cm]{figures/4/1174006/Teori/plagiat_teori.png}
	\centering
\end{figure}


\section{Damara Benedikta}
\subsection{Soal 1}
 CSV (Comma Separated Value) merupakan suatu  format basis data sederhana yang dimana setiap record yang ada dipisahkan dengan tanda koma (,) atau titik koma (;). Format data file csv dapat diolah dengan berbagai text editor dengan mudah. Anda tidak perlu (dan Anda tidak akan) membuat pengurai CSV Anda sendiri dari awal. Ada beberapa perpustakaan yang dapat diterima yang dapat Anda gunakan. Pustaka csv Python akan berfungsi untuk sebagian besar kasus. Jika pekerjaan Anda memerlukan banyak data atau analisis numerik, panda library juga memiliki kemampuan penguraian CSV, yang seharusnya menangani sisanya. Dalam bahasa pemrograman Python telah disediakan modul csv yang khusus untuk mengolah data berformat csv.  Untuk memanipulasi data csv dengan python tentunya yang pertama dilakukan adalah mengimport modul csv dengan perintah import csv. File CSV biasanya dibuat oleh program yang menangani sejumlah besar data. Mereka adalah cara yang nyaman untuk mengekspor data dari spreadsheet dan basis data serta mengimpor atau menggunakannya dalam program lain. Misalnya, Anda dapat mengekspor hasil program penambangan data ke file CSV dan kemudian mengimpornya ke dalam spreadsheet untuk menganalisis data, menghasilkan grafik untuk presentasi, atau menyiapkan laporan untuk publikasi. Contoh nya adalah sebagai berikut :

 \lstinputlisting[firstline=8, lastline=20]{src/4/1174012/Teori/damdam.py}

\subsection{Soal 2} 
 Ada beberapa aplikasi yang dapat menciptakan file dengan format csv diantaranya google sheet, number di MacOS dan microsoft excel.

\subsection{Soal 3}
 Cara membuat file csv di excel cukup mudah yaitu :
\begin{itemize}
	\item Buat foldernya
	\item Pilih save as
	\item pilih file dengan format csv
\end{itemize}
Cara membaca file di csv :
\begin{itemize}
	\item Klik data - get external data - form text
	\item Akan muncul Text Import Wizard, arahkan pada file csv yang ingin anda buka lalu Open.
	\item Setelah File terbuka, akan muncul Text Import Wizard.
	\item Pilih Delimited, Kemudian Next (Di sini, bisa juga menentukan baris awal yang akan di import)
	\item Centrang pada Tab dan Comma (Atau sesuai pengaturan File Anda) lalu Next.
	\item Atur Format data pada tiap kolom yang tampil dan klik Finish
\end{itemize}

\subsection{Soal 4}
 CSV digunakan untuk memudahkan data science dan analis karena dinilai terdapat banyak kemudahan yang diperoleh. CSV dapat dimaksimalkan jika dipaduka dengan python karena python adalah bahasa pemrograman yang support ke banyak library termasuk csv. Maka karena itulah perpaduan python dan csv seringkali digunakan oleh perusahaan-perushaan besar dalam mengolah datanya.

\subsection{Soal 5}
Pandas merupakan sebuah tool yang dapat digunakan sebagai alat analisis data dan struktur untuk bahasa pemrograman Python. Pandas dapat mengolah data dengan mudah, salah satu fitur yang ada dalam pandas adalah Dataframe. Fitur dataframe dapat membaca sebuah file dan menjadikannya tabble, juga dapat mengolah suatu data dengan menggunakan operasi seperti join, group by dan teknik lainnya yang terdapat pada SQL. Dalam hal ini pandas tidak jauh beda dengan csv yaitu memiliki keunggulan dalam pengolahan data-data besar dan dapat disupport dengan baik dengan python walaupun mengimport data dalam jumlah banyak.

\subsection{Soal 6}
 Library csv memiliki keunggula-keunggulan dibandingkan format data lainnya merupakah soal kompatibilitas. File csv dapat digunakan, diolah, diekspor/impor, dan dimodifikasi menggunakan berbagai macam perangkat lunak dan bahasa pemrograman. Pada library csv mempunyai fungsi import dan eksport data yang baik dan bisa digunakan dalam jumlah besar.

\subsection{Soal 7}
pandas menyediakan beberapa fungsi operasi untuk mengolah data. Contoh jika menggunakan series bisa mencari nilai max, min, dan mean secara langsung, bahkan juga bisa melakukan operasi perpangkatan pada nilai Series secara langsung.
Pandas dapat mengolah suatu data dan mengolahnya seperti join, distinct, group by, agregasi, dan teknik seperti pada SQL. Hanya saja dilakukan pada tabel yang dimuat dari file ke RAM.


\subsection{bukti bebas plagiarisme}
\begin{figure}[H]
\centering
\includegraphics[width=10cm]{figures/4/1174012/Teori/ss1.png}
\caption{SS Bebas Plagiarisme}
\label{damara}
\end{figure}
%%%%%%%%%%%%%%%%%%%%%%%%%%%%%%%%%%%%%%%%%%%%%%
\section{Felix Setiawan Lase}
\subsection{Soal 1}
\textbf{Pengenalan CSV}

File CSV (Nilai Terbatas Koma) adalah jenis file khusus yang dapat Anda buat atau edit di Excel. File CSV menyimpan informasi yang disimpan dengan koma alih-alih menyimpan informasi dalam kolom.

\textbf{Sejarah Format CSV}

Kompiler Fortran IBM (tingkat lanjut H) di bawah OS / 360 mendukung format CSV pada tahun 1972. FORTRAN 77 mendefinisikan penulisannya di mana penulisan input atau output menggunakan koma atau spasi untuk batas antara data dan penulisan disetujui pada tahun 1978.

Pada 2014 IETF menerbitkan RFC7111 yang menjelaskan penerapan fragmen URI dalam dokumen CSV. RFC7111 menentukan bagaimana berbagai baris, kolom, dan sel dapat dipilih dari dokumen CSV menggunakan indeks posisi.

Pada 2015, W3C, dalam upaya meningkatkan CSV dengan semantik formal, menerbitkan rancangan rekomendasi pertama untuk standar metadata CSV, yang dimulai sebagai rekomendasi pada bulan Desember tahun yang sama.

\textbf{Contoh penggunaan format CSV}

\lstinputlisting[caption = Contoh penggunaan format CSV., firstline=1, lastline=3]{src/4/1174026/Teori/teori.csv}

\subsection{Soal 2}
Aplikasi-aplikasi yang dapat menciptkan file csv, yaitu:

\begin{enumerate}
	\item Editor teks (Notepad, Sublime, Atom, dan lain-lain)
	\item Spreadsheet (Microsoft Excel dan lain-lain)
\end{enumerate}

\subsection{Soal 3}
Cara menulis dan membaca file csv di excel atau spreadsheet, sebagai berikut:

\textbf{Menulis File CSV}

\begin{enumerate}
	\item Pertama silahkan buka aplikasi Excel dengan cara klik ''Start'', cari Excel, kemudian tekan Enter.
	
	\begin{figure}[H]
		\includegraphics[width=9cm]{figures/4/1174026/Teori/t1.png}
		\centering
	\end{figure}
	
	\item Setelah aplikasi terbuka silahkan klik ''Blank Workbook''.
	
	\begin{figure}[H]
		\includegraphics[width=10cm]{figures/4/1174026/Teori/t2.png}
		\centering
	\end{figure}
	
	\item Kemudian isi sesuai dengan data yang ingin dibuat.
	
	\begin{figure}[H]
		\includegraphics[width=10cm]{figures/4/1174026/Teori/t3.png}
		\centering
	\end{figure}
	
	\item Setelah selesai dibuat, silahkan simpan file tersebut dengan cara mengklik ''File'', lalu klik ''Save''.
	
	\begin{figure}[H]
		\includegraphics[width=10cm]{figures/4/1174026/Teori/t4.png}
		\centering
	\end{figure}
	
	\item Kemudian isi kolom ''File name'' dengan nama file anda dan kolom ''Save as type'' pilih yang berekstensi .csv.
	
	\begin{figure}[H]
		\includegraphics[width=9cm]{figures/4/1174026/Teori/t5.png}
		\centering
	\end{figure}
	
	\item Lalu tinggal klik ''Yes''.
	
	\begin{figure}[H]
		\includegraphics[width=7cm]{figures/4/1174026/Teori/t6.png}
		\centering
	\end{figure}
	
	\item Kemudian file yang Anda telah terbuat tadi tersimpan dengan ekstensi .csv. Untuk melihat isi filenya tinggal klik dua kali pada file tersebut.
	
	\begin{figure}[H]
		\includegraphics[width=10cm]{figures/4/1174026/Teori/t8.png}
		\centering
	\end{figure}
	
	\item Berikut ini adalah isi dari file yang tadi Anda buat.
	
	\begin{figure}[H]
		\includegraphics[width=8cm]{figures/4/1174026/Teori/t7.png}
		\centering
	\end{figure}
\end{enumerate}

\textbf{Melihat File CSV di Excel atau Spreadsheet}

\begin{enumerate}
	\item Pertama klik dua kali pada file yang yang berekstensi CSV.
	
	\begin{figure}[H]
		\includegraphics[width=10cm]{figures/4/1174026/Teori/t8.png}
		\centering
	\end{figure}
	
	\item Kemudian file akan terbuka secara otomatis di aplikasi Excel atau spreadsheet.
	
	\begin{figure}[H]
		\includegraphics[width=10cm]{figures/4/1174026/Teori/t9.png}
		\centering
	\end{figure}
\end{enumerate}

\subsection{Soal 4}
Sejarah library csv

Perpustakaan CSV mengimplementasikan kelas untuk membaca dan menulis data tabular dalam format CSV. Ini memungkinkan programmer untuk mengatakan, "tulis data ini dalam format yang disukai Excel," atau "baca data dari file ini yang dihasilkan oleh Excel," tanpa mengetahui detail pasti dari format CSV yang digunakan oleh Excel. Pemrogram juga dapat menggambarkan format CSV yang dimengerti oleh aplikasi lain atau menentukan format CSV spesifik mereka sendiri.
	

\subsection{Soal 5}
Sejarah library pandas

Tahun 2008, pengembangan profesional dimulai di AQR Capital Management. Pada akhir 2009 ini telah menjadi open source, dan secara aktif didukung hari ini oleh komunitas individu yang berpikiran sama di seluruh dunia yang menyumbangkan waktu dan energi berharga mereka untuk membantu membuat panda open source menjadi mungkin.

	Sejak tahun 2015, Pandas adalah proyek yang disponsori oleh NumFOCUS. Ini akan membantu memastikan keberhasilan pengembangan Panda sebagai proyek open source kelas dunia.
	

\subsection{Soal 6}
Fungsi-fungsi yang terdapat di library csv, yaitu:
\begin{enumerate}
	\item reader
	
	Fungsi ini digunakan untuk membaca isi file berformat CSV dari list.
	
	\lstinputlisting[caption = Membaca file berformat CSV list., firstline=7, lastline=13]{src/4/1174026/Teori/1174026.py}
	
	\item DictReader
	
	Fungsi ini digunakan untuk membaca isi file berformat CSV dari dictionary.
	
	\lstinputlisting[caption =  Membaca file berformat CSV dictionary., firstline=15, lastline=21]{src/4/1174026/Teori/1174026.py}
	
	\item write
	
	Fungsi ini digunakan untuk menulis file berformat CSV dari list.
	
	\lstinputlisting[caption =  Menulis file berformat CSV list., firstline=23, lastline=30]{src/4/1174026/Teori/1174026.py}
	
	\item DictWrite
	
	Fungsi ini digunakan untuk menulis file berformat CSV dari dictionary.
	
	\lstinputlisting[caption =  Menulis file berformat CSV dictionary., firstline=32, lastline=41]{src/4/1174026/Teori/1174026.py}
	
\end{enumerate}

\subsection{Soal 7}
Fungsi-fungsi yang terdapat di library pandas, yaitu:
\begin{enumerate}
	\item read\_csv
	
	Fungsi ini digunakan untuk membaca isi file berformat CSV
	
	\lstinputlisting[caption =  Membaca file berformat CSV pandas., firstline=43, lastline=47]{src/4/1174026/Teori/1174026.py}
	
	\item to\_csv
	
	Fungsi ini digunakan untuk menulis file berformat CSV
	
	\lstinputlisting[caption =  Menulis file berformat CSV pandas., firstline=49, lastline=53]{src/4/1174026/Teori/1174026.py}
	
\end{enumerate}

\subsection{Kode Program Teori}
\begin{figure}[H]
	\includegraphics[width=10cm]{figures/4/1174026/Teori/kode_teori1.png}
	\centering
\end{figure}

\begin{figure}[H]
	\includegraphics[width=10cm]{figures/4/1174026/Teori/kode_teori2.png}
	\centering
\end{figure}

\subsection{Cek Plagiat Teori}

\begin{figure}[H]
	\includegraphics[width=10cm]{figures/4/1174026/Teori/plagiat_teori.png}
	\centering
\end{figure}
%%%%%%%%%%%%%%%%%%%%%%%%%%%%%%%%%%%%%%%%%%%%%

\section{Dwi Septiani Tsaniyah}
\subsection{Soal 1}
\textbf{Pengenalan CSV}

\textbf{Sejarah Format CSV}

File CSV (Nilai Berbatas Koma) adalah tipe file khusus yang dapat Anda buat atau edit di Excel. File CSV menyimpan informasi yang dipisahkan oleh koma, bukan menyimpan informasi dalam kolom. Saat teks dan angka disimpan dalam file CSV, mudah untuk memindahkannya dari satu program ke program lain. Misalnya, Anda dapat mengekspor kontak dari Google ke dalam file CSV, kemudian mengimpornya ke Outlook.
Creating Shared Value (CSV) adalah sebuah konsep dalam strategi bisnis yang menekankan pentingnya memasukkan masalah dan kebutuhan sosial dalam perancangan strategi perusahaan. CSV merupakan pengembangan dari konsep tanggung jawab sosial perusahaan (Corporate social responsibility, CSR). Konsep ini pertama kali diperkenalkan oleh Michael Porter dan Mark Kramer pada tahun 2006. Konsep CSV didasari pada ide adanya hubungan interdependen antara bisnis dan kesejahteraan sosial. Porter mengkritik bahwa selama ini bisnis dan kesejahteraan sosial selalu ditempatkan berseberangan. Pebisnis pun rela mengorbankan kesejahteraan sosial demi keuntungan semata, misalnya dengan melakukan proses produksi yang tidak memperhatikan lingkungan atau menciptakan polusi. CSV menekankan adanya peluang untuk membangun keunggulan kompetitif dengan cara memasukan masalah sosial sebagai bahan pertimbangan utama dalam merancang strategi perusahaan.
contoh : Ketika Toyota memperkenalkan Prius, sebuah kendaraan hybrid listrik/bensin, Toyota berhasil mendapatkan keunggulan kompetitif dengan memasarkan sebuah kendaraan yang tidak hanya memberikan keuntungan ekonomis, namun juga berdampak positif bagi lingkugan. Urbi, sebuah perusahaan konstruksi asal Meksiko, mengembangkan pasar perumahan dengan memberikan kredit murah untuk pekerja dengan gaji kecil, Whole Foods Market telah menjadi pemimpin kategori di segmen supermarket dengan menawarkan makanan organik dan alami kepada konsumen yang sadar lingkungan. Perusahaan juga dapat meningkatkan keunggulan kompetitif dengan melakukan investasi di komunitas di mana mereka beroperasi. Nestlé, misalnya, berhubungan sangat dekat dengan Distrik Susu Moga di India, melakukan investasi pada infrastruktur lokal, dan mentransfer teknologi kelas dunia untuk membangun rantai suplai yang kompetitif sekaligus meningkatkan kesejahteraan sosial melalui peningkatan kesehatan masyarakat, pendidikan yang lebih baik, dan pertumbuhan ekonomi.

\subsection{Soal 2}
Aplikasi-aplikasi yang dapat menciptkan file csv, yaitu:

\begin{itemize}
\item Texteditor , Seperti notepad,visual studio code,atom,sublime dan lain sebagainya
\item Program Spreadsheet , Seperti excell,google spreadshare,LibreOfficecalc
\end{itemize}

\subsection{Soal 3}
\begin{enumerate}
\item Cara menulis dan membaca file csv di excel atau spreadsheet, sebagai berikut:
 Ada dua cara untuk mengimpor data dari file teks dengan Excel dapat membukanya di Excel, atau mengimpornya sebagai rentang data eksternal. Untuk mengekspor data dari Excel menjadi file teks, gunakan perintah Simpan Sebagai dan ubah tipe file dari menu menurun.
\item Ada dua format file teks yang biasanya digunakan:
File teks berbatas (.txt), dengan karakter TAB (kode karakter ASCII 009) yang biasanya memisahkan setiap bidang teks.
File teks nilai yang dipisahkan koma (.csv), dengan karakter koma (,) yang biasanya memisahkan setiap bidang teks.
\end{enumerate}

\subsection{Soal 4}
Sejarah library csv

Library csv mengimplementasikan kelas untuk membaca dan menulis data tabular dalam format CSV. Hal ini memungkinkan programmer untuk mengatakan, "tulis data ini dalam format yang disukai oleh Excel," atau "baca data dari file ini yang dihasilkan oleh Excel," tanpa mengetahui detail yang tepat dari format CSV yang digunakan oleh Excel. Pemrogram juga dapat menggambarkan format CSV yang dipahami oleh aplikasi lain atau menentukan format CSV tujuan khusus mereka sendiri.

\subsection{Soal 5}
Sejarah library pandas

Pada 2008, pengembangan pandas dimulai di AQR Capital Management. Pada akhir 2009 telah menjadi open source, dan secara aktif didukung hari ini oleh komunitas individu yang berpikiran sama di seluruh dunia yang menyumbangkan waktu dan energi berharga mereka untuk membantu membuat panda open source menjadi mungkin.

Sejak 2015, pandas adalah proyek yang disponsori NumFOCUS. Ini akan membantu memastikan keberhasilan pengembangan panda sebagai proyek sumber terbuka kelas dunia.

\subsection{Soal 6}
Fungsi-fungsi yang terdapat di library csv, yaitu:
\begin{enumerate}
	\item reader
	Fungsi ini digunakan untuk membaca isi file berformat CSV dari list.
\end{enumerate}

\subsection{Soal 7}
Jelaskan fungsi-fungsi yang terdapat di library csv
\begin{enumerate}
	\item Terdapat 2 fungsi yang bisa digunakan oleh library csv
	Pertama,fungsi membaca file csv.
\end{enumerate}


\section{Muhammad Fahmi}
\subsection{Soal 1}
Pengenalan CSV

CSV adalah singkatan dari \textit{Comma Separated Value} adalah salah satu tipe file yang digunakan secara luas untuk keperluan programming. Tidak hanya itu, CSV pun sering digunakan dalam pengolahan suatu informasi yang dihasilkan dari spreadsheet yang akan diproses lebih lanjut melalui mesin analitik. CSV juga dianggap sebagai file yang agnostik karena dapat digunakan oleh berbagai database untuk keperluan proses backup data. File CSV sangat mudah untuk dikerjakan secara terprogram. Bahasa apa pun yang mendukung input file teks dan manipulasi string (seperti Python) dapat bekerja dengan file CSV secara langsung.
\textbf{Contoh}
\lstinputlisting[frame=single, caption=Contoh CSV, firstline=1, lastline=13]{src/4/1174021/Teori/1174021.py}

Hasil yang diatas adalah : 
	\begin{figure}[H]
		\includegraphics[width=10cm]{figures/4/1174021/Teori/7.png}
		\centering
	\end{figure}

\subsection{Soal 2}
Aplikasi-aplikasi menciptakan file CSV

\begin{itemize}
	\item Text Editor
	Ada beberapa Text Editor untuk menciptakan file CSV diantara lain : 
	\begin{enumerate}
		\item Notepad
		\item Notepad++
		\item Sublime Text
		\item Visual Studio Code
		dll	
	\end{enumerate}

	\item Program Spreadsheet 
	Ada beberapa Program Spreadsheet untuk menciptakan file CSV diantara lain : 
	\begin{enumerate}
		\item Microsoft Excel
		\item WPS
		\item Google Spreadsahre
		\item LibreOfficecalc 
		dll	
	\end{enumerate}
\end{itemize}

\subsection{Soal 3}
Menulis dan membaca file CSV

\begin{enumerate} 
	\item Menulis File CSV \\
	Cara membuat file CSV sederhana yang menulis sejumlah data. Hasilnya akan berupa file CSV di satu tempat dengan file Python, penulis file CSV.
	
	Berikutnya adalah kode untuk menulis file CSV menggunakan modul CSV bawaan yang dimiliki Python:
	
	\lstinputlisting[frame=single, caption=Menulis file CSV, firstline=17, lastline=37]{src/4/1174021/Teori/1174021.py}
	
	Hasil yang diatas adalah : 
	\begin{figure}[H]
		\includegraphics[width=10cm]{figures/4/1174021/Teori/8.png}
		\centering
	\end{figure}

	\item Membaca File CSV \\
	Sekarang kita akan mencoba membaca file CSV yang telah dihasilkan oleh aplikasi atau program lain. Dalam Python, hasil membaca setiap baris dalam file CSV akan dikonversi menjadi daftar Python.
	
	Berikut adalah sebuah kode sederhana untuk membaca file CSV :
	\lstinputlisting[frame=single, caption=Membaca file CSV, firstline=42, lastline=53]{src/4/1174021/Teori/1174021.py}
\end{enumerate}


\subsection{Soal 4}
Sejarah Library CSV

CSV diciptakan untuk memudahkan data science dan analis karena CSV terdapat beberapa kemudahan dalam menggunakannya, CSV dapat dimaksimalkan jika dipadukan dengan Python karena Python adalah salah satu bahasa pemrograman yang bisa support ke banyak library termasuk CSV. Maka CSV menjadi salah satu pilihan yang digunakan oleh perusahaan-perushaan besar dalam mengolah datanya. Library CSV juga dibuat untuk mempermudah jika ingin melakukan export dan import dalam file CSV.

\subsection{Soal 5}
Sejarah Library Pandas

Panda library dibuat agar bahasa pemrograman python dapat bersaing R dan matlab, yang digunakan untuk mengolah banyak data, membutuhkan data besar, data mining data sains dan sebagainya.
panda adalah pustaka berlisensi BSD dan sumber terbuka yang menyediakan struktur data yang mudah digunakan dan berkinerja tinggi serta analisis data untuk bahasa pemrograman Python.
Dengan demikian, Pandas adalah pustaka analisis data yang memiliki struktur data yang kita butuhkan untuk membersihkan data mentah menjadi bentuk yang cocok untuk analisis (mis. Tabel). Penting untuk dicatat di sini bahwa karena melakukan tugas-tugas penting seperti menyinkronkan data untuk perbandingan dan menggabungkan set data, menangani data yang hilang, dll. Pandas awalnya dirancang untuk menangani data keuangan, karena alternatif umum adalah menggunakan spreadsheet (seperti Microsoft Excel).

\subsection{Soal 6}
Jelaskan fungsi-fungsi yang terdapat di library CSV

Ada 2 fungsi yang terdapat pada library CSV yaitu :
\begin{enumerate} 
	\item Menulis File CSV 
	Cara membuat file CSV sederhana yang menulis sejumlah data. Hasilnya akan berupa file CSV di satu tempat dengan file Python, penulis file CSV.
	
	Berikutnya adalah kode untuk menulis file CSV menggunakan modul CSV bawaan yang dimiliki Python:
	
	\lstinputlisting[frame=single, caption=Menulis file CSV, firstline=17, lastline=37]{src/4/1174021/Teori/1174021.py}
	
	Hasil yang diatas adalah : 
	\begin{figure}[H]
		\includegraphics[width=10cm]{figures/4/1174021/Teori/8.png}
		\centering
	\end{figure}
	
	\item Membaca File CSV 
	Sekarang kita akan mencoba membaca file CSV yang telah dihasilkan oleh aplikasi atau program lain. Dalam Python, hasil membaca setiap baris dalam file CSV akan dikonversi menjadi daftar Python. \\
	
	Fungsi ini bisa menggunakan list dan dictionary
	
	\begin{itemize}
		\item Dengan List :
		Berikut adalah sebuah kode sederhana untuk membaca file CSV :
		\lstinputlisting[frame=single, caption=List, firstline=1, lastline=13]{src/4/1174021/Teori/1174021.py}
		
		\item Dengan Dictionary : 
		\lstinputlisting[frame=single, caption=Dictionary, firstline=57, lastline=68]{src/4/1174021/Teori/1174021.py}
		
	\end{itemize}
	
\end{enumerate}

\subsection{Soal 7}
Jelaskan fungsi-fungsi yang terdapat di library pandas.

Tidak jauh berbeda dengan fungsi yang ada pada Library CSV, hanya saja panda lebih mudah, singkat dan lebih rapih. Berikut contohnya :
\lstinputlisting[frame=single, caption=Pandas, firstline=73, lastline=75]{src/4/1174021/Teori/1174021.py}



\section{Harun Ar-Rasyid}
\subsection{Soal 1}
Isi jawaban soal ke-1

Kalau mau dibikin paragrap \textbf{cukup enter aja}, tidak usah pakai \verb|par| dsb

%\subsection{Soal 2}
%Isi jawaban soal ke-2

%\subsection{Soal 3}
%Isi jawaban soal ke-3

\section{Sri Rahayu}
\subsection{Soal 1}
Isi jawaban soal ke-1

Kalau mau dibikin paragrap \textbf{cukup enter aja}, tidak usah pakai \verb|par| dsb

%\subsection{Soal 2}
%Isi jawaban soal ke-2

%\subsection{Soal 3}
%Isi jawaban soal ke-3

\section{Doli Jonviter}
\subsection{Soal 1}
Isi jawaban soal ke-1

Kalau mau dibikin paragrap \textbf{cukup enter aja}, tidak usah pakai \verb|par| dsb

%\subsection{Soal 2}
%Isi jawaban soal ke-2

%\subsection{Soal 3}
%Isi jawaban soal ke-3

\section{Rahmatul Ridha}
\subsection{Soal 1}
Isi jawaban soal ke-1

Kalau mau dibikin paragrap \textbf{cukup enter aja}, tidak usah pakai \verb|par| dsb

%\subsection{Soal 2}
%Isi jawaban soal ke-2

%\subsection{Soal 3}
%Isi jawaban soal ke-3

\section{Tomy Prawoto}
\subsection{Soal 1}
Isi jawaban soal ke-1

Kalau mau dibikin paragrap \textbf{cukup enter aja}, tidak usah pakai \verb|par| dsb

%\subsection{Soal 2}
%Isi jawaban soal ke-2

%\subsection{Soal 3}
%Isi jawaban soal ke-3

%PRAKTEK
\chapter{Praktek Library CSV dan Pandas}
%\section{Kadek Diva Krishna Murti}
\subsection{Soal 1}
Buatlah  fungsi  (file  terpisah/library  dengan  nama  NPMcsv.py)  untuk  membuka file csv dengan lib csv mode list.

\lstinputlisting[caption = Fungsi untuk membuka file CSV dengan lib CSV mode list., firstline=10, lastline=15]{src/4/1174006/Praktek/1174006csv.py}

\subsection{Soal 2}
Buatlah  fungsi  (file  terpisah/library  dengan  nama  NPMcsv.py)  untuk  membuka file csv dengan lib csv mode dictionary.

\lstinputlisting[caption =  Fungsi untuk membuka file CSV dengan lib CSV mode dictionary., firstline=17, lastline=22]{src/4/1174006/Praktek/1174006csv.py}

\subsection{Soal 3}
Buatlah fungsi (file terpisah/library dengan nama NPMpandas.py) untuk membuka file csv dengan lib pandas mode list.

\lstinputlisting[caption =  Fungsi untuk membuka file CSV dengan lib Pandas mode list., firstline=10, lastline=13]{src/4/1174006/Praktek/1174006pandas.py}

\subsection{Soal 4}
Buatlah fungsi (file terpisah/library dengan nama NPMpandas.py) untuk membuka file csv dengan lib pandas mode dictionary.

\lstinputlisting[caption =  Fungsi untuk membuka file CSV dengan lib Pandas mode dictionary., firstline=10, lastline=13]{src/4/1174006/Praktek/1174006pandas.py}

\subsection{Soal 5}
Buat fungsi baru di NPMpandas.py untuk mengubah format tanggal menjadi standar dataframe.

\lstinputlisting[caption =  Fungsi untuk mengubah format tanggal menjadi standar dataframe., firstline=15, lastline=19]{src/4/1174006/Praktek/1174006pandas.py}

\subsection{Soal 6}
Buat fungsi baru di NPMpandas.py untuk mengubah index kolom.

\lstinputlisting[caption =  Fungsi untuk mengubah index kolom., firstline=21, lastline=24]{src/4/1174006/Praktek/1174006pandas.py}

\subsection{Soal 7}
Buat fungsi baru di NPMpandas.py untuk mengubah atribut atau nama kolom.

\lstinputlisting[caption =  Fungsi untuk mengubah atribut atau nama kolom., firstline=26, lastline=30]{src/4/1174006/Praktek/1174006pandas.py}

\subsection{Soal 8}
Buat program main.py yang menggunakan library NPMcsv.py yang membuat dan membaca file csv.

\lstinputlisting[caption =  Membuat dan mebaca file CSV menggunakan library 1174006pandas., firstline=8, lastline=13]{src/4/1174006/Praktek/main.py}

\subsection{Soal 9}
Buat program main2.py yang menggunakan library NPMpandas.py yang membuat dan membaca file csv.

\lstinputlisting[caption = Membuat dan mmebaca file CSV menggunakan library 1174006pandas., firstline=8, lastline=13]{src/4/1174006/Praktek/main2.py}

\subsection{Kode Program Praktek}
\begin{figure}[H]
	\includegraphics[width=9cm]{figures/4/1174006/Praktek/k1.png}
	\centering
\end{figure}
\begin{figure}[H]
	\includegraphics[width=10cm]{figures/4/1174006/Praktek/k2.png}
	\centering
\end{figure}
\begin{figure}[H]
	\includegraphics[width=10cm]{figures/4/1174006/Praktek/k3.png}
	\centering
\end{figure}
\begin{figure}[H]
	\includegraphics[width=9cm]{figures/4/1174006/Praktek/k4.png}
	\centering
\end{figure}
\begin{figure}[H]
	\includegraphics[width=10cm]{figures/4/1174006/Praktek/k5.png}
	\centering
\end{figure}

\subsection{Cek Plagiat Praktek}
\begin{figure}[H]
	\includegraphics[width=10cm]{figures/4/1174006/Praktek/plagiatketrampilan.png}
	\centering
\end{figure}

\subsection{Soal 1}
Tuliskan  peringatan  error  yang  didapat  dari  mengerjakan  praktek  keempat  ini, dan  jelaskan  cara  penanganan  error  tersebut.   dan  Buatlah  satu  fungsi  yang menggunakan gunakan try except untuk menanggulangi error tersebut.

Peringatan error di praktek keempat ini, yaitu:
\begin{itemize}
	\item Syntax Errors
	Syntax Errors adalah suatu keadaan saat kode python mengalami kesalahan penulisan. Solusinya adalah memperbaiki penulisan kode yang salah.
	
	\item Name Error
	NameError adalah exception yang terjadi saat kode melakukan eksekusi terhadap local name atau global name yang tidak terdefinisi. Solusinya adalah memastikan variabel atau function yang dipanggil ada atau tidak salah ketik.
	
	\item Type Error
	TypeError adalah exception yang akan terjadi apabila pada saat dilakukannya eksekusi terhadap suatu operasi atau fungsi dengan type object yang tidak sesuai. Solusi dari error ini adalah mengkoversi varibelnya sesuai dengan tipe data yang akan digunakan.
\end{itemize}

Fungsi yang menggunakan try except
\lstinputlisting[caption= Fungsi yang menggunakan try except .,firstline=55, lastline=67]{src/4/1174006/Teori/1174006.py}

\subsection{Kode Program Penanganan Error}
\begin{figure}[H]
	\includegraphics[width=10cm]{figures/4/1174006/Praktek/p1.png}
	\centering
\end{figure}

\subsection{Plagiat Penanganan Error}
\begin{figure}[H]
	\includegraphics[width=10cm]{figures/4/1174006/Praktek/plagiatpenanganan.png}
	\centering
\end{figure}

%%%%%%%%%%%%%%%%%%%%%%%%%%%%%%%%%%%%%%%%%%%%%%%%%%%%%%%%%%%%%%%%%%%%

\section{Damara Benedikta}
\subsection{Soal 1}
Berikut adalah pemanggilan file csv dengan library csv yang menggunakan list
\lstinputlisting[firstline=10, lastline=20]{src/4/1174012/praktek/c_1174012_csv.py}

\subsection{Soal 2}
Berikut adalah pemanggilan file csv dengan library csv yang menggunakan dictionary
\lstinputlisting[firstline=22, lastline=31]{src/4/1174012/praktek/c_1174012_csv.py}

\subsection{Soal 3}
Berikut adalah pemanggilan file csv dengan library pandas yang menggunakan list
\lstinputlisting[firstline=9, lastline=11]{src/4/1174012/praktek/p_1174012_pandas.py}

\subsection{Soal 4}
Berikut adalah pemanggilan file csv dengan library pandas yang menggunakan dictionary
\lstinputlisting[firstline=13, lastline=16]{src/4/1174012/praktek/p_1174012_pandas.py}

\subsection{Soal 5}
Berikut penggunaan untuk merubah standar penulisan tanggal, yang mengikuti standar penulisan dari pandas.
\lstinputlisting[firstline=18, lastline=20]{src/4/1174012/praktek/p_1174012_pandas.py}

\subsection{Soal 6}
Berikut merupakan pergantian index kolom
\lstinputlisting[firstline=22, lastline=24]{src/4/1174012/praktek/p_1174012_pandas.py}

\subsection{Soal 7}
berikut merupakan penggunaan untuk merename atribut yang digunakan, atau merubah nama header 0
\lstinputlisting[firstline=26, lastline=30]{src/4/1174012/praktek/p_1174012_pandas.py}

\subsection{Soal 8}
\lstinputlisting[firstline=8, lastline=10]{src/4/1174012/praktek/main_damara.py}

\subsection{Soal 9}
\lstinputlisting[firstline=11, lastline=14]{src/4/1174012/praktek/main_damara.py}

\subsection{Penanganan Error}
Tidak ada error
%%%%%%%%%%%%%%%%%%%%%%%%%%%%%%%%%%%%%%%%%%%%%%%%%%%%
\section{Felix Setiawan Lase}
\subsection{Soal 1}
Buatlah  fungsi  (file  terpisah/library  dengan  nama  NPMcsv.py)  untuk  membuka file csv dengan lib csv mode list.

\lstinputlisting[caption = Fungsi untuk membuka file CSV dengan lib CSV mode list., firstline=10, lastline=15]{src/4/1174026/Praktek/1174026_csv.py}

\subsection{Soal 2}
Buatlah  fungsi  (file  terpisah/library  dengan  nama  NPMcsv.py)  untuk  membuka file csv dengan lib csv mode dictionary.

\lstinputlisting[caption =  Fungsi untuk membuka file CSV dengan lib CSV mode dictionary., firstline=17, lastline=22]{src/4/1174026/Praktek/1174026_csv.py}

\subsection{Soal 3}
Buatlah fungsi (file terpisah/library dengan nama NPMpandas.py) untuk membuka file csv dengan lib pandas mode list.

\lstinputlisting[caption =  Fungsi untuk membuka file CSV dengan lib Pandas mode list., firstline=10, lastline=13]{src/4/1174026/Praktek/1174026_pandas.py}

\subsection{Soal 4}
Buatlah fungsi (file terpisah/library dengan nama NPMpandas.py) untuk membuka file csv dengan lib pandas mode dictionary.

\lstinputlisting[caption =  Fungsi untuk membuka file CSV dengan lib Pandas mode dictionary., firstline=10, lastline=13]{src/4/1174026/Praktek/1174026_pandas.py}

\subsection{Soal 5}
Buat fungsi baru di NPMpandas.py untuk mengubah format tanggal menjadi standar dataframe.

\lstinputlisting[caption =  Fungsi untuk mengubah format tanggal menjadi standar dataframe., firstline=15, lastline=19]{src/4/1174026/Praktek/1174026_pandas.py}

\subsection{Soal 6}
Buat fungsi baru di NPMpandas.py untuk mengubah index kolom.

\lstinputlisting[caption =  Fungsi untuk mengubah index kolom., firstline=21, lastline=24]{src/4/1174026/Praktek/1174026_pandas.py}

\subsection{Soal 7}
Buat fungsi baru di NPMpandas.py untuk mengubah atribut atau nama kolom.

\lstinputlisting[caption =  Fungsi untuk mengubah atribut atau nama kolom., firstline=26, lastline=30]{src/4/1174026/Praktek/1174026_pandas.py}

\subsection{Soal 8}
Buat program main.py yang menggunakan library NPMcsv.py yang membuat dan membaca file csv.

\lstinputlisting[caption =  Membuat dan mebaca file CSV menggunakan library 1174006pandas., firstline=8, lastline=13]{src/4/1174026/Praktek/main.py}

\subsection{Soal 9}
Buat program main2.py yang menggunakan library NPMpandas.py yang membuat dan membaca file csv.

\lstinputlisting[caption = Membuat dan mmebaca file CSV menggunakan library 1174006pandas., firstline=8, lastline=13]{src/4/1174026/Praktek/main2.py}

\subsection{Kode Program Praktek}
\begin{figure}[H]
	\includegraphics[width=9cm]{figures/4/1174026/Praktek/k1.png}
	\centering
\end{figure}
\begin{figure}[H]
	\includegraphics[width=10cm]{figures/4/1174026/Praktek/k2.png}
	\centering
\end{figure}
\begin{figure}[H]
	\includegraphics[width=10cm]{figures/4/1174026/Praktek/k3.png}
	\centering
\end{figure}
\begin{figure}[H]
	\includegraphics[width=9cm]{figures/4/1174026/Praktek/k4.png}
	\centering
\end{figure}
\begin{figure}[H]
	\includegraphics[width=10cm]{figures/4/1174026/Praktek/k5.png}
	\centering
\end{figure}

\subsection{Cek Plagiat Praktek}
\begin{figure}[H]
	\includegraphics[width=10cm]{figures/4/1174026/Praktek/plagiatketerampilan.png}
	\centering
\end{figure}

\subsection{Soal 1}
Tuliskan  peringatan  error  yang  didapat  dari  mengerjakan  praktek  keempat  ini, dan  jelaskan  cara  penanganan  error  tersebut.   dan  Buatlah  satu  fungsi  yang menggunakan gunakan try except untuk menanggulangi error tersebut.

Peringatan error di praktek keempat ini, yaitu:
\begin{itemize}
	\item Syntax Errors
	Kesalahan Sintaksis adalah suatu kondisi ketika kode python mengalami kesalahan penulisan. Solusinya adalah memperbaiki penulisan kode yang salah.
	
	\item Name Error
	NameError adalah pengecualian yang terjadi ketika kode mengeksekusi nama lokal atau nama global yang tidak ditentukan. Solusinya adalah memastikan variabel atau fungsi yang dipanggil ada atau tidak salah ketik.
	
	\item Type Error
	TypeError adalah pengecualian yang akan terjadi jika eksekusi operasi atau fungsi dengan tipe objek tidak sesuai ketika dieksekusi. Solusi untuk kesalahan ini adalah mengubah variabel sesuai dengan tipe data yang akan digunakan.
\end{itemize}

Fungsi yang menggunakan try except
\lstinputlisting[caption= Fungsi yang menggunakan try except .,firstline=55, lastline=67]{src/4/1174006/Teori/1174026.py}

\subsection{Kode Program Penanganan Error}
\begin{figure}[H]
	\includegraphics[width=10cm]{figures/4/1174026/Praktek/p1.png}
	\centering
\end{figure}

\subsection{Plagiat Penanganan Error}
\begin{figure}[H]
	\includegraphics[width=10cm]{figures/4/1174026/Praktek/plagiatpenanganan.png}
	\centering
\end{figure}

%%%%%%%%%%%%%%%%%%%%%%%%%%%%%%%%%%%%%%%%%%%%%%%%%%%%

\section{Harun Ar-Rasyid}
\subsection{Soal 1}
Isi jawaban soal ke-1

Kalau mau dibikin paragrap \textbf{cukup enter aja}, tidak usah pakai \verb|par| dsb

%\subsection{Soal 2}
%Isi jawaban soal ke-2

%\subsection{Soal 3}
%Isi jawaban soal ke-3

\section{Sri Rahayu}
\subsection{Soal 1}
Isi jawaban soal ke-1

Kalau mau dibikin paragrap \textbf{cukup enter aja}, tidak usah pakai \verb|par| dsb

%\subsection{Soal 2}
%Isi jawaban soal ke-2

%\subsection{Soal 3}
%Isi jawaban soal ke-3

\section{Doli Jonviter}
\subsection{Soal 1}
Isi jawaban soal ke-1

Kalau mau dibikin paragrap \textbf{cukup enter aja}, tidak usah pakai \verb|par| dsb

%\subsection{Soal 2}
%Isi jawaban soal ke-2

%\subsection{Soal 3}
%Isi jawaban soal ke-3

\section{Rahmatul Ridha}
\subsection{Soal 1}
Isi jawaban soal ke-1

Kalau mau dibikin paragrap \textbf{cukup enter aja}, tidak usah pakai \verb|par| dsb

%\subsection{Soal 2}
%Isi jawaban soal ke-2

%\subsection{Soal 3}
%Isi jawaban soal ke-3

\section{Tomy Prawoto}
\subsection{Soal 1}
Isi jawaban soal ke-1

Kalau mau dibikin paragrap \textbf{cukup enter aja}, tidak usah pakai \verb|par| dsb

%\subsection{Soal 2}
%Isi jawaban soal ke-2

%\subsection{Soal 3}
%Isi jawaban soal ke-3


%TEORI
\chapter{PySerial}
%\section{Kadek Diva Krishna Murti}
{\Large \textbf{Pemahaman Teori}}
\subsection{Soal No. 1}
Apa itu fungsi device manager di windows dan folder /dev di linux?

\hfill \break
Device manager merupakan perangkat lunak untuk menampilkan seluruh perangkat keras yang di-inisialisasi atau dikenali oleh sistem operasi Windows. Device Manager membantu dalam mengelola atau me-manage semua perangkat keras yang terpasang dan terdeteksi dalam sistem Windows. Perangkat keras tersebut bisa berupa harddisk, kartu VGA, sound, keyboard, perangkat USB dan lain-lainnya.

\hfill \break
Fungsi device manager antara lain :
\begin{enumerate}
	\item Menunjukkan status mengenai suatu perangkat keras.
	\item Menunjukkan informasi detail mengenai suatu perangkat keras.
	\item Mengelola driver perangkat keras.
	\item Menonaktifkan dan mengaktifkan perangkat keras.
	\item Mengidentifikasi konflik antar perangkat keras.
	\item Memberitahukan terjadinya masalah pada perangkat keras.
\end{enumerate}

\hfill \break
Folder /dev merupakan representasi dari drive yang terhubung ke sistem operasi Linux dan oleh sistem dianggap sebagai file-file direktori. Biasanya sering ditampilkan direktori seperti /dev/sda1 yang mewakili Drive SATA pertama dalam sistem.

\subsection{Soal No. 2}
Jelaskan langkah-langkah instalasi driver dari arduino!

\hfill \break
Berikut ini adalah langkah-langkah instalasi driver dari Arduino UNO di Windows:

\begin{enumerate}
	\item Pertama pastikan Arduino IDE telah terinstall.
	\item Lalu hubungkan port USB Arduino Uno ke port USB PC.
	\item Kemudian PC anda akan mendeteksi perangkat baru yang terpasang dan akan muncul pop seperti ini.
	\begin{figure}[H]
		\includegraphics[width=10cm]{figures/5/1174006/Teori/1.png}
		\centering
	\end{figure}
	\item Karena Arduino Uno baru pertama kali terpasang, maka akan muncul pop up error seperti ini.
	\begin{figure}[H]
		\includegraphics[width=10cm]{figures/5/1174006/Teori/2.png}
		\centering
	\end{figure}
	\item Buka ''Start'' lalu cari Device Manager, kemudian klik ''Device Manager''.
	\begin{figure}[H]
		\includegraphics[width=10cm]{figures/5/1174006/Teori/3.png}
		\centering
	\end{figure}
	\item Setelah Device Manager terbuka, silahkan cari ''Unknown Device'' yang berada di Other Device.
	\begin{figure}[H]
		\includegraphics[width=10cm]{figures/5/1174006/Teori/4.png}
		\centering
	\end{figure}
	\item Kemudian klik kanan pada ''Unknown Device'', lalu pilih ''Update Driver Software''.
	\begin{figure}[H]
		\includegraphics[width=10cm]{figures/5/1174006/Teori/5.png}
		\centering
	\end{figure}
	\item Setelah itu muncul window baru, lalu pilih ''Browse my computer for driver software''.
	\begin{figure}[H]
		\includegraphics[width=10cm]{figures/5/1174006/Teori/6.png}
		\centering
	\end{figure}
	\item Lalu cari folder yang terinstall Arduino IDE dengan mengklik browse. Kemudian klik ''Next''.
	\begin{figure}[H]
		\includegraphics[width=10cm]{figures/5/1174006/Teori/7.png}
		\centering
	\end{figure}
	\item Windows akan mencari dan menginstall driver yang berada pada folder tersebut.
	\begin{figure}[H]
		\includegraphics[width=10cm]{figures/5/1174006/Teori/8.png}
		\centering
	\end{figure}
	\item Setelah itu akan muncul window, lalu klik ''Install''.
	\begin{figure}[H]
		\includegraphics[width=10cm]{figures/5/1174006/Teori/9.png}
		\centering
	\end{figure}
	\item Jika berhasil terinstal maka akan muncul window seperti ini.
	\begin{figure}[H]
		\includegraphics[width=10cm]{figures/5/1174006/Teori/10.png}
		\centering
	\end{figure}
\end{enumerate}

\subsection{Soal No. 3}
Jelaskan bagaimana cara membaca baudrate dan port dari komputer yang sudah terinstall driver!

\hfill \break
\textbf{Membaca Baudrate dari Komputer}
\begin{enumerate}
	\item Pertama buka ''Start''. Cari ''Device Manager'', lalu klik.
	\begin{figure}[H]
		\includegraphics[width=10cm]{figures/5/1174006/Teori/d1.png}
		\centering
	\end{figure}
	
	\item Kemudian pilih ''Ports (COM \& LPT)''.
	\begin{figure}[H]
		\includegraphics[width=10cm]{figures/5/1174006/Teori/d3.png}
		\centering
	\end{figure}
	
	\item Klik dua kali pada COM yang terhubung.
	\begin{figure}[H]
		\includegraphics[width=10cm]{figures/5/1174006/Teori/d2.png}
		\centering
	\end{figure}

	\item Pilih tab ''Port Settings'', lalu lihat di ''Bit per second''.
	\begin{figure}[H]
		\includegraphics[width=8cm]{figures/5/1174006/Teori/d4.png}
		\centering
	\end{figure}
\end{enumerate}


\hfill \break
\textbf{Membaca Port dari Komputer}

\begin{enumerate}
	\item Pertama buka ''Start''. Cari ''Device Manager'', lalu klik.
	\begin{figure}[H]
		\includegraphics[width=10cm]{figures/5/1174006/Teori/d1.png}
		\centering
	\end{figure}

	\item Kemudian pilih ''Ports (COM \& LPT)''.
	\begin{figure}[H]
		\includegraphics[width=10cm]{figures/5/1174006/Teori/d3.png}
		\centering
	\end{figure}

	\item Port dari Arduino telah terbaca oleh PC.
	\begin{figure}[H]
		\includegraphics[width=10cm]{figures/5/1174006/Teori/d2.png}
		\centering
	\end{figure}
\end{enumerate}



\subsection{Soal No. 4}
Jelaskan sejarah library pyserial!

\hfill \break
PySerial adalah paket Python yang menfasilitasi komunikasi serial antara PC dengan perangkat keras eksternal. PySerial menyediakan antarmuka untuk berkomunikasi melalui protokol komunikasi serial. Komunikasi serial adalah salah satu protokol komunikasi komputer tertua. Protokol komunikasi serial mendahului spesifikasi USB yang digunakan oleh komputer dan perangkat keras lain seperti mouse, keyboard, dan webcam. USB adalah singkatan dari Universal Serial Bus. USB dan dibangun di atas dan memperluas antarmuka komunikasi serial asli.

\subsection{Soal No. 5}
Jelaskan fungsi-fungsi apa saja yang dipakai dari library pyserial!

\hfill \break
Fungsi-fungsi yang dipakai dari library PySerial, yaitu:
\begin{enumerate}
	\item Serial - fungsi ini untuk membuka port serial.
	\item write(data) - fungsi ini menulis data lewat port serial.
	\item readline() - fungsi ini membaca sebuah string dari port serial.
	\item read(size) - fungsi ini untuk membaca jumlah byte dari port serial.
	\item close() - fungsi ini untuk menutup port serial.
\end{enumerate}

\subsection{Soal No. 6}
Jelaskan kenapa butuh perulangan dan tidak butuh perulangan dalam membaca serial!

\hfill \break
Pada saat membaca serial di Arduino diperlukan perulangan agar bisa membaca data secara berulang kali sehingga data yang muncul banyak. Sedangkan apabila tidak membutuhkan perulangan maka Arduino hanya akan membaca data sekali saja.

\subsection{Soal No. 7}
Jelaskan bagaimana cara membuat fungsi yang mengunakan pyserial!

\lstinputlisting[caption = Fungsi yang menggunakan pyserial., firstline=1, lastline=7]{src/5/1174006/Teori/1174006.py}

\begin{figure}[H]
	\includegraphics[width=10cm]{figures/5/1174006/Teori/hasil.png}
	\centering
	\caption{Hasil pembuatan fungsi pyserial.}
\end{figure}

\subsection{Cek Plagiat}
\begin{figure}[H]
	\includegraphics[width=10cm]{figures/5/1174006/Teori/plagiat.png}
	\centering
	\caption{Hasil cek plagiat.}
\end{figure}

\subsection{Kode Program}
\begin{figure}[H]
	\includegraphics[width=10cm]{figures/5/1174006/Teori/kodeprogram.png}
	\centering
	\caption{Kode program file 1174006.py.}
\end{figure}

%%%%%%%%%%%%%%%%%%%%%%%%%%%%%%%%%%%%%%%%%%%%%%%%%%%%%%%%%%%%%%%%%%%%%%%%%%%%%%%%%%%%%

\section{Muhammad Tomy Nur Maulidy}
{\Large \textbf{Pemahaman Teori}}
\subsection{Soal No. 1}
Apa itu fungsi device manager di windows dan folder /dev di linux?

\hfill \break
Fungsi device manager antara lain :
\begin{enumerate}
	\item Menunjukkan status suatu hardware.
	\item Menunjukkan informasi detil suatu hardware.
	\item Mengelola driver hardware
	\item Disable dan Enable hardware
	\item Mengidentifikasi konflik antar perangkat keras.
\end{enumerate}

\hfill \break
Folder /dev berisi file device, baik device blok maupun device karakter. Di dalamnya setodaknya ada file biner yang beernama MAKEDEV untuk membuat device secara manual.

\subsection{Soal No. 2}
Jelaskan langkah-langkah instalasi driver dari arduino!

\hfill \break
Berikut ini adalah langkah-langkah instalasi driver dari Arduino UNO di Windows:

\begin{enumerate}
	\item Hubungkan sistem minimun Arduino Uno ke komputer dengan kabel USB type B (kabel Printer).
	\item Lalu pada bagian kanan didesktop PC anda, akan muncul popup “Installing device driver software”.
	\item SIstem operasi Windows tidak menyediakan driver untuk Arduino Uno.
	\item Buka Device Manager, caranya pada bagian Search Program and Files lalu ketikkan “device manager” (tanpa tanda petik). Kemudian bagian Control Panel akan muncul halaman Device Manager, selanjutnya klik untuk menjalankan.
	\item Cari yang bernama Unknown device yang berada pada bagian Other device, biasanya ada tanda seru berwarna kuning, itu disebabkan karena penginstallan tidak berjalan dengan sempurna.
	\item Klik kanan pada “Unknown device” kemudian pilih Update Driver Software.
	\item Pilih Browse my computer for driver software.
	\item Arahkan lokasi folder ke folder ..arduino-1.0.5 drivers. Pastikan check-box lalu centang include subfolders. Klik Next untuk melanjutkan instalasi driver.
	\item Kemudian lanjutkan dengan mengklik Install pada tampilan Windows Security.
	\item Jika instalasi driver berhasil maka akan muncul Windows has successfully updated your driver software.
	\item Perhatikan dan ingat nama COM Arduino Uno, karena nama COM ini yang akan digunakan untuk meng-upload program nantinya.
\end{enumerate}

\subsection{Soal No. 3}
Jelaskan bagaimana cara membaca baudrate dan port dari komputer yang sudah terinstall driver!

\hfill \break
\textbf{Membaca Port dari Komputer}

\begin{enumerate}
	\item Hubungkan modul TX-RX serial dengan komputer melalui serial port menggunakan DB9 cable extension.
	\item Buka Hyper Terminal dengan menekan start kemudian All progams lalu Accessories kemudian Communications lalu Hyper Terminal.
	\item Ketik nama untuk Connection Description, misal coba, kemudian tekan OK.
	\item Pada Connect to, pilihlah COM port yang dipakai di Connect using, kemudian tekan OK.
	\item Masukkan nilai-nilai port settingnya, sesuai dengan DCE-nya. Kemudian tekan OK.
\end{enumerate}



\subsection{Soal No. 4}
Jelaskan sejarah library pyserial!

\hfill \break
PySerial adalah library/modul Python siap-pakai dan gratis yang dibuat untuk memudahkan kita dalam membuat program komunikasi data serial RS232 dalam bahasa Python.
Jika modul USB-2REL dapat kita kontrol dengan mudah menggunakan Python dan PyUSB (lihat pembahasannya di sini dan di sini), maka modul SER-2REL juga dapat kita kontrol dengan mudah menggunakan Python dengan bantuan modul PySerial.

\subsection{Soal No. 5}
Jelaskan fungsi-fungsi apa saja yang dipakai dari library pyserial!

\hfill \break
Fungsi-fungsi yang dipakai dari library PySerial, yaitu:
\begin{enumerate}
	\item Serial - fungsi ini untuk membuka port serial.
	\item write(data) - fungsi ini menulis data lewat port serial.
	\item readline() - fungsi ini membaca sebuah string dari port serial.
	\item read(size) - fungsi ini untuk membaca jumlah byte dari port serial.
	\item close() - fungsi ini untuk menutup port serial.
\end{enumerate}

\subsection{Soal No. 6}
Jelaskan kenapa butuh perulangan dan tidak butuh perulangan dalam membaca serial!

\hfill \break
Pada saat membaca serial di Arduino diperlukan perulangan agar bisa membaca data secara berulang kali sehingga data yang muncul banyak. Sedangkan apabila tidak membutuhkan perulangan maka Arduino hanya akan membaca data sekali saja.

\subsection{Soal No. 7}
Jelaskan bagaimana cara membuat fungsi yang mengunakan pyserial!

\hfill \break
Fungsi yang berada pada Python, dibuat dengan nama kata kunci def kemudian diikuti dengan nama fungsinya pada pyhton.
Seperti halnya dengan blok kode yang lain, kita juga harus memberikan identasi untuk menuliskan isi fungsi.

\subsection{Cek Plagiat}
\begin{figure}[H]
	\includegraphics[width=10cm]{figures/5/1174031/Teori/Plagiat.png}
	\centering
	\caption{Hasil cek plagiat.}
\end{figure}


%PRAKTEK
\chapter{Praktek PySerial}
%\section{Kadek Diva Krishna Murti}
{\Large \textbf{Ketrampilan Pemrograman}}
\subsection{Soal No. 1}
Buatlah  fungsi  (file  terpisah/library  dengan  nama  NPMrealtime.py)  untuk mendapatkan data langsung dari arduino!

\subsection{Soal No. 2}
Buatlah fungsi (file terpisah/library dengan nama NPMsave.py) untuk mendapatkan data langsung dari arduino dengan looping!

\subsection{Soal No. 3}
Buatlah  fungsi  (file  terpisah/library  dengan  nama  NPMrealtime.py)  untuk mendapatkan data dari arduino dan langsung ditulis kedalam file csv!

\subsection{Soal No. 4}
Buatlah fungsi (file terpisah/library dengan nama NPMcsv.py) untuk membaca file csv hasil arduino dan mengembalikan ke fungsi!


\hfill \break
{\Large \textbf{Ketrampilan Penanganan Error}}

\subsection{Soal No. 1}
Tuliskan  peringatan  error  yang  didapat  dari  mengerjakan  praktek  ketiga  ini,dan  jelaskan  cara  penanganan  error  tersebut.   dan  Buatlah  satu  fungsi  yangmenggunakan gunakan try except untuk menanggulangi error tersebut.

\chapter{Matplotlib}
%%%%%%%%%%%%%%%%%%%%%%%%%%%%%%%%%%%%%%%%%%%%%%%%%%%%%%%%%%%%%%
\section{Muh. Rifky Prananda (1174017)}
\subsection{Teori}
\subsubsection{Soal No. 1}
\hfill \break
Apa itu fungsi library matplotlib?

\hfill \break
Matplotlib adalah salah satu perpustakaan Python 2D yang dapat menghasilkan plot kualitas lebih tinggi dalam berbagai format dan dapat digunakan pada berbagai platform. Matplotlib berfungsi sebagai pembuat grafik di berbagai platform, seperti Jupyter dan Python.

\subsubsection{Soal No. 2}
\hfill \break
Jelaskan langkah-langkah membuat sumbu X dan Y di matplotlib!

\begin{enumerate}
	\item Pertama yaitu memasukkan atau mengimport library.	
	\lstinputlisting[firstline=2, lastline=2]{src/6/1174017/1174017.py}
	
	\item Selanjutnya menghasilkan nilai sumbu x dan sumbu y.	
	\lstinputlisting[firstline=4, lastline=5]{src/6/1174017/1174017.py}
	
	\item Kemudian membuat fungsi untuk mem-plot diagram batang.
	\lstinputlisting[firstline=7, lastline=7]{src/6/1174017/1174017.py}	

	\item Terakhir kita menampilkan plot nya.
	\lstinputlisting[firstline=9, lastline=9]{src/6/1174017/1174017.py}
	
\end{enumerate}
\hfill \break
\textbf{Kode Program}

\lstinputlisting[caption = Kode program membuat diagram menggunakan Matplotlib., firstline=2, lastline=9]{src/6/1174017/1174017.py}

\hfill \break
\textbf{Gambar yang dihasilkan}

\begin{figure}[H]
	\includegraphics[width=12cm]{figures/6/1174017/2.png}
	\centering
	\caption{Diagram Batang}
\end{figure}
 
\subsubsection{Soal No. 3}
\hfill \break
Jelaskan bagaimana perbedaan fungsi dan cara pakai untuk berbagai jenis(bar, histogram ,scatter ,line, dll) jenis plot di matplotlib!

\begin{enumerate}
	\item \textbf{Bar Graph}
	
	Perbedaan antara grafik batang dan jenis plot lainnya adalah grafik batang menggunakan bar atau balok (batang) untuk membandingkan data antara berbagai kategori.
	
	\textbf{Kode Program}
	
	\lstinputlisting[caption = Kode program membuat bar graph menggunakan Matplotlib., firstline=2, lastline=9]{src/6/1174017/1174017.py}
	
	\textbf{Hasil Compile}
	
	\begin{figure}[H]
		\includegraphics[width=12cm]{figures/6/1174017/bar.png}
		\centering
		\caption{Hasil compile membuat bar graph menggunakan Matplotlib.}
	\end{figure}
	
	\item \textbf{Histogram}
	
	Perbedaan antara histogram dan tipe plot lainnya adalah histogram akan membuat plot di mana plot yang diangkat adalah kombinasi dari beberapa data yang telah dikelompokkan.
	
	\textbf{Kode Program}
	
	\lstinputlisting[caption = Kode program membuat histogram menggunakan Matplotlib., firstline=29, lastline=36]{src/6/1174017/1174017.py}
	
	\textbf{Hasil Compile}
	
	\begin{figure}[H]
		\includegraphics[width=12cm]{figures/6/1174017/histogram.png}
		\centering
		\caption{Hasil compile membuat histogram menggunakan Matplotlib.}
	\end{figure}
	
	\item \textbf{Scatter Plot}
	
	Perbedaan antara Scatter plot dan jenis plot lainnya adalah bahwa scatter plot menampilkan data sebagai kumpulan titik, yang masing-masing memiliki nilai satu variabel yang menentukan posisi pada sumbu horizontal dan nilai variabel lain menentukan posisi pada sumbu vertikal.
	
	\textbf{Kode Program}
	
	\lstinputlisting[caption = Kode program membuat scatter plot menggunakan Matplotlib., firstline=40, lastline=53]{src/6/1174017/1174017.py}
	
	\textbf{Hasil Compile}
	
	\begin{figure}[H]
		\includegraphics[width=12cm]{figures/6/1174017/scatter.png}
		\centering
		\caption{Hasil compile membuat scatter plot menggunakan Matplotlib.}
	\end{figure}
	
	\item \textbf{Area Plot}
	
	Perbedaan area plot dengan tipe plot lain adalah area plot dapat digunakan buat melacak perubahan dari waktu ke waktu untuk dua atau lebih kelompok terkait yang dapat membentuk satu kategori secara menyeluruh.
	
	\textbf{Kode Program}
	
	\lstinputlisting[caption = Kode program membuat diagram menggunakan Matplotlib., firstline=57, lastline=76]{src/6/1174017/1174017.py}
	
	\textbf{Hasil Compile}
	
	\begin{figure}[H]
		\includegraphics[width=12cm]{figures/6/1174017/area.png}
		\centering
		\caption{Hasil compile membuat diagram menggunakan Matplotlib.}
	\end{figure}
	
	\item \textbf{Pie Plot}
	
	Perbedaan pie plot dengan jenis plot yang lainnya yaitu pie plot digunakan untuk bisa menunjukkan presentase atau data proporsional di mana di setiap potongan pie dapat mewakili kategori.
	
	\textbf{Kode Program}
	
	\lstinputlisting[caption = Kode program membuat Pie Plot menggunakan Matplotlib., firstline=80, lastline=101]{src/6/1174017/1174017.py}
	
	\textbf{Hasil Compile}
	
	\begin{figure}[H]
		\includegraphics[width=9cm]{figures/6/1174017/pie.png}
		\centering
		\caption{Hasil compile membuat Pie Plot menggunakan Matplotlib.}
	\end{figure}
	
	\item \textbf{Line Graph}
	
	Perbedaan line graph dengan jenis plot lain adalah line graph menampilkan diagram dalam bentuk garis.
	
	\textbf{Kode Program}
	
	\lstinputlisting[caption = Kode program membuat diagram menggunakan Matplotlib., firstline=105, lastline=113]{src/6/1174017/1174017.py}
	
	\textbf{Hasil Compile}
	
	\begin{figure}[H]
		\includegraphics[width=12cm]{figures/6/1174017/line.png}
		\centering
		\caption{Hasil compile membuat diagram menggunakan Matplotlib.}
	\end{figure}
	
\end{enumerate}

\subsubsection{Soal No. 4}
\hfill \break
Jelaskan bagaimana cara menggunakan legend dan label serta kaitannya dengan fungsi tersebut!

\textbf{Legend}
Legend merupakan pendefinisian garis yang dilengkapi dengan sampel garis yang dijelaskan. Untuk bisa membuat legenda pada plot kita dapat menggunakan syntax fungsi legend pada MATLAB. 

\textbf{Label}
Untuk menambah label pada garis sumbu pada grafik dapat menggunakan syntax fungsi xlabel dan fungsi ylabel pada MATLAB. Kedua label ditulis setelah syntax deklarasi plot.

\subsubsection{Soal No. 5}
\hfill \break
Jelaskan apa fungsi dari subplot di matplotlib, dan bagaimana cara kerja dari fungsi subplot, sertakan ilustrasi dan gambar sendiri dan apa parameternya jika ingin menggambar plot dengan 9 subplot di dalamnya!

\hfill \break
Fungsi suatu subplot yaitu untuk membuat beberapa plot di dalam satu gambar.
\hfill \break
Cara kerja subplot, yaitu fungsi subplot memiliki parameter pertama adalah jumlah kolom, parameter kedua adalah jumlah baris, dan parameter ketiga adalah index plot keberapanya.

\hfill \break
\textbf{Kode Program}

\lstinputlisting[caption = Kode program membuat subplot menggunakan Matplotlib., firstline=134, lastline=146]{src/6/1174017/1174017.py}

\hfill \break
\textbf{Hasil Compile}

\begin{figure}[H]
	\includegraphics[width=12cm]{figures/6/1174017/subplot.png}
	\centering
	\caption{Hasil compile membuat subplot menggunakan Matplotlib.}
\end{figure}

\subsubsection{Soal No. 6}
\hfill \break
Sebutkan semua parameter color yang bisa digunakan (contoh:  m,c,r,k,...  dkk)!

\begin{itemize}
	\item 'b' (blue)
	\item 'g' (green)
	\item 'r' (red)
	\item 'c' (cyan)
	\item 'm' (magenta)
	\item 'y' (yellow)
	\item 'k' (black)
	\item 'w' (white)
\end{itemize}

\subsubsection{Soal No. 7}
\hfill \break
Jelaskan bagaimana cara kerja dari fungsi hist, sertakan ilustrasi dan gambar sendiri!

\hfill \break
Cara kerja dari sebuah fungsi hist adalah fungsi hist akan menerima parameter yang telah diberikan, selanjutnya fungsi hist akan bekerja sesuai dengan parameter yang diberikan.

\hfill \break
\textbf{Kode Program}

\lstinputlisting[caption = Kode program membuat diagram menggunakan Matplotlib., firstline=150, lastline=157]{src/6/1174017/1174017.py}

\hfill \break
\textbf{Hasil Compile}

\begin{figure}[H]
	\includegraphics[width=12cm]{figures/6/1174017/histogram.png}
	\centering
	\caption{Hasil compile membuat diagram menggunakan Matplotlib.}
\end{figure}

\subsubsection{Soal No. 8}
\hfill \break
 Jelaskan lebih mendalam tentang parameter dari fungsi pie diantaranya labels, colors, startangle, shadow, explode, autopct!
 
 \begin{itemize}
 	\item labels : yaitu untuk memberi label di setiap persentase.
 	\item colors : yaitu untuk memberikan warna di tiap persentase.
 	\item startangle : yaitu untuk memutar plot sesuai dengan derajat yang ditentukan.
 	\item shadow : yaitu untuk memberikan bayangan pada plot.
 	\item explode : yaitu untuk memisahkan antar tiap potongan pie di plot.
 	\item autopct : yaitu untuk menentukan jumlah angka yang berada dibelakang koma.
 \end{itemize}

\subsection{Praktek}
\subsubsection{Soal No. 1}
\hfill \break
Buatlah librari fungsi (file terpisah/library dengan nama NPMbar.py) untuk plot dengan jumlah subplot adalah NPM mod 3 + 2!

\hfill \break
\textbf{Kode Program}

\lstinputlisting[caption = Kode program membuat fungsi Bar Plot menggunakan Matplotlib., firstline=1, lastline=21]{src/6/1174017/1174017_bar.py}

\hfill \break
\textbf{Hasil Compile}

\begin{figure}[H]
	\includegraphics[width=12cm]{figures/6/1174017/p1.png}
	\centering
	\caption{Hasil compile membuat fungsi Bar Plot menggunakan Matplotlib.}
\end{figure}

\subsubsection{Soal No. 2}
\hfill \break
Buatlah librari fungsi (file terpisah/library dengan nama NPMscatter.py) untuk plot dengan jumlah subplot NPM mod 3 + 2!

\hfill \break
\textbf{Kode Program}

\lstinputlisting[caption = Kode program membuat fungsi Scatter Plot menggunakan Matplotlib., firstline=1, lastline=23]{src/6/1174017/1174017_scatter.py}

\hfill \break
\textbf{Hasil Compile}

\begin{figure}[H]
	\includegraphics[width=12cm]{figures/6/1174017/p2.png}
	\centering
	\caption{Hasil compile membuat fungsi Scatter Plot menggunakan Matplotlib.}
\end{figure}

\subsubsection{Soal No. 3}
\hfill \break
Buatlah librari fungsi (file terpisah/library dengan nama NPMpie.py) untuk plot dengan jumlah subplot NPM mod 3 + 2!

\hfill \break
\textbf{Kode Program}

\lstinputlisting[caption = Kode program membuat fungsi Pie Plot menggunakan Matplotlib., firstline=1, lastline=23]{src/6/1174017/1174017_pie.py}

\hfill \break
\textbf{Hasil Compile}

\begin{figure}[H]
	\includegraphics[width=12cm]{figures/6/1174017/p3.png}
	\centering
	\caption{Hasil compile membuat fungsi Pie Plot menggunakan Matplotlib.}
\end{figure}

\subsubsection{Soal No. 4}
\hfill \break
Buatlah librari fungsi (file terpisah/library dengan nama NPMplot.py) untuk plot dengan jumlah subplot NPM mod 3 + 2

\hfill \break
\textbf{Kode Program}

\lstinputlisting[caption = Kode program membuat fungsi Plot menggunakan Matplotlib., firstline=1, lastline=23]{src/6/1174017/1174017_plot.py}

\hfill \break
\textbf{Hasil Compile}

\begin{figure}[H]
	\includegraphics[width=12cm]{figures/6/1174017/p4.png}
	\centering
	\caption{Hasil compile membuat fungsi Plot menggunakan Matplotlib.}
\end{figure}


\subsection{Penanganan Error}
Tuliskan  peringatan  error  yang  didapat  dari  mengerjakan  praktek  keenam  ini, dan  jelaskan  cara  penanganan  error  tersebut. dan  Buatlah  satu  fungsi  yang menggunakan try except untuk menanggulangi error tersebut.

\hfill \break
Peringatan error di praktek kelima ini, yaitu:
\begin{itemize}
	\item Syntax Errors
	Syntax Errors adalah suatu keadaan saat kode python mengalami kesalahan penulisan. Solusinya adalah memperbaiki penulisan kode yang salah.
	
	\item Name Error
	NameError adalah exception yang terjadi saat kode melakukan eksekusi terhadap local name atau global name yang tidak terdefinisi. Solusinya adalah memastikan variabel atau function yang dipanggil ada atau tidak salah ketik.
	
	\item Type Error
	TypeError adalah exception yang akan terjadi apabila pada saat dilakukannya eksekusi terhadap suatu operasi atau fungsi dengan type object yang tidak sesuai. Solusi dari error ini adalah mengkoversi varibelnya sesuai dengan tipe data yang akan digunakan.
\end{itemize}
\hfill \break
Fungsi yang menggunakan try except untuk menanggulangi error.

\hfill \break
\textbf{Kode Program}

\lstinputlisting[caption = Kode program membuat fungsi penanganan error., firstline=161, lastline=178]{src/6/1174017/1174017.py}

\hfill \break
\textbf{Hasil Compile}

\begin{figure}[H]
	\includegraphics[width=12cm]{figures/6/1174017/error.png}
	\centering
	\caption{Hasil compile membuat fungsi penanganan error.}
\end{figure}


%%%%%%%%%%%%%%%
%%  The default LaTeX Index
%%  Don't need to add any commands before \begin{document}
\printindex

%%%% Making an index
%%
%% 1. Make index entries, don't leave any spaces so that they
%% will be sorted correctly.
%%
%% \index{term}
%% \index{term!subterm}
%% \index{term!subterm!subsubterm}
%%
%% 2. Run LaTeX several times to produce <filename>.idx
%%
%% 3. On command line, type  makeindx <filename> which
%% will produce <filename>.ind
%%
%% 4. Type \printindex to make the index appear in your book.
%%
%% 5. If you would like to edit <filename>.ind
%% you may do so. See docs.pdf for more information.
%%
%%%%%%%%%%%%%%%%%%%%%%%%%%%%%%

%%%%%%%%%%%%%% Making Multiple Indices %%%%%%%%%%%%%%%%
%% 1.
%% \usepackage{multind}
%% \makeindex{book}
%% \makeindex{authors}
%% \begin{document}
%%
%% 2.
%% % add index terms to your book, ie,
%% \index{book}{A term to go to the topic index}
%% \index{authors}{Put this author in the author index}
%%
%% \index{book}{Cows}
%% \index{book}{Cows!Jersey}
%% \index{book}{Cows!Jersey!Brown}
%%
%% \index{author}{Douglas Adams}
%% \index{author}{Boethius}
%% \index{author}{Mark Twain}
%%
%% 3. On command line type
%% makeindex topic
%% makeindex authors
%%
%% 4.
%% this is a Wiley command to make the indices print:
%% \multiprintindex{book}{Topic index}
%% \multiprintindex{authors}{Author index}

\end{document}

