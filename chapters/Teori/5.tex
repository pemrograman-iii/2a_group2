\section{Kadek Diva Krishna Murti}
{\Large \textbf{Pemahaman Teori}}
\subsection{Soal No. 1}
Apa itu fungsi device manager di windows dan folder /dev di linux?

\hfill \break
Device manager merupakan perangkat lunak untuk menampilkan seluruh perangkat keras yang di-inisialisasi atau dikenali oleh sistem operasi Windows. Device Manager membantu dalam mengelola atau me-manage semua perangkat keras yang terpasang dan terdeteksi dalam sistem Windows. Perangkat keras tersebut bisa berupa harddisk, kartu VGA, sound, keyboard, perangkat USB dan lain-lainnya.

\hfill \break
Fungsi device manager antara lain :
\begin{enumerate}
	\item Menunjukkan status mengenai suatu perangkat keras.
	\item Menunjukkan informasi detail mengenai suatu perangkat keras.
	\item Mengelola driver perangkat keras.
	\item Menonaktifkan dan mengaktifkan perangkat keras.
	\item Mengidentifikasi konflik antar perangkat keras.
	\item Memberitahukan terjadinya masalah pada perangkat keras.
\end{enumerate}

\hfill \break
Folder /dev merupakan representasi dari drive yang terhubung ke sistem operasi Linux dan oleh sistem dianggap sebagai file-file direktori. Biasanya sering ditampilkan direktori seperti /dev/sda1 yang mewakili Drive SATA pertama dalam sistem.

\subsection{Soal No. 2}
Jelaskan langkah-langkah instalasi driver dari arduino!

\hfill \break
Berikut ini adalah langkah-langkah instalasi driver dari Arduino UNO di Windows:

\begin{enumerate}
	\item Pertama pastikan Arduino IDE telah terinstall.
	\item Lalu hubungkan port USB Arduino Uno ke port USB PC.
	\item Kemudian PC anda akan mendeteksi perangkat baru yang terpasang dan akan muncul pop seperti ini.
	\begin{figure}[H]
		\includegraphics[width=10cm]{figures/5/1174006/Teori/1.png}
		\centering
	\end{figure}
	\item Karena Arduino Uno baru pertama kali terpasang, maka akan muncul pop up error seperti ini.
	\begin{figure}[H]
		\includegraphics[width=10cm]{figures/5/1174006/Teori/2.png}
		\centering
	\end{figure}
	\item Buka ''Start'' lalu cari Device Manager, kemudian klik ''Device Manager''.
	\begin{figure}[H]
		\includegraphics[width=10cm]{figures/5/1174006/Teori/3.png}
		\centering
	\end{figure}
	\item Setelah Device Manager terbuka, silahkan cari ''Unknown Device'' yang berada di Other Device.
	\begin{figure}[H]
		\includegraphics[width=10cm]{figures/5/1174006/Teori/4.png}
		\centering
	\end{figure}
	\item Kemudian klik kanan pada ''Unknown Device'', lalu pilih ''Update Driver Software''.
	\begin{figure}[H]
		\includegraphics[width=10cm]{figures/5/1174006/Teori/5.png}
		\centering
	\end{figure}
	\item Setelah itu muncul window baru, lalu pilih ''Browse my computer for driver software''.
	\begin{figure}[H]
		\includegraphics[width=10cm]{figures/5/1174006/Teori/6.png}
		\centering
	\end{figure}
	\item Lalu cari folder yang terinstall Arduino IDE dengan mengklik browse. Kemudian klik ''Next''.
	\begin{figure}[H]
		\includegraphics[width=10cm]{figures/5/1174006/Teori/7.png}
		\centering
	\end{figure}
	\item Windows akan mencari dan menginstall driver yang berada pada folder tersebut.
	\begin{figure}[H]
		\includegraphics[width=10cm]{figures/5/1174006/Teori/8.png}
		\centering
	\end{figure}
	\item Setelah itu akan muncul window, lalu klik ''Install''.
	\begin{figure}[H]
		\includegraphics[width=10cm]{figures/5/1174006/Teori/9.png}
		\centering
	\end{figure}
	\item Jika berhasil terinstal maka akan muncul window seperti ini.
	\begin{figure}[H]
		\includegraphics[width=10cm]{figures/5/1174006/Teori/10.png}
		\centering
	\end{figure}
\end{enumerate}

\subsection{Soal No. 3}
Jelaskan bagaimana cara membaca baudrate dan port dari komputer yang sudah terinstall driver!

\hfill \break
\textbf{Membaca Baudrate dari Komputer}
\begin{enumerate}
	\item Pertama buka ''Start''. Cari ''Device Manager'', lalu klik.
	\begin{figure}[H]
		\includegraphics[width=10cm]{figures/5/1174006/Teori/d1.png}
		\centering
	\end{figure}
	
	\item Kemudian pilih ''Ports (COM \& LPT)''.
	\begin{figure}[H]
		\includegraphics[width=10cm]{figures/5/1174006/Teori/d3.png}
		\centering
	\end{figure}
	
	\item Klik dua kali pada COM yang terhubung.
	\begin{figure}[H]
		\includegraphics[width=10cm]{figures/5/1174006/Teori/d2.png}
		\centering
	\end{figure}

	\item Pilih tab ''Port Settings'', lalu lihat di ''Bit per second''.
	\begin{figure}[H]
		\includegraphics[width=8cm]{figures/5/1174006/Teori/d4.png}
		\centering
	\end{figure}
\end{enumerate}


\hfill \break
\textbf{Membaca Port dari Komputer}

\begin{enumerate}
	\item Pertama buka ''Start''. Cari ''Device Manager'', lalu klik.
	\begin{figure}[H]
		\includegraphics[width=10cm]{figures/5/1174006/Teori/d1.png}
		\centering
	\end{figure}

	\item Kemudian pilih ''Ports (COM \& LPT)''.
	\begin{figure}[H]
		\includegraphics[width=10cm]{figures/5/1174006/Teori/d3.png}
		\centering
	\end{figure}

	\item Port dari Arduino telah terbaca oleh PC.
	\begin{figure}[H]
		\includegraphics[width=10cm]{figures/5/1174006/Teori/d2.png}
		\centering
	\end{figure}
\end{enumerate}



\subsection{Soal No. 4}
Jelaskan sejarah library pyserial!

\hfill \break
PySerial adalah paket Python yang menfasilitasi komunikasi serial antara PC dengan perangkat keras eksternal. PySerial menyediakan antarmuka untuk berkomunikasi melalui protokol komunikasi serial. Komunikasi serial adalah salah satu protokol komunikasi komputer tertua. Protokol komunikasi serial mendahului spesifikasi USB yang digunakan oleh komputer dan perangkat keras lain seperti mouse, keyboard, dan webcam. USB adalah singkatan dari Universal Serial Bus. USB dan dibangun di atas dan memperluas antarmuka komunikasi serial asli.

\subsection{Soal No. 5}
Jelaskan fungsi-fungsi apa saja yang dipakai dari library pyserial!

\hfill \break
Fungsi-fungsi yang dipakai dari library PySerial, yaitu:
\begin{enumerate}
	\item Serial - fungsi ini untuk membuka port serial.
	\item write(data) - fungsi ini menulis data lewat port serial.
	\item readline() - fungsi ini membaca sebuah string dari port serial.
	\item read(size) - fungsi ini untuk membaca jumlah byte dari port serial.
	\item close() - fungsi ini untuk menutup port serial.
\end{enumerate}

\subsection{Soal No. 6}
Jelaskan kenapa butuh perulangan dan tidak butuh perulangan dalam membaca serial!

\hfill \break
Pada saat membaca serial di Arduino diperlukan perulangan agar bisa membaca data secara berulang kali sehingga data yang muncul banyak. Sedangkan apabila tidak membutuhkan perulangan maka Arduino hanya akan membaca data sekali saja.

\subsection{Soal No. 7}
Jelaskan bagaimana cara membuat fungsi yang mengunakan pyserial!

\lstinputlisting[caption = Fungsi yang menggunakan pyserial., firstline=1, lastline=7]{src/5/1174006/Teori/1174006.py}

\begin{figure}[H]
	\includegraphics[width=10cm]{figures/5/1174006/Teori/hasil.png}
	\centering
	\caption{Hasil pembuatan fungsi pyserial.}
\end{figure}

\subsection{Cek Plagiat}
\begin{figure}[H]
	\includegraphics[width=10cm]{figures/5/1174006/Teori/plagiat.png}
	\centering
	\caption{Hasil cek plagiat.}
\end{figure}

\subsection{Kode Program}
\begin{figure}[H]
	\includegraphics[width=10cm]{figures/5/1174006/Teori/kodeprogram.png}
	\centering
	\caption{Kode program file 1174006.py.}
\end{figure}

%%%%%%%%%%%%%%%%%%%%%%%%%%%%%%%%%%%%%%%%%%%%%%%%%%%%%%%%%%%%%%%%%%%%%%%%%%%%%%%%%%%%%

\section{Muhammad Tomy Nur Maulidy}
{\Large \textbf{Pemahaman Teori}}
\subsection{Soal No. 1}
Apa itu fungsi device manager di windows dan folder /dev di linux?

\hfill \break
Fungsi device manager antara lain :
\begin{enumerate}
	\item Menunjukkan status suatu hardware.
	\item Menunjukkan informasi detil suatu hardware.
	\item Mengelola driver hardware
	\item Disable dan Enable hardware
	\item Mengidentifikasi konflik antar perangkat keras.
\end{enumerate}

\hfill \break
Folder /dev berisi file device, baik device blok maupun device karakter. Di dalamnya setodaknya ada file biner yang beernama MAKEDEV untuk membuat device secara manual.

\subsection{Soal No. 2}
Jelaskan langkah-langkah instalasi driver dari arduino!

\hfill \break
Berikut ini adalah langkah-langkah instalasi driver dari Arduino UNO di Windows:

\begin{enumerate}
	\item Hubungkan sistem minimun Arduino Uno ke komputer dengan kabel USB type B (kabel Printer).
	\item Lalu pada bagian kanan didesktop PC anda, akan muncul popup “Installing device driver software”.
	\item SIstem operasi Windows tidak menyediakan driver untuk Arduino Uno.
	\item Buka Device Manager, caranya pada bagian Search Program and Files lalu ketikkan “device manager” (tanpa tanda petik). Kemudian bagian Control Panel akan muncul halaman Device Manager, selanjutnya klik untuk menjalankan.
	\item Cari yang bernama Unknown device yang berada pada bagian Other device, biasanya ada tanda seru berwarna kuning, itu disebabkan karena penginstallan tidak berjalan dengan sempurna.
	\item Klik kanan pada “Unknown device” kemudian pilih Update Driver Software.
	\item Pilih Browse my computer for driver software.
	\item Arahkan lokasi folder ke folder ..arduino-1.0.5 drivers. Pastikan check-box lalu centang include subfolders. Klik Next untuk melanjutkan instalasi driver.
	\item Kemudian lanjutkan dengan mengklik Install pada tampilan Windows Security.
	\item Jika instalasi driver berhasil maka akan muncul Windows has successfully updated your driver software.
	\item Perhatikan dan ingat nama COM Arduino Uno, karena nama COM ini yang akan digunakan untuk meng-upload program nantinya.
\end{enumerate}

\subsection{Soal No. 3}
Jelaskan bagaimana cara membaca baudrate dan port dari komputer yang sudah terinstall driver!

\hfill \break
\textbf{Membaca Port dari Komputer}

\begin{enumerate}
	\item Hubungkan modul TX-RX serial dengan komputer melalui serial port menggunakan DB9 cable extension.
	\item Buka Hyper Terminal dengan menekan start kemudian All progams lalu Accessories kemudian Communications lalu Hyper Terminal.
	\item Ketik nama untuk Connection Description, misal coba, kemudian tekan OK.
	\item Pada Connect to, pilihlah COM port yang dipakai di Connect using, kemudian tekan OK.
	\item Masukkan nilai-nilai port settingnya, sesuai dengan DCE-nya. Kemudian tekan OK.
\end{enumerate}



\subsection{Soal No. 4}
Jelaskan sejarah library pyserial!

\hfill \break
PySerial adalah library/modul Python siap-pakai dan gratis yang dibuat untuk memudahkan kita dalam membuat program komunikasi data serial RS232 dalam bahasa Python.
Jika modul USB-2REL dapat kita kontrol dengan mudah menggunakan Python dan PyUSB (lihat pembahasannya di sini dan di sini), maka modul SER-2REL juga dapat kita kontrol dengan mudah menggunakan Python dengan bantuan modul PySerial.

\subsection{Soal No. 5}
Jelaskan fungsi-fungsi apa saja yang dipakai dari library pyserial!

\hfill \break
Fungsi-fungsi yang dipakai dari library PySerial, yaitu:
\begin{enumerate}
	\item Serial - fungsi ini untuk membuka port serial.
	\item write(data) - fungsi ini menulis data lewat port serial.
	\item readline() - fungsi ini membaca sebuah string dari port serial.
	\item read(size) - fungsi ini untuk membaca jumlah byte dari port serial.
	\item close() - fungsi ini untuk menutup port serial.
\end{enumerate}

\subsection{Soal No. 6}
Jelaskan kenapa butuh perulangan dan tidak butuh perulangan dalam membaca serial!

\hfill \break
Pada saat membaca serial di Arduino diperlukan perulangan agar bisa membaca data secara berulang kali sehingga data yang muncul banyak. Sedangkan apabila tidak membutuhkan perulangan maka Arduino hanya akan membaca data sekali saja.

\subsection{Soal No. 7}
Jelaskan bagaimana cara membuat fungsi yang mengunakan pyserial!

\hfill \break
Fungsi yang berada pada Python, dibuat dengan nama kata kunci def kemudian diikuti dengan nama fungsinya pada pyhton.
Seperti halnya dengan blok kode yang lain, kita juga harus memberikan identasi untuk menuliskan isi fungsi.

\subsection{Cek Plagiat}
\begin{figure}[H]
	\includegraphics[width=10cm]{figures/5/1174031/Teori/Plagiat.png}
	\centering
	\caption{Hasil cek plagiat.}
\end{figure}
%%%%%%%%%%%%%%%%%%%%%%%%%%%%%%%%%%%%%%%%%%%%%%%%%%%%%%%%%%%%%%%%%5
\section{Damara Benedikta}
\subsection{Apa itu fungsi device manager di windows dan folder /dev di linux}
Windows
Device Manager merupakan sebuah Panel Kontrol dalam sistem operasi Microsoft Windows.
 Yang memungkinkan pengguna untuk dapat melihat dan mengontrol perangkat keras (hardware) yang terpasang pada komputer. 
 Saat beberapa bagian perangkat keras (hardware) tidak dapat berfungsi,maka perangkat keras yang terkait akan disorot oleh pengguna.
 Daftar perangkat keras dapat disortir berdasarkan berbagai kriteria.

Untuk setiap perangkat, pengguna dapat:
\begin{itemize}
     \item Menyediakan driver perangkat sesuai dengan Model Driver pada Windows
     \item Aktifkan atau menonaktifkan perangkat
     \item Memberi tahu Windows untuk mengabaikan perangkat yang tidak berfungsi
     \item Melihat sifat teknis lainnya
\end{itemize}
Device Manager diperkenalkan dengan Windows 95 dan kemudian ditambahkan ke dalam Windows 2000. Dalam versi berbasis NT, ini dimasukkan sebagai snap-in Konsol Manajemen Microsoft.

Linux
/ dev adalah lokasi file khusus atau perangkat. Ini adalah direktori yang sangat menarik yang menyoroti satu aspek penting dari sistem file Linux - semuanya adalah file atau direktori. yang fungsinya untuk menyimpan sebuah konfigurasi device ataupun hardware dari system

\subsection{langkah-langkah instalasi driver dari arduino}
Hubungkan sistem minimun Arduino Uno dan komputer dengan kabel USB type B
selanjutnya  pada bagian paling kanan didesktop PC kalian, akan muncul popup “Installing device driver software". Dalam sistem operasi Windows tidak tersedia driver untuk menginstal Arduino Uno sehuingga pada proses instalasinya akan dilakukan secara manual.
Yang pertama kalian buka terlebih dahulu Device Manager, caranya adalah pada bagian Search Program and Files kemudian ketikkan “device manager” (tanpa tanda petik), Pada bagian Control Panel akan muncul Device Manager,selanjutnya kalian klik untuk dapat menjalankannya.
Setelah itu kalian cari Unknown device pada bagian Other device, yang seringkali terdapat tanda seru berwarna kuning, tanda tersebut idisebabkan karena penginstallan yang tidak berjalan dengan sempurna.
Selanjutnya kalian Klik kanan pada “Unknown device” kemudian pilihlah Update Driver Software.
Kemudian kalian pilih Browse my computer for driver software.
Arahkan lokasi folder ke folder."\"arduino-1.0.5"\"drivers. Pastikan check-box kemudian centang include subfolders. Kemudian kalian  Klik Next untuk melanjutkan instalasi driver.
Setelah itu  lanjutkan dengan mengklik Install pada tampilan Windows Security.
Jika instalasi driver telah berhasil terinstal maka akan muncul Windows has successfully updated your driver software.
Perhatikan dan ingat nama COM Arduino Uno, karena nama COM  tersebut nantinya akan berguna untuk mengupload sebuah program selanjutnya. .
\subsection{Jelaskan bagaimana cara membaca baudrate dan port dari komputer yang sudah terinstal driver}
Diabawah ini merupakan cara cara untuk membaca baudrate dan port dari komputer yang sudah terinstal driver :
\begin{itemize}
	\item Pertama yang dilakukan Sambungkan port USB arduino dengan port USB pc kalian 
	\item Kemudian buka software arduino pada pc kalian 
	\item Setelah itu, pilih tipe arduino yang akan kalian gunakan
	\item Kemudian kalian pilih  serial port yang aktif  
	\item Selanjutnya untuk memasukkan program tersebut pada arduino, kalian klik tombol "upload"
	\item Setelah proses upload selesai, kemudian kalian buka fitur serial monitor
	\item Lalu kalian sesuaikan Baudrate pada serial monitor tersebut dengan Baudrate yang terdapat didalam program
\end{itemize}

\subsection{Jelaskan sejarah library pyserial}
Library pyserial merupakan sebuah modul yang memudahkan kita untuk  merangkum akses untuk membuat data port serial. Dimana library pyserial menyediakan backends untuk Python yang berjalan di Windows, Linux, BSD , Jython dan IronPython .  
Akses ke pengaturan port melalui properti Python.
Dukungan untuk berbagai ukuran byte, bit stop, paritas dan kontrol aliran dengan RTS / CTS dan / atau Xon / Xoff.
dapat bekerja dengan atau tanpa menerima batas waktu.
library pyserial juga mendukung jenis File seperti API dengan "read" dan "write" ("readline" dll. Juga didukung).
File-file dalam paket tersebut merupakan   Python murni.
Port didalam library pyserial ini diatur untuk transmisi biner.

\subsection{Jelaskan fungsi-fungsi apa saja yang dipakai dari library pyserial}
Serial – fungsi serial merupakan fungsi yang digunakan  untuk membuka port serial
Write(data) –fungsi write merupakan fungsi yang digunakan untuk menulis data melalui port serial
Readline() – fungsi readline merupakan fungsi untuk membaca string dari port serial
Read(size) – fungsi read merupakan fungsi untuk membaca jumlah byte dari port serial
Close() – fungsi close merupakan fungsi yang digunakan untuk menutup port serial 

\subsection{Jelaskan kenapa butuh perulangan dalam tidak butuh perulangan dalam membaca serial}
Dalam sebuah bahasa pemrograman  Perualangan berfungsi untuk memerintahkan sebuah komputer melakukan sesuatu secara berulang-ulang. Terdapat dua jenis perualangan dalam bahasa pemrograman python, yaitu perulangan dengan for dan while.
Dimana perulangan for disebut counted loop (perulangan yang terhitung), sementara perulangan while disebut uncounted loop (perulangan yang tak terhitung). Perbedaan dari kedua perulangan tersebut  yang terlihat adalah pada perulangan for digunakan untuk melakukan perulangan kode yang sudah diketahui banyak perulangannya. Sedangkan perulangan while digunakan pada perulangan yang memiliki syarat dan tidak tentu berapa banyak perulangannya.
Perulangan diperlukan agar dapat membaca data secara berulang kali. sehingga data-data dapat terbaca beberapa kali tidak hanya sekali saja. Sedangkan apabila tidak memakai perulangan maka data akan terbaca satu kali saja.

\subsection{Jelaskan bagaimana cara membuat fungsi yang mengunakan pyserial}
Berikut merupakan contoh penggunaan fungsi yang menggunakan pyserial
\lstinputlisting[firstline=5, lastline=15]{src/5/1174012/T1174012.py}

\subsection{plagiarisme}
\begin{figure}[h]
\centering
\includegraphics[scale=0.2]{figures/5/1174012/SS2.png}
\caption{plagiarisme}
\label{fig:plagiat}
\end{figure}

%%%%%%%%%%%%%%%%%%%%%%%%%%%%%%%%%%%%%%%%%%%%%%%%%%%%%%%%%%%%%%%%%%%%%%%%%%%%%%%%%%%%%

\section{Dwi Septiani Tsaniyah}
{\Large \textbf{Pemahaman Teori}}
\subsection{Soal No. 1}

Apa itu fungsi device manager di windows dan folder /dev di linux?

\hfill \break
Fungsi device manager dan folder /dev itu berfungsi untuk mengetahui device apa saja yang telah terinstal di leptop anda serta mengetahui port yang digunakan oleh device tersebut.

\hfill \break
Fungsi device manager antara lain :
\begin{enumerate}
	\item Menunjukkan status mengenai suatu perangkat keras.
	\item Menunjukkan informasi detail mengenai suatu perangkat keras.
	\item Mengelola driver perangkat keras.
	\item Menonaktifkan dan mengaktifkan perangkat keras.
	\item Mengidentifikasi konflik antar perangkat keras.
	\item Memberitahukan terjadinya masalah pada perangkat keras.
\end{enumerate}

\subsection{Soal No. 2}
Jelaskan langkah-langkah instalasi driver dari arduino!

\subsection{Jelaskan langkah-langkah instalasi driver dari arduino}
\begin{enumerate}
    \item Cara Auto
    \begin{itemize}
        \item Pertama Hubungkan sistem minimum Arduino Uno ke komputer dengan kabel USB type B(kabel Printer)
        \begin{figure}[H]	
            \includegraphics[width=5cm]{figures/5/1174003/teori/kabel.jpg}
            \centering
            \caption{Membuat file csv}
        \end{figure}

        \item Lalu pada bagian kanan didesktop PC anda, akan muncul popup “Installing device driver software” seperti pada gambar dibawah ini.
        \begin{figure}[H]	
            \includegraphics[width=5cm]{figures/5/1174003/teori/1.png}
            \centering
            \caption{Membuat file csv}
        \end{figure}

        \item Tunggu hingga selesai.
        \item Jika sudah selesai anda bisa mengecheck di device manager.
        \begin{figure}[H]	
            \includegraphics[width=5cm]{figures/5/1174003/teori/2.png}
            \centering
            \caption{Membuat file csv}
        \end{figure}
    \end{itemize}

    \item Cara Manual

    \begin{itemize}
        \item Penginstalan secara manual akan dilakukan jika penginstalan secara auto gagal dilakukan.
        \item Buka Device Manager, caranya pada bagian Search Program and Files lalu ketikkan “device manager”, perhatikan gambar dibawah ini. Pada bagian Control Panel akan muncul Device Manager, klik untuk menjalankan.
            \begin{figure}[H]	
                \includegraphics[width=5cm]{figures/5/1174003/teori/3.png}
                \centering
                \caption{Membuat file csv}
            \end{figure}

        \item Cari Unknown device pada bagian Other device, biasanya terdapat tanda seru berwarna kuning, itu disebabkan karena penginstallan tidak berjalan dengan sempurna.
        \begin{figure}[H]	
            \includegraphics[width=5cm]{figures/5/1174003/teori/4.png}
            \centering
            \caption{Membuat file csv}
        \end{figure}

        \item Klik kanan pada “Unknown device” kemudian pilih Update Driver Software.
        \begin{figure}[H]	
            \includegraphics[width=5cm]{figures/5/1174003/teori/5.png}
            \centering
            \caption{Membuat file csv}
        \end{figure}

        \item Pilih Browse my computer for driver software.
        \begin{figure}[H]	
            \includegraphics[width=5cm]{figures/5/1174003/teori/6.png}
            \centering
            \caption{Membuat file csv}
        \end{figure}

        \item Arahkan lokasi folder ke folder ..arduino-1.0.5 drivers. Pastikan check-box lalu centang include subfolders. Klik Next untuk melanjutkan instalasi driver.
        \begin{figure}[H]	
            \includegraphics[width=5cm]{figures/5/1174003/teori/7.png}
            \centering
            \caption{Membuat file csv}
        \end{figure}

        \item Kemudian lanjutkan dengan mengklik Install pada tampilan Windows Security.
        \begin{figure}[H]	
            \includegraphics[width=5cm]{figures/5/1174003/teori/8.png}
            \centering
            \caption{Membuat file csv}
        \end{figure}

        \item Jika instalasi driver berhasil maka akan muncul Windows has successfully updated your driver software.
        \begin{figure}[H]	
            \includegraphics[width=5cm]{figures/5/1174003/teori/9.png}
            \centering
            \caption{Membuat file csv}
        \end{figure}

        \item Perhatikan dan ingat nama COM Arduino Uno, karena nama COM ini yang akan digunakan untuk meng-upload program nantinya.
        \begin{figure}[H]	
            \includegraphics[width=5cm]{figures/5/1174003/teori/10.png}
            \centering
            \caption{Membuat file csv}
        \end{figure}
        \end{itemize}
\end{enumerate}

\subsection{Soal No. 3}
Jelaskan bagaimana cara membaca baudrate dan port dari komputer yang sudah terinstall driver!

Untuk baudrate itu bisa dicek melalui arduino IDE, kemudian untuk mengecheck port bisa dilakukan dengan device manager

\subsection{Soal No. 4}
Jelaskan sejarah library pyserial!

Modul ini merangkum akses untuk port serial. Ini menyediakan backends untuk Python yang berjalan di Windows, Linux, BSD (mungkin sistem yang mendukung POSIX), Jython dan IronPython (.NET dan Mono). Modul bernama "serial" secara otomatis memilih backend yang sesuai. Antarmuka berbasis kelas yang sama pada semua platform yang didukung.

\subsection{Soal No. 5}
Jelaskan fungsi-fungsi apa saja yang dipakai dari library pyserial!

Fungsi-fungsi yang dipakai dari library PySerial, yaitu:
\begin{enumerate}
	\item Serial - fungsi ini untuk membuka port serial.
	\item write(data) - fungsi ini menulis data lewat port serial.
	\item readline() - fungsi ini membaca sebuah string dari port serial.
	\item read(size) - fungsi ini untuk membaca jumlah byte dari port serial.
	\item close() - fungsi ini untuk menutup port serial.
\end{enumerate}

\subsection{Soal No. 6}
Jelaskan kenapa butuh perulangan dan tidak butuh perulangan dalam membaca serial!
\begin{itemize}
\item Perulangan for disebut counted loop (perulangan yang terhitung), sementara perulangan while disebut uncounted loop (perulangan yang tak terhitung). Perbedaannya adalah perulangan for biasanya digunakan untuk mengulangi kode yang sudah diketahui banyak perulangannya. Sementara while untuk perulangan yang memiliki syarat dan tidak tentu berapa banyak perulangannya.
\end{itemize}
\subsection{Soal No. 7}
Jelaskan bagaimana cara membuat fungsi yang mengunakan pyserial!

\subsection{Jelaskan bagaimana cara membuat fungsi yang mengunakan pyserial}
Berikut merupakan contoh penggunaan fungsi yang menggunakan pyserial
\lstinputlisting[firstline=8, lastline=15]{src/5/1174003/T1174003.py}

%%%%%%%%%%%%%%%%%%%%%%%%%%%%%%%%%%%%%%%%%%%%%%%%%%%%%%%%%%%%%%%%%%%

\section{Muh. Rifky Prananda}
{\Large \textbf{Pemahaman Teori}}
\subsection{Soal No. 1}
Apa itu fungsi device manager di windows dan folder /dev di linux?

\hfill \break
Fungsi sebuah device manager yaitu diantaranya :
\begin{enumerate}
	\item Memerlihatkan atau menunjukkan status suatu hardware.
	\item Memperlihatkan atau menunjukkan informasi detail suatu hardware.
	\item Dapat mengelola driver hardware
	\item Enable dan disable suatu hardware
	\item Dapat mengidentifikasi konflik antar perangkat keras.
\end{enumerate}

\hfill \break
Folder /dev berisi file device, baik device blok maupun device karakter. Di dalamnya setidaknya ada file biner yang bernama MAKEDEV yang dapat membuat suatu device secara manual.

\subsection{Soal No. 2}
Jelaskan langkah-langkah instalasi driver dari arduino!

\hfill \break
Berikut adalah sebuah langkah instalasi driver dari Arduino UNO di Windows:

\begin{enumerate}
	\item yang pertama itu, Hubungkan sistem minimun Arduino Uno ke komputer dengan kabel USB type B atau kabel Printer.
	\item Selanjutnya pada bagian kanan didesktop PC, akan muncul popup “Installing device driver software”.
	\item Sistem operasi Windows tidak menyediakan sebuah driver untuk Arduino Uno.
	\item Selanjutnya Buka Device Manager, caranya yaitu pada bagian Search Program and Files dan ketikkan “device manager” (tanpa tanda petik). Selanjutnya pada bagian Control Panel akan muncul halaman Device Manager, selanjutnya klik untuk menjalankan.
	\item Cari yang bernama Unknown device yang berada pada bagian Other device, biasanya ada tanda seru berwarna kuning, itu disebabkan karena penginstallan tidak berjalan dengan sempurna.
	\item Klik kanan pada “Unknown device” lalu setelahnya pilih Update Driver Software.
	\item Pilih Browse my computer for driver software.
	\item Arahkan lokasi folder ke folder arduino-1.0.5 drivers. Pastikan check-box lalu centang include subfolders. Klik Next untuk melanjutkan instalasi driver.
	\item Kemudian yaitu lanjutkan dengan mengklik Install pada tampilan Windows Security.
	\item Jika instalasi driver sudah berhasil maka selanjutnya akan muncul Windows has successfully updated your driver software.
	\item Perhatikan kembali dan ingat nama COM Arduino Uno, karena nama COM ini yang akan digunakan untuk meng-upload program nantinya.
\end{enumerate}

\subsection{Soal No. 3}
Jelaskan bagaimana cara membaca baudrate dan port dari komputer yang sudah terinstall driver!

\hfill \break
\textbf{Membaca Port dari Komputer}

\begin{enumerate}
	\item Yang pertama hubungkan modul TX-RX serial dengan komputer melalui serial port menggunakan DB9 cable extension.
	\item Selanjutnya buka Hyper Terminal dengan menekan start kemudian All progams lalu Accessories kemudian Communications lalu pilih Hyper Terminal.
	\item Ketikkan nama buat Connection Description, misal coba, kemudian tekan OK.
	\item Pada Connect to, pilih yang COM port yang dipakai di Connect using, kemudian tekan OK.
	\item Selanjutnya masukkan nilai-nilai port settingnya, sesuai dengan DCE-nya. Kemudian tekan OK.
\end{enumerate}



\subsection{Soal No. 4}
Jelaskan sejarah library pyserial!

\hfill \break
PySerial adalah sebuah modul/library Python siap-pakai dan gratis yang sengaja dibuat untuk memudahkan kita dalam membuat suatu program komunikasi data serial RS232 dalam bahasa Python.
Jika modul USB-2REL bisa kita kontrol akan dengan mudah menggunakan Python dan PyUSB (lihat pembahasannya di sini dan di sini), maka modul SER-2REL juga dapat kita kontrol dengan mudah menggunakan Python dengan bantuan modul PySerial.

\subsection{Soal No. 5}
Jelaskan fungsi-fungsi apa saja yang dipakai dari library pyserial!

\hfill \break
Berikut beberapa fungsi yang dipakai dari library PySerial, diantaranya:
\begin{enumerate}
	\item write(data) - fungsi ini menulis data lewat port serial.
	\item Serial - fungsi ini untuk membuka port serial.
	\item readline() - fungsi ini dapat membaca sebuah string dari port serial.
	\item read(size) - fungsi ini bisa untuk membaca jumlah byte dari port serial.
	\item close() - fungsi ini dapat menutup port serial.
\end{enumerate}

\subsection{Soal No. 6}
Jelaskan kenapa butuh perulangan dan tidak butuh perulangan dalam membaca serial!

\hfill \break
Pada saat membaca serial di Arduino diperlukan sebuah perulangan agar dapat membaca data secara berulang kali sehingga data yang muncul akan banyak. Sedangkan lagi apabila tidak membutuhkan suatu perulangan maka Arduino hanya membaca data sekali.

\subsection{Soal No. 7}
Jelaskan bagaimana cara membuat fungsi yang mengunakan pyserial!

\hfill \break
Fungsi yang terdapat di Python, dibikin dengan menggunakan nama kata kunci def lalu diikuti dengan nama fungsinya pada pyhton.
Sama halnya dengan blok kode program yang lain, kita juga harus dapat memberikan identasi untuk menuliskan isi fungsi.
%%%%%%%%%%%%%%%%%%%%%%%%%%%%%%%%%%%%%%%%%%%%%%%%%%%%%%%%%%%%%%%%%%%%%%%%%%%%%%%%%%%%%
\section{Felix Setiawan Lase}
{\Large \textbf{Pemahaman Teori}}
\subsection{Soal No. 1}
Apa itu fungsi device manager di windows dan folder /dev di linux?

\hfill \break
Device Manager  dapat  membantu dalam mengelola  semua hardware yang terpasang  dalam suatu sistem Windows. 
 Berikut fungsi kegunaan Device Manager antara lain adalah :
\begin{enumerate}
	\item Menunjukkan status suatu hardware.
	\item Menunjukkan informasi detil suatu hardware.
	\item Mengelola driver hardware
	\item Disable dan Enable hardware
	\item Mengidentifikasi konflik antar perangkat keras.
\end{enumerate}

\hfill \break
Folder /bin merupakan isi program binner yang harus ada apabila sistem yang dipasang dalam mode single-user, dan juga  ada beberapa program penting seperti bash.

\subsection{Soal No. 2}
Jelaskan langkah-langkah instalasi driver dari arduino!

\hfill \break
Berikut ini adalah langkah-langkah instalasi driver dari Arduino UNO di Windows:

\begin{enumerate}
	\item Langkah pertama Hubungkan sistem minimun Arduino Uno ke komputer dengan kabel USB .
	\item Lalu pada bagian kanan didesktop PC , akan muncul popup “Installing device driver software” seperti pada gambar dibawah ini.
	\item Kemudian jika sistem  operasi Windows tidak menyediakan driver untuk Arduino Uno,maka harus  melakukan instalasinya harus dilakukan secara manual.
	\item Lalu  Buka Device Manager,  dengan cara pada bagian Search Program and Files lalu ketikkan “device manager” (tanpa tanda petik). 
	\item kemudian Pada bagian COntrol Panel akan muncul Device Manager, lalu klik untuk menjalankan program tersebut.
	\item Setelah itu cari  Unknown device pada bagian Other device, biasanya terdapat tanda seru berwarna kuning, itu disebabkan karena penginstallan gagal.
	\item Klik kanan pada bagian  “Unknown device” kemudian pilih Update Driver Software.
	\item kemudian cari Browse my computer for driver software pada laptop anda.
	\item setelah itu lakukan dengan mengklik Install pada tampilan Windows Security.
	\item Jika instalasi driver pada laptop anda berhasil maka akan muncul Windows has successfully updated your driver software.
	\item Perhatikan dan ingat nama COM Arduino Uno, karena nama COM ini yang akan digunakan untuk meng-upload program nantinya
\end{enumerate}

\subsection{Soal No. 3}
Jelaskan bagaimana cara membaca baudrate dan port dari komputer yang sudah terinstall driver!

\hfill \break
\textbf{Membaca Port dari Komputer}

\begin{enumerate}
	\item Hubungkan modul TX-RX serial dengan komputer melalui serial port menggunakan DB9 cable extension.
	\item Buka Hyper Terminal dengan menekan start kemudian All progams lalu Accessories kemudian Communications lalu Hyper Terminal.
	\item Ketik nama untuk Connection Description, misal coba, kemudian tekan OK.
	\item Pada Connect to, pilihlah COM port yang dipakai di Connect using, kemudian tekan OK.
	\item Masukkan nilai-nilai port settingnya, sesuai dengan DCE-nya. Kemudian tekan OK.
\end{enumerate}



\subsection{Soal No. 4}
Jelaskan sejarah library pyserial!

\hfill \break
PySerial adalah library/modul Python siap-pakai dan gratis yang dibuat untuk memudahkan kita dalam membuat program komunikasi data serial RS232 dalam bahasa Python.

\subsection{Soal No. 5}
Jelaskan fungsi-fungsi apa saja yang dipakai dari library pyserial!

\hfill \break
Fungsi-fungsi yang dipakai dari library PySerial, yaitu:
\begin{enumerate}
	\item Serial - fungsi ini untuk membuka port serial.
	\item write(data) - fungsi ini menulis data lewat port serial.
	\item readline() - fungsi ini membaca sebuah string dari port serial.
	\item read(size) - fungsi ini untuk membaca jumlah byte dari port serial.
	\item close() - fungsi ini untuk menutup port serial.
\end{enumerate}

\subsection{Soal No. 6}
Jelaskan kenapa butuh perulangan dan tidak butuh perulangan dalam membaca serial!

\hfill \break
Pada saat membaca serial di Arduino diperlukan perulangan agar dapat membaca data secara berulang kali sehingga data yang muncul banyak. Sedangkan apabila tidak membutuhkan perulangan maka Arduino hanya akan membaca data sekali saja.

\subsection{Soal No. 7}
Jelaskan bagaimana cara membuat fungsi yang mengunakan pyserial!

\hfill \break
Fungsi yang berada pada Python, dibuat dengan nama kata kunci def kemudian diikuti dengan nama fungsinya pada pyhton.
Seperti halnya dengan blok kode yang lain, kita juga harus memberikan identasi untuk menuliskan isi fungsi.

\lstinputlisting[caption = Fungsi yang menggunakan pyserial., firstline=1, lastline=8]{src/5/1174026/Teori/serial.py}

\begin{figure}[H]
	\includegraphics[width=10cm]{figures/5/1174026/Teori/hasil.png}
	\centering
	\caption{Hasil pembuatan fungsi pyserial.}
\end{figure}

\subsection{Cek Plagiat}
\begin{figure}[H]
	\includegraphics[width=10cm]{figures/5/1174026/Teori/Plagiat.png}
	\centering
	\caption{Hasil cek plagiat.}
\end{figure}



\section{Muhammad Fahmi}
\subsection{Pemahaman Teori}
\subsubsection{Soal No. 1}
Apa itu fungsi device manager di windows dan folder /dev di linux.

\textbf{Pengertian Device Manager}
Device manager ialah perangkat lunak yang berfungsi untuk menampilkan seluruh perangkat keras yang di-inisialisasi atau dikenali oleh sebuah sistem operasi Windows. Device Manager juga membantu untuk mengelola semua perangkat keras yang terpasang dan terdeteksi dalam sistem Windows. Perangkat keras tersebut bisa berupa harddisk, kartu VGA, sound, keyboard, perangkat USB dan yang lainnya.

\textbf{Fungsi Device Manager}
Device Manager memiliki fungsi-fungsi antara lain :
\begin{enumerate}
	\item Menunjukkan status mengenai suatu perangkat keras.
	\item Menunjukkan informasi detail mengenai suatu perangkat keras.
	\item Mengelola driver perangkat keras.
	\item Menonaktifkan dan mengaktifkan perangkat keras.
	\item Mengidentifikasi konflik antar perangkat keras.
	\item Memberitahukan terjadinya masalah pada perangkat keras.
\end{enumerate}

\textbf{Folder /dev di linux}
Folder /dev merupakan representasi dari drive yang terhubung ke sistem operasi Linux dan oleh sistem dianggap sebagai file-file direktori. Biasanya sering ditampilkan direktori seperti /dev/sda1 yang mewakili Drive SATA pertama dalam sistem.

\subsubsection{Soal No. 2}
Jelaskan langkah-langkah instalasi driver dari arduino.

Berikut ini adalah langkah-langkah instalasi driver dari Arduino UNO di Windows:

\begin{enumerate}
	\item Pertama pastikan Arduino IDE telah terinstall pada PC anda.
	\item Hubungkan port USB Arduino Uno ke port USB PC.
	\item Kemudian PC anda akan mendeteksi perangkat baru yang terpasang dan akan muncul oemberitahuan seperti ini : 
	\begin{figure}[H]
		\includegraphics[width=10cm]{figures/5/1174021/Teori/1.png}
		\centering
	\end{figure}
	\item Karena Arduino Uno baru pertama kali terpasang, maka akan muncul pemberitahuan seperti ini :
	\begin{figure}[H]
		\includegraphics[width=10cm]{figures/5/1174021/Teori/2.png}
		\centering
	\end{figure}
	\item Buka windows start lalu cari Device Manager lalu klik.
	\begin{figure}[H]
		\includegraphics[width=10cm]{figures/5/1174021/Teori/3.png}
		\centering
	\end{figure}
	\item Setelah Device Manager yang ada klik terbuka, silahkan cari Unknown Device yang berada di Other Device.
	\begin{figure}[H]
		\includegraphics[width=10cm]{figures/5/1174021/Teori/4.png}
		\centering
	\end{figure}
	\item Kemudian klik kanan, lalu pilih Update Driver Software.
	\begin{figure}[H]
		\includegraphics[width=10cm]{figures/5/1174021/Teori/5.png}
		\centering
	\end{figure}
	\item Setelah itu muncul halaman baru, lalu pilih Browse my computer for driver software.
	\begin{figure}[H]
		\includegraphics[width=10cm]{figures/5/1174021/Teori/6.png}
		\centering
	\end{figure}
	\item Lalu cari folder yang terinstall Arduino IDE dengan mengklik browse. Kemudian klik Next.
	\begin{figure}[H]
		\includegraphics[width=10cm]{figures/5/1174021/Teori/7.png}
		\centering
	\end{figure}
	\item Kemudian Windows akan mencari dan menginstall driver yang berada pada folder tersebut.
	\begin{figure}[H]
		\includegraphics[width=10cm]{figures/5/1174021/Teori/8.png}
		\centering
	\end{figure}
	\item Setelah itu akan muncul halaman baru, lalu klik Install.
	\begin{figure}[H]
		\includegraphics[width=10cm]{figures/5/1174021/Teori/9.png}
		\centering
	\end{figure}
	\item Jika berhasil terinstall maka akan muncul halaman seperti ini :
	\begin{figure}[H]
		\includegraphics[width=10cm]{figures/5/1174021/Teori/10.png}
		\centering
	\end{figure}
\end{enumerate}

\subsection{Soal No. 3}
Jelaskan bagaimana cara membaca baudrate dan port dari komputer yang sudah terinstall driver.

\hfill \break
\textbf{Membaca Baudrate dari Komputer}
\begin{enumerate}
	\item Pertama buka windows start. Cari Device Manager, lalu klik.
	\begin{figure}[H]
		\includegraphics[width=10cm]{figures/5/1174021/Teori/11.png}
		\centering
	\end{figure}
	
	\item Kemudian pilih ''Ports (COM \& LPT)''.
	\begin{figure}[H]
		\includegraphics[width=10cm]{figures/5/1174021/Teori/12.png}
		\centering
	\end{figure}
	
	\item Klik dua kali pada COM yang terhubung.
	\item Pilih tab ''Port Settings'', lalu lihat di ''Bit per second''.
\end{enumerate}


\hfill \break
\textbf{Membaca Port dari Komputer}

\begin{enumerate}
	\item Pertama buka ''Start''. Cari ''Device Manager'', lalu klik.
	\begin{figure}[H]
		\includegraphics[width=10cm]{figures/5/1174006/Teori/d1.png}
		\centering
	\end{figure}

	\item Kemudian pilih ''Ports (COM \& LPT)''.
	\begin{figure}[H]
		\includegraphics[width=10cm]{figures/5/1174006/Teori/d3.png}
		\centering
	\end{figure}

	\item Port dari Arduino telah terbaca oleh PC.
	\begin{figure}[H]
		\includegraphics[width=10cm]{figures/5/1174006/Teori/d2.png}
		\centering
	\end{figure}
\end{enumerate}


\subsection{Soal No. 4}
Jelaskan sejarah library pyserial.

\hfill \break
PySerial adalah sebuah paket yang disedakan Python untuk menfasilitasi komunikasi serial antara PC dengan perangkat keras eksternal. PySerial juga menyediakan antarmuka untuk berkomunikasi melalui protokol komunikasi serial. Kemudian, Komunikasi serial adalah sebuah protokol komunikasi komputer tertua. Protokol komunikasi serial mendahului spesifikasi USB yang digunakan oleh beberapa komputer dan perangkat keras lain seperti mouse, keyboard, dan webcam. USB adalah singkatan dari Universal Serial Bus. USB dan dibangun di atas dan memperluas antarmuka komunikasi serial asli.

\subsection{Soal No. 5}
Jelaskan fungsi-fungsi apa saja yang dipakai dari library pyserial.

\hfill \break
Fungsi-fungsi yang dipakai dari library PySerial, yaitu:
\begin{enumerate}
	\item Serial - ini berfungsi untuk membuka port serial.
	\item write(data) - ini berfungsi menulis data lewat port serial.
	\item readline() - ini berfungsi membaca sebuah string dari port serial.
	\item read(size) - ini berfungsi untuk membaca jumlah byte dari port serial.
	\item close() - ini berfungsi untuk menutup port serial.
\end{enumerate}

\subsection{Soal No. 6}
Jelaskan kenapa butuh perulangan dan tidak butuh perulangan dalam membaca serial.

\hfill \break
Pada saat membaca serial di Arduino sangat diperlukan perulangan, agar bisa membaca data secara berulang kali sehingga hasil atau data yang muncul nantinya banyak. Sedangkan apabila tidak membutuhkan perulangan maka Arduino hanya akan membaca data sekali saja.

\subsection{Soal No. 7}
Jelaskan bagaimana cara membuat fungsi yang mengunakan pyserial.

\lstinputlisting[caption = Fungsi yang menggunakan pyserial, firstline=1, lastline=7]{src/5/1174021/Teori/1174021.py}