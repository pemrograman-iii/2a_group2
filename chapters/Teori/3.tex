\section{Muhammad Tomy Nur Maulidy}
{\Large \textbf{Pemahaman Teori}}
\subsection{Soal No. 1}
Apa itu fungsi, inputan fungsi dan kembalian fungsi dengan contoh kode program lainnya.

\hfill \break
Fungsi memiliki tujuan agar kita dapat memecah program besar menjadi sub-sub program yang lebih sederhana.pada masing-masing  fitur pada program dapat dibuat dalam satu fungsi. Pada saat kita membutuhkan suatu fitur maka kita tinggal memanggil fungsi yang telah kita buat. Fungsi pada python dibuat dengan menggunakan kata kunci def dan diikuti dengan nama fungsi yang telah kita buat seperti contoh dibawah ini :
 \lstinputlisting[firstline=10, lastline=10]{src/3/1174031/chapter3/1174031.py}
Inputan fungsi merupakan masukan yang kita berikan pada program dan program akan menampilkan hasil dari inputan yang telah kita masukkan atau akan menampilkan hasil pada proses selanjutnya. contoh dari inputan fungsi sebagai berikut :
 \lstinputlisting[firstline=11, lastline=11]{src/3/1174031/chapter3/1174031.py}
Pengembalian fungsi memiliki tujuan untuk mengembalikan nilai dari hasil yang telah di proses. Dalam hal ini menggunakan kata kunci return yang diikuti dengan nilai atau variabel yang akan dikembalikan.
 \lstinputlisting[firstline=10, lastline=14]{src/3/1174031/chapter3/1174031.py}

\subsection{Soal 2}
Apa itu paket dan cara pemanggilan paket atau library dengan contoh kode program lainnya.

\hfill \break
Library atau paket adalah modul-modul yang menyusun python. Modul-modul tersebut ditulis oleh berbagai orang dari seluruh dunia dan memiliki fungsi masing-masing untuk melakukan suatu hal. contoh kode programnya adalah sebagai berikut :
 \lstinputlisting[firstline=17, lastline=18]{src/3/1174031/chapter3/1174031.py}

\subsection{Soal 3}
Jelaskan Apa itu kelas, apa itu objek, apa itu atribut, apa itu method dan contoh kode program lainnya masing-masing.

\hfill \break
kelas adalah Prototype atau blueprint untuk menciptakan suatu object  yang mendefinisikan seperangkat atribut yang menjadi ciri objek kelas apa pun. Objek ialah instansiasi atau perwujudan dari sebuah kelas. Bila kelas adalah prototipenya, dan objek adalah hasil dari class jadinya. Atribut merupakan data dari anggota (variabel kelas, variabel contoh) dan metode, yang diakses dengan notasi titik. Sedangkan method fungsi yang didefinisikan di dalam suatu kelas.
 \lstinputlisting[firstline=21, lastline=40]{src/3/1174031/chapter3/1174031.py}

\subsection{Soal 4}
Jelaskan cara pemanggilan library kelas dari instansiasi dan pemakaiannya dengan contoh program lainnya.

\hfill \break
cara pemanggilan  library kelas dari instansiasi dan pemakaiannya adalah dengan cara meng-import library yang ada di dalam satu folder dengan menggunakan kode berikut :
 \lstinputlisting[firstline=43, lastline=51]{src/3/1174031/chapter3/1174031.py}

\subsection{Soal 5}
Jelaskan dengan contoh pemakaian paket dengan perintah from kalkulator import Penambahan disertai dengan contoh kode lainnya.

\hfill \break
contoh kodenya adalah sebagai berikut :
 \lstinputlisting[firstline=54, lastline=57]{src/3/1174031/chapter3/1174031.py}

\subsection{Soal 6}
Jelaskan dengan contoh kodenya, pemakaian paket fungsi apabila file library ada di dalam folder.

\hfill \break
 Pemakaian paket adalah perkumpulan fungsi-fungsi. contoh kodenya adalah sebagai berikut :
 \lstinputlisting[firstline=60, lastline=73]{src/3/1174031/chapter3/1174031.py}

\subsection{Soal 7}
Jelaskan dengan contoh kodenya, pemakaian paket kelas apabila file library ada di dalam folder.

\hfill \break
 \lstinputlisting[firstline=76, lastline=84]{src/3/1174031/chapter3/1174031.py}


\section{Dwi Yulianingsih}
\subsection{Soal 1}
Isi jawaban soal ke-1

Kalau mau dibikin paragrap \textbf{cukup enter aja}, tidak usah pakai \verb|par| dsb

%\subsection{Soal 2}
%Isi jawaban soal ke-2

%\subsection{Soal 3}
%Isi jawaban soal ke-3

\section{Harun Ar-Rasyid}
\subsection{Soal 1}
Isi jawaban soal ke-1

Kalau mau dibikin paragrap \textbf{cukup enter aja}, tidak usah pakai \verb|par| dsb

%\subsection{Soal 2}
%Isi jawaban soal ke-2

%\subsection{Soal 3}
%Isi jawaban soal ke-3

\section{Sri Rahayu}
\subsection{Soal 1}
Isi jawaban soal ke-1

Kalau mau dibikin paragrap \textbf{cukup enter aja}, tidak usah pakai \verb|par| dsb

%\subsection{Soal 2}
%Isi jawaban soal ke-2

%\subsection{Soal 3}
%Isi jawaban soal ke-3

\section{Doli Jonviter}
\subsection{Soal 1}
Isi jawaban soal ke-1

Kalau mau dibikin paragrap \textbf{cukup enter aja}, tidak usah pakai \verb|par| dsb

%\subsection{Soal 2}
%Isi jawaban soal ke-2

%\subsection{Soal 3}
%Isi jawaban soal ke-3

\section{Rahmatul Ridha}
\subsection{Soal 1}
Isi jawaban soal ke-1

Kalau mau dibikin paragrap \textbf{cukup enter aja}, tidak usah pakai \verb|par| dsb

%\subsection{Soal 2}
%Isi jawaban soal ke-2

%\subsection{Soal 3}
%Isi jawaban soal ke-3

\section{Tomy Prawoto}
\subsection{Soal 1}
Isi jawaban soal ke-1

Kalau mau dibikin paragrap \textbf{cukup enter aja}, tidak usah pakai \verb|par| dsb

%\subsection{Soal 2}
%Isi jawaban soal ke-2

%\subsection{Soal 3}
%Isi jawaban soal ke-3
